\chapter{Grundlagen Kategorientheorie}
\section{Kategorien und Morphismen}
\begin{Def}[Kategorie]
    Eine Kategorie \( e^x \) besteht aus den folgenden Daten:
    \begin{enumerate}
        \item Eine Klasse von Objekten \(\ob(\calC)\).
        \item Für alle \(x,y\in\ob(\calC)\) eine Klasse von Morphismen \(\Mor_\calC(x,y)\).
        \item Für alle \(x\in\ob(\calC)\) einen Identitätsmorphismus \(\id_x\in\Mor_\calC(x,x)\).
        \item Für alle \(x,y,z\in\ob(\calC)\) eine Verkettungsabbildung 
        \[\Mor_\calC(y,z)\times \Mor_\calC(x,y)\to \Mor_\calC(x,z),\ (g,f)\mapsto g\circ f.\]
    \end{enumerate}
    Dabei fordern wir, dass folgende Bedingungen erfüllt sind:
    \begin{enumerate}
        \item Für alle \(f\in\Mor_\calC(x,y)\) ist
        \(f\circ \id_x=f=\id_y\circ f.\)
        \item Für alle \(f\in\Mor_\calC(w,x), g\in\Mor_\calC(x,y)\) und \(h\in\Mor_\calC(y,x)\) ist 
        \[h\circ (g\circ f)=(h\circ g)\circ f.\]
    \end{enumerate}
    Wir nennen eine Kategorie \(\calC\) lokal klein, falls für alle \(x,y\in\ob(\calC)\) die Klasse \(\Mor_\calC(x,y)\) eine Menge ist, und klein, falls zusätzlich auch die Klasse \(\ob(\calC)\) eine Menge ist.
\end{Def}
\begin{Def}[Mono-,Epi-,Isomorphismus]
    Sei \(f\colon x\to y\) ein Morphismus in einer Kategorie \(\calC\).
    \begin{enumerate}
        \item Wir sagen \(f\) ist ein Monomorphismus, falls für alle \(z\in\calC\) die Abbildung 
        \[f_*\colon \Mor_\calC(z,x)\to\Mor_\calC(z,y),\; g\mapsto f\circ g\] injektiv ist.
        Dual dazu sagen wir \(f\) ist ein Epimorphismus, falls für alle \(z\in\calC\) die Abbildung
        \[f^*\colon\Mor_\calC(y,z)\to\Mor_\calC(x,z),\; g\mapsto g\circ f\] injektiv ist.
        \item Wir sagen \(f\) ist ein spaltender Monomorphismus, falls es einen Morphismus \(g\colon y\to x\) gibt mit \(g\circ f=\id_x\). Wir nenn dann \(g\) eine Retraktion voon \(f\)\\
        Dual dazu sagen wir \(f\) ist ein spaltender Epimorphismus falls es einen Morphismus \(g\colon y\to x\) mit \(f\circ g=\id_y\). Wir nennen dann \(g\) einen Schnitt von \(f\).
        \item Wir sagen \(f\) ist ein Isomorphismus falls es einen Morphismus \(f'\colon y\to x\) gibt der gleichzeitig Retraktion und Schnitt von \(f\) ist. So ein \(f'\) ist dann eindeutig und wir schreiben \(f^{-1}=f'\)
    \end{enumerate}
\end{Def}
\begin{Lemma}
    Sei \(f\colon x\to y\) ein Morphismus in einer Kategorie \(\calC.\) Dann sind die folgenden Bedinungen äquivalent:
    \begin{enumerate}
        \item \(f\) ist ein Isomorphismus.
        \item \(f\) ist sowohl ein Monomorphismus als auch ein spaltender Epimorphismus.
        \item \(f\) ist sowohl ein Epimorphisus als auch ein spaltender Monomorphismus.
         
        \item Für alle \(z\in\calC\) ist die Abbildung \(f_*\colon\Mor_\calC(z,x)\to\Mor_\calC(z,y)\) bijektiv.
        \item Für alle \(z\in\calC\) ist die Abbildung \(f^*\colon \Mor_\calC(y,z)\to \Mor_\calC(x,z)\) bijektiv.
       
    \end{enumerate}
\end{Lemma}
\begin{proof}
    Gelte 1). Dann ist für alle \(z\in\calC:\) \[f^{-1}_*f_*=(f^{-1}f)_*=\id_{\Mor_\calC(z,x)}\] also ist \(f_*\) injektiv und damit ist \(f\) ein Monomorphismus. Also gilt 2).\\
    Gelte 2). Dann sei \(g\colon y\to x\) ein Schnitt.
    Es gilt für \(z=x\) dass \(f_\star (\id_x)=f=f\circ g\circ f=f_*(g\circ f)\) also ist \(\id_x=g\circ f\). Damit gilt 1).
    Gelte 1). Dann ist für Retration und Schnitt \(g\colon y\to x\) und alle \(z\in\calC\):
    \(\id=(\id_x)_*=(g\circ f)_*=g_*\circ f_*\) und analog
    \(\id=f_*\circ g_*\). Also ist \(f_*\) injektiv und surjektiv. Also gilt 4).
    Gelte 4). Dann ist \(f\) ein Monomorphismus. Zu \(z=y\) gibt es \(g\colon y\to x\) mit \(f_*(g)=\id_y\) also \(f\circ g=\id_y\). Dann ist \(g\) ein Schnitt und es gilt \(2).\)
    Das zeigt \(1)\iff 2)\iff 4)\). Analog zeigt man \(1)\iff 3)\iff 5).\)
\end{proof}
\section{Funktoren}
\begin{Def}[Funktor]
    Seien \(\calC\) und \(\calD\) Kategorien. Ein Funktor \(F\colon\calC\to\calD\) besteht aus folgenden Daten:
    \begin{enumerate}
        \item Eine Abbildung \[\ob(\calC)\to\ob(\calC),\; x\mapsto F(x).\]
        \item Für alle Objekte \(x,y\in\calC\) eine Abbildung 
        \[\Mor_\calC(x,y)\to\Mor_\calD(F(x),F(y)),\; f\mapsto F(f).\]
    \end{enumerate} Dabei fordern wir, dass die folgenden Bedingungen erfüllt sind:
    \begin{enumerate}
        \item Für alle \(x\in\calC\) ist \(F(\id_x)=\id_{F(x)}.\)
        \item Für alle \(f\colon x\to y\) und \(g\colon y\to z\) in \(\calC\) ist \(F(g\circ f)=F(g)\circ F(f).\)
    \end{enumerate}
    Ein Funktor \(F\colon\calC\to\calD\) ist treu (bzw. volltreu), falls für alle \(x,y\in \calC\) die Abbildung \[F\colon\Mor_\calC(x,y)\to\Mor\calD(F(x),F(y))\]  injektiv (bzw. bijektiv ist).
\end{Def}
\begin{Bem}
    Sei \(F\colon\calC\to \calD\) ein treuer Funktor. Dann reflektiert \(F\) Mono- und Epimorphismen. Wenn \(F\) volltreu ist, dann reflektiert \(F\) auch spaltende Mono- und Epimorphismen.
\end{Bem}
\begin{Def}
    Sei \(\Cat\) die Kategorie deren Objekte kleine Kategorien \(\calC\) sind mit Morphismen sind die Funktoren.
\end{Def}
\begin{Def}
    Sei \(I\) eine partiell geordnete Menge (zB. \([n]=Set{0,1,\dots n}\)). Dann definiere eine Kategorie \(N P\) durch $$\ob(N P)=P\text{ und } \Mor_{N P}(p,q)=\begin{cases}
        \Set{*} & \text{ falls } p\leq q\\
        \emptyset & \text{ sonst.}
    \end{cases}$$
    Das gibt einen volltreuen Funktor \(N\colon \CatOrd\to\Cat\) 
\end{Def}
\section{Natürliche Transformationen}
\begin{Def}
    Seien \(F,G\colon\calC\to\calD\) Funktoren zwischen Kategorien. Eine natürliche Transformation \(\tau\colon F\to G\) besteht aus den folgenden Daten:
    \begin{enumerate}
        \item Für alle \(x\in\calC\) ein Morphismus \(\tau_x\colon F(x)\to G(x)\) in \(\calD\).
    \end{enumerate}
\end{Def}
Dabei fordern wir, dass folgende Bedingung erfüllt ist:
\begin{enumerate}
    \item Für alle Morphismen \(f\colon x\to y\) in \(\calC\) kommutiert das Diagramm 
    % https://tikzcd.yichuanshen.de/#N4Igdg9gJgpgziAXAbVABwnAlgFyxMJZABgBpiBdUkANwEMAbAVxiRADEAKADwEoQAvqXSZc+QigCM5KrUYs2AcR78hI7HgJEyk2fWatEHTgE9VwkBg3ii03dX0Kjys4NkwoAc3hFQAMwAnCABbJDIQHAgkaTkDNgAdeJw6JgB9bkELQJCw6kikACYHeUNjP3N-INDEIoioxABmYrijROS0jLUQbOqY-MbmpxBlcrcBIA
\begin{tikzcd}
F(x) \arrow[r, "\tau_x"] \arrow[d, "F(f)"] & G(x) \arrow[d, "G(f)"] \\
F(y) \arrow[r, "\tau_y"]                   & G(y)                  
\end{tikzcd} Wir schreiben \(\Nat(F,G)\) für die Klasse der natürlichen Transformationen \(\tau\colon F\to G$.
\end{enumerate}
\begin{Def}[Funktorkategorie]
    Seien \(\mathcal{C}\) und \(\mathcal{D}\) Kategorien. Dann definieren wir die Funktorkategorie \(\Fun(\calC,\calD)\) wie folgt
    \begin{enumerate}
        \item Die Objekte sind Funktoren \(F\colon \calC\to\calD.\)
        \item Morphismen \(\tau\colon F\to G\) sind natürliche Transformationen  mit Verkettung \((\nu\circ \tau)_x=v_x\circ\tau_x\) für \(x\in\calC\).
    \end{enumerate}
\end{Def}
\begin{Lemma}
    Sei \(\calC\) eine kleine und \(\calD\) eine lokal kleine Kategorie. Dann ist \(\Fun(\calC,\calD)\) wieder lokal klein.
\end{Lemma}
\begin{proof}
    Seien \(F,G\colon\calC\to\calD\) Funktoren. Dann ist 
    \[\Nat(F,G)\subseteq \prod_{x\in\calC}\Mor_\calD(F(x),G(x))\] und das letze ist eine Menge.
\end{proof}
\begin{Lemma}
    Eine natürliche Transformation \(\tau\colon F\to G\) ist natürlicher Isomorphismus, wenn \(\tau_x\) ein Isomorphismus für alle \(x\in \calC\) ist.
\end{Lemma}
\begin{proof}
    Behauptung: \((\tau^{-1}_x)_x\) ist eine natürliche Transformation. Sei \(f\colon x\to y\) ein Morphismus. Dann ist 
    \begin{align*}
        \tau_y^{-1}\circ G(f)&=\tau_y^{-1}\circ (G(f)\circ \tau_x)\circ \tau_x^{-1}\\
        &= \tau_y^{-1}\circ( \tau_y \circ F(f))\circ\tau_x^{-1}\\
        &= F(f)\circ \tau_x^{-1}
    \end{align*}
\end{proof}
\begin{Satz}\label{Satz:BildFunktor}
    Sei \(i\colon\calD'\to\calD\) ein volltreuer Funktor und \(F\colon\calC\to\calD\) ein Funktor mit \(\Image(F)\subseteq \Image(i)\). Dann existiert ein Paar \((F',\kappa)\) bestehend aus einem Funktor \(F'\colon\calC\to \calD'\) und einen natürlichen Isomorphismus \(\kappa\colon i\circ F'\cong F\). Weiter gibt es für zwei solcher Paare \((F_1',\kappa_1)\) und \((F_2',\kappa_2)\) einen eindeutigen natürlichen Isomorphismus \(\nu\colon F'_1\cong F_2'\) sodass \(\kappa_2\circ i(\nu)=\kappa_1\), das heißt das Paar \((F',\kappa)\) ist eindeutig bis auf eindeutigen Isomorphismus.
\end{Satz}
\begin{Lemma}
    Seien \(\calC,\calD\) und \(\calE\) Kategorien. Dann gibt es einen natürlichen Isomorphismus von Kategorien 
    \[\Fun(\calC\times\calD,\calE)\cong \Fun(\calC,\Fun(\calD,\calE)).\]
\end{Lemma}
\begin{proof}
    Klar.
\end{proof}
\begin{Def}[Kategorienäquivalenz]
Sei \(F\colon\calC\to\calD\) ein Funktor. Ein Quasiinverses zu \(F\) ist ein Funktor \(G\colon\calD\to\calC\) zusammen mit natürlichen Isomorphismen \(\alpha\colon G\circ F\cong \id_\calC\) und \(\beta\colon F\circ G\cong \id_\calD\) sodass 
\[F(\alpha)=\beta_F\colon FGF\cong F \text{ und } G(\beta)=\alpha_G\colon GFG\cong G.\]
Falls ein Quasiinverses existiert dann nennen wir \(F\) eine Kategorienäquivalenz.
    
\end{Def}
\begin{Satz}
    Für einen Funktor \(F\colon\calC\to\calD\) sind die folgenden Bedingunen äquivalent:
    \begin{enumerate}
        \item \(F\) ist eine Kategorienäquivalenz.
        \item Es existiert ein Funktor \(G\colon\calD\to\calC\) mit natürlichen Isomorphismen \(\alpha\colon G\circ F\cong \id_\calC\) und \(\beta\colon F\circ G\cong \id_\calD.\)
        \item \(F\) ist volltreu und essentiell surjektiv.
    \end{enumerate}
    Außerdem ist ein Quasiinverses zu \(F\) eindeutig bis auf eindeutigen Isomorphismus.
\end{Satz}
\begin{proof}
    1) nach 2) ist klar. Gelte 2). Für \(y\in\calD\) ist \(\beta_y^{-1}\colon y\cong F(G(y))\) also ist \(F\) essentiell surjektiv. Außerdem sind für \(x,y\in\calC\) die Diagramme 
    $$% https://tikzcd.yichuanshen.de/#N4Igdg9gJgpgziAXAbVABwnAlgFyxMJZABgBpiBdUkANwEMAbAVxiRAB12BZCAJwH1OAY0YBhABQAPUgE8AlCAC+pdJlz5CKAEzkqtRizaceA4YwAi4gGJS5pG-IXLV2PASIAWXdXrNWiDm4+QXYRBgkAcRtJOyjxRyUVEAxXDU9SAEY9X0MA42CzcKlZJz0YKABzeCJQADNeCABbJAzqHAgkHX0-NgjEuobmxC72pABmagY6ACMYBgAFNTdNEAYYWpwQHwN-QMY0AAs6fhlhLF4hAAJxAFo5M4vOfaP+SQA9YBuMxX6QeqakGQQKNEBNVlgwLsoBAmNM1lsQAcYHQoGxIJCETg6FgGGiCKxnH9BoC2h1EK1urkQFYlBRFEA
\begin{tikzcd}
{\Mor_\calC(x,y)} \arrow[rrrrd, no head, Rightarrow] \arrow[rr, "F"] &  & {\Mor_\calD(F(x),F(y))} \arrow[rr, "G"] &  & {\Mor_\calC(GF(x),GF(y))} \arrow[d, "\alpha_y\circ (-)\circ\alpha_x^{-1}"] \\
                                                                     &  &                                         &  & {\Mor_\calC(x,y)}                                                         
\end{tikzcd}$$
\end{proof}
\begin{Bem}
    Sei \(F\colon \calC\to\calD\) ein Funktor. Dann ist äquivalent 
    \begin{enumerate}
        \item \(F\) ist Kategorienäquivalenz
        \item Für alle Kategroien \(\calE\) ist \(F_*\colon \Fun(\calE,\calC)\to \Fun(\calE,\calD)\) eine Kategorienäquivalenz
        \item Für alle Kategroien \(\calE\) ist \(F^*\colon \Fun(\calD,\calE)\to \Fun(\calC,\calE)\) eine Kategorienäquivalenz    \end{enumerate}
\end{Bem}
\section{Yoneda Lemma}
\begin{Satz}[Yoneda Lemma]\label{Lem:Yoneda}
    Sei \(\calC\) eine lokal kleine Kategorie. Sei \(F\colon \calC^\op\to\Sets\) ein Funktor und \(x\in\calC\). Dann ist die Abbildung 
    \[\Nat(\Mor_\calC(-,x),F)\to F(x),\; \tau\mapsto \tau_x(\id_x)\] bijektiv.
\end{Satz}
\begin{proof}
    Für \(s\in F(x)\) definiere \(\tau^{(s)}\colon\Mor_\calC(-,x)\to F\) durch \[\tau_y^{(s)}\colon \Mor_\calC(y,x)\to F(y),\; f\mapsto F(f)(s)\] für \(y\in \calC\).\\
    Es gilt \(\tau_x^{(s)}(\id_x)=s\) und \(\tau_y^{(\tau_x(\id_x))}(f)=\tau_y(f)\).
    Also gilt die Aussage.
\end{proof}
\begin{Kor}[Yoneda-Einbettung]
    Sei \(\calC\) eine lokal kleine Kategorie. Dann ist der Funktor \[Y_\calC\colon\calC\to\Fun(\calC^\op,\Sets),\; x\mapsto \Mor_\calC(-,x)\] volltreu.
\end{Kor}
\begin{proof}
    Seien \(x,y\in\calC\). Zeige 
    \[\Mor_\calC(x,y)\to\Nat(\Mor_\calC(-,x),\Mor_\calC(-,y)),\; f\mapsto f_*\] ist eine Bijektion. Das ist aber genau die Inverse Abbildung aus dem \nameref{Lem:Yoneda} für den Funktor \(F=\Mor_\calC(-,y).\)
\end{proof}
\begin{Bem}
    Es gibt eine duale Version vom Yoneda Lemma. Sei \(\calC\) eine lokal kleine Kategorie und \(F\colon\calC\to\Sets\) ein Funktor und \(x\in\calC\). Dann ist die Abbildung 
    \[Nat(\Mor_\calC(x,-),F))\to F(x),\; \tau\mapsto \tau_x(\id_x))\] bijektiv. Dann ist die duale Yoneda Einbettung \(Y^\calC\colon\calC^\op\to\Fun(\calC,\Sets),\; x\mapsto \Mor_\calC(x,-)\) volltreu.
\end{Bem}
\begin{Def}
    Sei \(\calC\) eine lokal kleine Kategorie. Ein Funktor \(F\colon\calC^\op\to \Sets\) heißt darstellbar falls er im essentiellen Bild der Yoneda Einbettung liegt.  Analog heißt ein FUnktor \(F\colon\calC\to\Sets\)  kodarstellbar, wenn er im essentiellen Bild von \(Y^\calC\) liegt.
\end{Def}
\section{Adjunktionen}
\begin{Def}[Adjungiertes Objekt]
    Sei \(G\colon\calD\to\calC\) ein Funktor und \(x\in\calC\). Ein unter \(G\) zu \(x\) linksadjungiertes Objekt ist ein Objekt \(y\in\calD\) zusammen mit einem Morphismus \(\eta\colon x\to G(y)\) sodass für alle \(y'\in\calD\) die Abbildung 
    \[\Mor_\calD(y,y')\to\Mor_\calC(x,G(y')), \; f\mapsto G(f)\circ\eta\] bijektiv ist.
\end{Def}
\begin{Bem}
    sei \(G\calD\to\calC\) ein Funktor sodass \(\calC,\calD\) lokal kleine Kategorien sind. Dann ist ein zu \(x\) linksadjungiertes Objekt das gleiche wie ein kodarstellendes Objekt für den Funktor \[\Mor_\calC(x,G(-))\colon \calD\to\Sets\]
\end{Bem}
\begin{Satz}[Adjungierter Fuktor]\label{Satz:AdjFun}
    Sei \(G\colon\calD\to\calC\) ein Funktor und sei \(i\colon \calC'\subseteq \calC\) eine volle Unterkategorie, sodass alle Objekte in \(\calC'\) unter \(G\) ein linksadjungiertes Objekt besitzen. Dann existiert ein Funktor \(F\colon \calC'\to\calD\) und eine natürliche Transformation \(\eta\colon i\to G\circ F\) sodass \((F(x) \eta_x)\) für jedes \(x\in\calC\) ein unter \(G\) zu \(x\) linksadjungiertes Objekt ist. Das Paar
 \((F,\eta )\) ist eindeutig bis auf eindeutigen Isomorphismus.
 Wir nennen \(F\), zusammen mit der natürlichen Transformation , einen partiellen links
adjungierten Funktor zu \(G\). Wenn \(\calC= \calC'\) dann nennen wir \(F\colon \calC\to \calD\) auch einfach einen
 linksadjungierten Funktor zu \(G\).
\end{Satz}
\begin{proof}
    Wähle ein linksadjungiertes Objekt \((F(x), \eta_x)\) für jedes \(x\in\calC.\) Für einen Morphis
mus \(f\colon x\to y\) in \(\calC\) definiere \(F(f)\colon F(x) F(y)\) als das eindeutige Urbild von \(\eta_y\circ f\) unter
 der bijektiven Abbildung
 \[\Mor_\calD (F(x),F(y))\to \Mor_\calC (x,G(F(y)),\; g\mapsto G(g)\circ \eta_x;\]
 wir haben also \(G(F(f))\circ\eta_x = \eta_y\circ f.\)
 Wir behaupten nun noch, dass diese Daten einen Funktor $F\colon \calC
 \to \calD$ definieren; danach
 ist es dann sofort klar, dass die \(\eta_x\) eine natürliche Transformation \(\eta\colon i\to G\circ F\) definieren.
 Für ein Objekt \(x\in\calC'\) ist
\[ G(\id_{F(x)})\circ\eta_x = \eta_x= \eta_x \circ\id_x\]
 und somit \(F(\id_x) = \id_{F(x)}\). Weiter ist für Morphismen \(f\colon x\to y\) und \(g\colon  y\to  z\)
\begin{align*}
    G(F(g)\circ F(f))\circ \eta_x&=G(F(g))\circ G(F(f))\circ\eta_x\\
 &=G(F(g))\circ\eta_y\circ f\\
 &= \eta_z\circ g\circ f
\end{align*} 
 und somit \(F(g\circ f) = F(g)\circ F(f).\)
 Die Eindeutigkeit von \((F,\eta)\) bis auf eindeutigen Isomorphismus folgt aus der Eindeutigkeit
 von linksadjungierten Objekten.
\end{proof}
\begin{Bem}
     Sei \(G\colon \calD\to \calC\) ein Funktor und sei \(\calC'\subseteq \calC\) die volle Unterkategorie der
 Objekte die unter \(G\) ein linksadjungiertes Objekt besitzen. Wir nehmen außerdem an, dass \(\calC\)
 und \(\calD\) lokal klein sind.
 Dann ist ein partieller Linksadjungierter \(F\colon\calC'\to \calD\) von \(G\) das Gleiche wie eine Faktorisierung des Funktors
 \[\calC'\to\Fun(\calD,\Sets)^\op,\; x\mapsto\Mor_\calC(x,G(-))\]
 durch die (volltreue) duale Yoneda-Einbettung \[Y^{\calD,\op}\colon \calD\to\Fun(\calD,\Sets)^\op.\] In dieser Situation
 ist also \Cref{Satz:AdjFun} eine Folgerung aus \Cref{Satz:BildFunktor}.
\end{Bem}
\begin{Def}\label{Def:ArrowCat}
    Sei \(\calC\) eine Kategorie. Wir definieren die Pfeilkategorie von \(\calC\) als \[\Arr(\calC)=\Fun(N[1],\calC).\]
\end{Def}
\begin{Lemma}\label{Lem:CodomAdj}
    Sei \(\codom\colon \Arr(\calC)\to \calC\) der Funktor, der \(\phi\colon A\to B\) auf \(B\) schickt und ein kommutatives Diagramm auf den rechten Morphismus.
    Dann ist \(\codom\) links-adjungiert zum Funktor \(R\colon \calC\to \Arr(\calC), C\mapsto (\id\colon C\to C)\)
\end{Lemma}
\begin{proof}
    Klar
\end{proof}
\section{Moduln}
\begin{Def}
    Sei \(\calC\) eine Kategorie mit Pullbacks und sei \(A\in\calC\). Ein Modul ist ein abelsches Gruppenobjekt in der Slice Kategorie \(\calC_{/A}\).
\end{Def}