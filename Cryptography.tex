\chapter{Cryptography}
Book recommendations: 
Buchmann Einführung in die Kryptografie\\
Bernstein/Buchmann/Dahmen Post Quantum Cryptography\\
Hoffstein/Pipher/Silverman An introduction to mathematical Cryptography\\
Werner Elliptische Kurven in der Kryptografie\\
Criteria for passing: 50 percent of the points, 75 percent useful answers to questions, present twice per semester\\
\section{Basics and historical examples}
\subsection{Ceasar-Chiffre}
It was used during (100-44 bc)
\begin{Bsp}
\textbf{WKDRO SCD CMRYOX}
Procedure: We choose a number of \(i \in\{1,\dots,26\}\) and move every letter of the plaintext by the fixed number \(i\).
For example \(i=5\) and the plaintext \textbf{TODAY}. This becomes \textbf{YTIFD}.
The encyphered plaintext is called cipher text.
The set of Keys is in this example \(\ZZ/26\ZZ\).
\end{Bsp}
\begin{Def}[Cryptosystem]
A Cryptosystem consists of 
\begin{enumerate}
\item a set \(\mathcal{P}\) of plaintext
\item a set \(\mathcal{C}\) of cipher text
\item a set \(\mathcal{K}\) of Keys
\item an encyphering function \(e\colon \mathcal{P}\times\mathcal{K}\to\mathcal{C}\)
\item a decyphering function \(d\colon \mathcal{C\times K\to P}\)
\end{enumerate}
Such that for each \(k\in K\) there is a \(h\in K\) such that
\[d(e(m,k),h)=m\] for all \(m\in \mathcal{P}\).
Write for \(e(m,k)\) also \(e_k(m)\) and for \(d(c,k)=\) also \(d_k(m)\).
\end{Def}
\subsection{Kerkhoff’s principle(Jean Guillaume Hubert Victoir François Alexandre August Kerkhoff von Nieuwenhof 1883)}
\textbf{The security of a crypto system should not be based on the knowledge of the used system. The security should only be based on the secrecy of the Key}
\section{Alphabet and Words}
\begin{Def} 
An Alphabet is a non-empty finite set. The elements of the Alphabet are called Letters.
\end{Def}
\begin{Bsp}
Typical alphabets:
\begin{enumerate}
\item \(\mathcal A=\{a,\dots,z\}\)
\item \(\mathcal A=\{0,1\}\)
\item \(\mathcal A=\text{ASCII-Code}\}\) (American standard Code for information interchange)
\item \(\mathcal A=\ZZ/n\ZZ\)
\end{enumerate}
\end{Bsp}
\begin{Def}
Let \(calA\) be an alphabet. A finite sequence of elements from \(\calA\) is called a Word over \(\calA\). The length of a Word \(\omega\) over \(\calA\) is the length of the corresponding sequence.
\(\calA^n\) is the set of Words of length \(n\) and \(A^\infty=\bigcup_{n\in \NN} \calA^n\) the set of all Words.
\end{Def}
\begin{Bem}
\begin{enumerate}
\item[]
\item The Word of length 0 is \(\{\}\) the empty Word.
\item Let \(w_1=\sigma_1\dots\sigma_r\) and \(w_2=\tilde \sigma_1\dots\tilde\sigma_s\) be Words over \(\calA\). Then
\[w_1\cdot w_2:= \sigma_1\dots\sigma_r\tilde \sigma_1\dots\tilde\sigma_s\] is again a Word.
This makes \((\calA^\infty,\cdot)\) a Monoid (Muliplication is associative with a neutral element \(\{\}\).
\end{enumerate}
\end{Bem}
\begin{Bsp} In Caesar chiffre we have \(\calP=\calC=\calA^\infty\) with \(\calA=\ZZ/26\ZZ\).
\(\calK=\calA\)  and
\(e_k(w)=e_k(\sigma_1\dots\sigma_r)= \sigma_1+k\dots\sigma_r+k\)
And \(d_k(w)=d_k(\sigma_1\dots\sigma_r)= \sigma_1-k\dots\sigma_r-k\)
\end{Bsp}
Generalization
\begin{Bsp}[Permutation Chiffre]
\(\calA=\ZZ/n\ZZ\, \calP=\calC=\calA^\infty, \calK=S_n\)
For \(\pi\in S_n\) we set
\[e_\pi(w)=\pi(\sigma_1)\dots\pi(\sigma_r)\in\calC\]

\[d_\pi(w)=\pi^{-1}(\sigma_1)\dots\pi^{-1}(\sigma_r)\in\calC\]
\end{Bsp}
\subsection{Statistical analysis}
In German and in English the five most often used letters are 
\textbf{E,N,I,R,S} with percentages 18, 11 , 8, 7, 7 in German and \textbf{E,T,A,I,N} with percentages
12,9,8,7,7 in English respectively.
In our Crypto Systems we have to take care that the frequency of a letter has no importance.
\begin{Def} 
A block Cipher is a crypto system in which all the plaintexts have the same length, as well as every cipher text. I.E. \(\calP=\calA^n=\calC\) for some \(m\in \NN\).
Then for each \(k\in K\) we have \(e_k\colon \calA^n\to \calA^n\)

\end{Def}
\begin{Bem}
We can split any text in blocks of equal size.
\end{Bem}
\begin{Bsp} The Vignère cipher is based on such a splitting.
\end{Bsp}
\subsection{Types of Attacks}
\begin{enumerate}
\item[]
\item \textbf{Attack with the ciphertext(ciphertext only attack)}
The attacker Eve has only the cipher text.
\item \textbf{Knowing plaintext attack}
Eve has plain and Chiffre text for certain parts of the message.
\item \textbf{Chosen plaintext attack}
Eve can produce to a given plaintext the cipher text
\end{enumerate}
\begin{Bsp}[Ceasar Cipher]
\begin{enumerate}
\item[]
\item Eve can try all the 26 Keys to decipher
\item Eve only needs to know the cipher text of one letter
\item Eve only needs to ask for the cipher text of one letter
\end{enumerate}
\end{Bsp}
\begin{Bsp}[Permutation Cipher]
\begin{enumerate}
\item[]
\item Eve can use the statistical analysis 
\item Eve knows the encyphering for some letters and can use first point for the remaining
\item She can ask to Encipher the Alphabet 
\end{enumerate}
\end{Bsp}
\chapter{2}
\section{The \(O\)-Notation}
Running time of Algorithms is taken in \(O\) notation(Landau 1877-1938).
\begin{Def} Let \(f,g\colon \NN^k\to\RR\). We write 
\(f=O(g)\) if there is a constant \(0<c\in\RR\) and some \(n_0\) such that \(f(n_1,\dots,n_k)<=cg(n_1,\dots,n_k)\) for all \(n_1,\dots,n_k\geq n_0\).
\end{Def}
\begin{Bsp} \[2n^2+17n+10^{11}=O(n^2)\] quadratic 
\end{Bsp}
\begin{Bsp}
let \(f(n)\) the length of the \(a\)-adic expansion of \(n\) where \(a\geq 2\).
Then \(f(n)=O(\log(n))\).
\(f(n)=\floor{\log_a(n)}+1\leq \log_an+1=\frac{1}{\log(a)}\log(a)+1\) and here \(n_0=3\) and \(c=\frac{1}{\log }+1\)
\end{Bsp}

\section{Running times}
Assume that 
\begin{enumerate}
\item The addition of 2 bits needs constant time \(O(1)\)
\item The subtraction of 2 bits needs constant time \(O(1)\)
\item Writing costs nothing
\end{enumerate}
\begin{Lemma}
Let \(a,b\in\Nn\) in their binary expansion. 
\begin{enumerate}
\item The addition of \(a\) and \(b\) costs \(O(\max(\log(a),\log(b))\) time
\item The Multiplication of \(a\) and \(b\) takes \(O(\log(a)\log(b))\) time. Clear because you always add 2 numbers of length \(a\) exactly the amount of times of the length of the other number. So it follows from the first part.
\item Division with remainder of \(a\) and \(b\) costs \(O(\log(b)\log(\frac{a}{\floor{b}}))\).
\end{Lemma}
\begin{Def} An algorithm is polynomial (has polynomial running time) if there are exponents \(e_1,\dots,e_n\) such that the running time is \[O(\log(|z_1)^{e_1}\dots\log(|z_k|)^{e_k})\] by input or \(z_1,\dots,z_k\in \ZZ\) They are also called efficient.
\end{Def}
\section{Number Theory}
We can calculate the greatest common divisor using Euclidean algorithm.
\subsection{Euclidean Algorithm}
The idea is that \(\ggT(a,b)=\ggT(a,a\mod b)\).
\begin{Satz} Let \(a,b\in\NN_{>0}\) and \(a>b\). The number of iterations of the Euclidean algorithm is at most \(\log_2(a)/\log_2(\theta)+1=O(\log(a))\) where \(\theta=\frac{sqrt{5}+1}{2}\)
\end{Satz}
\begin{Satz} Let \(a,b\in\NN\). Then the running time of the Euclidean algorithm is \(O(\log(a)\log(b))\).
\end{Satz}
\begin{proof} Wlog \(a>b\). In each iteration of the Euclidean algorithm we do a division by remainder. It takes \(O(\log\floor(r_k/r_{k-1})\log(r_k))=O(\log(q_{k+1})\log(r_k))\) to determine \(q_{k+1}\) and \(r_{k+1}\).
It is \(r_k\leq b\) so \(\log(r_k)\leq\log(b)\) and \(\log(q_k)=\log_2(q_k)+1\). Thus the Euclidean algorithm needs \(O(\log(b)(\sum_{k=0}^n\log(q_k)))=O(\log(b)(n+\sum_{k=0}^n\log(q_k)))\)
\(n=O(\log(a))\).
We can also calculate backwards that \(a\geq q_1\cdots q_n\). 
So \(\sum_{k=1}^n \log_2(q_k)\leq \log_2(a)\).
The statement follows.
\end{proof}
