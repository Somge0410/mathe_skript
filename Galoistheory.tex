\chapter{Hauptsatz der Galoistheorie}
\section{Galois Erweiterungen}
\begin{Def}
    Eine Körpererweiterung \(L/K\) ist eine Galoiserweiterung, wenn \(L/K\) normal und separable ist. In dem Fall ist \(\Gal(L/K)=\Aut_K(L)\)
\end{Def}
\begin{Satz}
    Sei \(L/K\) endlich. Dann ist \(|\Aut_K(L)|\leq [L:K]\) und Gleichheit gilt genau dann, wenn \(L/K\) Galoisch.
\end{Satz}
\begin{proof}
    Wähle algebraischen Abschluss \(\bar L/L\). Es gilt 
    \[\Aut_K(L)=\Hom_K(L,L)\subseteq \Hom_K(L,\bar L)\]
    Somit \(|\Aut_K(L)|\leq [L:K]_s\leq [L:K]\).
    Gleichheit in 1 ist genau dann, wenn \(L/K\) normal und Gleichheit in 2 genau dann, wenn \(L/K\) separable.
\end{proof}
\begin{Bsp}
    Sei \(L=\QQ(\sqrt[3]{2},\zeta)/\QQ\) wobei \(\zeta=e^{2\pi i/3}\). Es ist \([L:K]=6\) und \(L/K\) Galoisch. Sei \(N=\{a,\zeta a,\zeta^2 a\}\) und \(\sigma\in\Gal(L/K)\) \(\sigma(N)\subseteq N\). Haben also 
    \[\psi\colon \Gal(L/K)\to S(N)\cong S_3,\ \sigma\mapsto \sigma|_N\]
    \(\psi\) ist injektiv denn \(\sigma\) ist Identität auf \(\QQ\) also muss es sich auf Erzeugern unterscheiden. Da \(|\Gal(L/K)|=[L:K]=6=|S_3|\) ist \(\psi\) bijektiv.
\end{Bsp}
\section{Galois Korrespondenz}
\begin{Def}
    Sei \(L\) ein Körper und \(G\subseteq \Aut(L)\) Untergruppe.
    Der Körper der \(G\)-Invarianten ist \(L^G=\set{x\in L}{\sigma(x)=x\ \forall \sigma\in G}\)
\end{Def}
\begin{Lemma}\label{Lem:Galois1}
    Wenn \(L/K\) endlich Galoisch ist mit \(G=\Gal(L/K)\), dann ist \(K=L^G\).
\end{Lemma}
\begin{proof}
    Es gilt \(K\subseteq L^G\subseteq L\) und \(G\subseteq\Aut_{L^G}(L)\). Es ist \(L/L^G\) Galoisch und deshalb
    \[[L:K]\geq [L:L^G]=|\Aut_{L^G}(L)|\geq |G|=[L:K]\]
    Somit \([L^G:K]=1\) und \(L^G=K\).
\end{proof}
\begin{Satz}\label{Satz:Galois2}
    Sei \(G\subseteq \Aut(L)\) eine endliche Untergruppe und \(K=L^G\). Dann ist \(L/K\) Galoisch und \([L:K]=|G|\) und \(\Gal(L/K)=G\).
\end{Satz}
\begin{proof}
    Sei \(b\in L\) und \(N\subset L\) die \(G-\)Bahn von \(b\), das heißt \(N=\set{\sigma(b)}{ \sigma\in G}=\{b_1,\dots,b_r\}\) wobei \(r=|N|\). Sei \(f=\prod_{i=1}^r(X-b_i)\in L[X]\).
    Es ist \[f^\sigma=\prod_i(X-b_i)^\sigma=\prod_i(X-\sigma(b_i))=\prod_i(X-b_i)=f\] für alle \(\sigma\in G.\) Also ist \(f\in K[X]\) und \(b\) algebraisch über \(K\) und separable da \(f\) in verschiedene Nullstellen zerfällt.
    Also ist \(L/K\) Galoisch.
    Angenommen \([L:K]>|G|\). Sei \(G=\Set{\sigma_1,\dots,\sigma_n}\). Wähle \(y_1,\dots,y_m\in L\) \(K\)-linear unabhängig und sei \(A\) die Matrix \(A=(\sigma_i(y_j))_{i,j}\in M_{n\times m}(L)\).
    Aus \(m>n\) folgt es gibt ein \(b\in L^m\setminus\Set{0}\) im Kern von \(A\).
    Sei \(b=(b_1,\dots b_m)\) und \(\sigma(b)=(\sigma(b_1),\dots,\sigma(b_m))\).
    Dann ist \(A\cdot \sigma(b)=0\).
    Sei \(\ell(b)\) die Anzahl der \(j\) mit \(b_j\neq 0\) Wähle also \(b\neq 0\) mit \(Ab=0\) und \(\ell(b)\) minimal. Ohne Einschränkung \(b_j=1\) für ein \(j\).
    Für \(\sigma\in G\) ist \(\ell(b-\sigma(b))<\ell(b)\) da \(\sigma(1)=1\) und \(A(b-\sigma(b))=0\). Da \(\ell(b)\) minimal ist \(b=\sigma(b)\) und somit \(b\in K^m\).
    Da \(Ab=0\) ist \(\sum_{j=1}^my_jb_j=0\) und somit sind die \(y_i\) linear abhängig Was ein Widerspruch darstellt.
    Also ist \([L:K]=|G|\) und damit 
    \[|G|\leq |\Gal(L/K)|=[L:K]=|G|\] also \(G=\Gal(L/K)\)
\end{proof}
\begin{Bsp}
    \(\FF_q\) ist Galoisch über \(\FF_p\). Sei \(\phi\colon \FF_q\to \FF_q, \phi(x)=x^p\) der Frobenius. Es ist \(G=\Gal(\FF_q/\FF_p)=\anglebr{\phi}\)
, denn \(\phi(x)=x\iff x^p-x=0\) hat nur \(p\) verschiedene Lösungen. Also ist \(\FF_p\subseteq \FF_q^{\anglebr{\phi}}\) eine Gleichheit. Der Satz zeigt, dass \(G\) von \(\phi\) erzeugt ist.
\end{Bsp}
\begin{Satz}[Galoiskorrespondenz]
Sei \(L/K\) eine endliche Galoiserweiterung und \(G=\Gal(L/K)\).
\begin{enumerate}
    \item Folgende Abbildungen sind zueinander inverse Bijektionen:
% https://tikzcd.yichuanshen.de/#N4Igdg9gJgpgziAXAbVABwnAlgFyxMJZABgBpiBdUkANwEMAbAVxiRAB12GYAzHT4JxwwAHjmABVMMIBOAcxlM0aGGAAEAEgASnOEwBGcGMICOagOIaAvpxlY5AC37srIK6XSZc+QigBM5FS0jCxsnNx8AkKi4gBaAO5YcADGDqoA1gBvaDIwMpoA0roGRqZqALLFhsYwZgAy1rb2TpyuVkEwUHLwRKA8MhAAtkhkIDgQSACM1HAOWHxIALQALNT0zKyIIDrsg3RocONqdQB6Wm4eIP1DU9TjIzNzC4grayGbHOzCYsDmjFYACjqAHpygBKTh7A7XQYVNwUKxAA
\begin{tikzfigure}
\Set{\text{Untergruppen \(H\subseteq G\)}} \arrow[rr, "H\mapsto L^H", shift left=4] &  & \Set{\text{Zwischenkörper \(K\subseteq M\subseteq L\)}} \arrow[ll, "\Gal(L/M)\mapsfrom M", shift left=4]
\end{tikzfigure}
\item Wenn \(H\) mit \(M\) korrespondiert, dann ist \([M:K]=[G:H]\) oder äquivalent \([L:M]=|H|\)
\item Wenn \(\sigma\in G\) ist und \(M\) zu \(H\) korrespondiert, dann korrespondiert \(\sigma(M)\) zu \(\sigma H\sigma^{-1}\)
\item wenn \(H_1\) zu \(M_1\) korrespondiert und \(H_2\) zu \(M_2\) korrespondiert, dann ist \[H_1\subseteq H_2\iff M_1\supseteq M_2\]
\item \(M=K\) korrespondiert zu \(H=G\) und \(M=L\) korrespondiert zu \(H=\Set{e}\).
\end{enumerate}
\begin{proof}
    \begin{enumerate}
        \item[] 
        \item Man prüft leicht, dass beide Abbildungen wohldefiniert sind.
        Wenn \(K\subseteq M\subseteq L\) Zwischenkörper ist, dann ist \(M=L^{\Gal(L/M)}\) wegen \Cref{Lem:Galois1}.
        Wenn \(H\subseteq G\) eine Untergruppe ist, dann ist \(H=\Gal(L/L^H)\) nach \Cref{Satz:Galois2}.
        \item \[[M:K][L:M]=[L:K]=|G|=|H|[G:H]\] und \([L:M]=|\Gal(L/M)|=H\)
        \item Sei \(\tau\in G\)
        \begin{align*}
            \tau\in \Gal(L/M)&\iff \tau(b)=b,\ \forall b\in M\\
            &\iff \sigma \tau\sigma^{-1}\sigma(b)=\sigma(b),\ \forall b\in M\\
            &\iff \sigma\tau\sigma^{-1}(c)=c,\ \forall c\in \sigma(M)\\
            & \iff \sigma\tau\sigma^{-1}\in\Gal(L/\sigma(M))
        \end{align*}
        \item Klar
        \item klar
    \end{enumerate}
\end{proof}

\end{Satz}
\begin{Satz}
    Sei \(L/K\) endlich Galoisch. Korrespondiere \(M\) zu \(H\) in der Galoiskorrespondenz.
    Es gilt 
    \begin{align*}
        M/K \text{ Galoisch}&\iff M/K \text{ normal}\\
        & \iff H\subseteq G \text{ ist normale Untergruppe.}
    \end{align*} In dem Fall gilt \(\Gal(M/K)\cong G/H\).
\end{Satz}
\begin{proof}
    \(L/K\) separable \(\implies M/K\) separable. Somit gilt die erste Äquivalenz.
    Jeder \(K\)-Homomorphismus \[\sigma\colon M\to \bar L\] hat Fortsetzung zu einem \(K\)-Homomorphismus \[\tilde\sigma\colon L\to\bar L .\]  Da \(L/K\) normal ist, gilt \(\tilde\sigma(L)=L\) das heißt \(\tilde\sigma\in\Aut(L/K)=\Gal(L/K)=G\)
    \begin{align*}
        M/K \text{ normal}&\iff \forall\sigma\in\Hom_K(M,\bar L): \sigma(M)=M\\
        & \iff \forall\tilde\sigma\in G: \tilde\sigma(M)=M\\
        &\iff \text{Für jedes } \tilde \sigma\in G: \tilde\sigma H\tilde\sigma^{-1}=H\\
        &\iff H\subseteq G \text{ normal.}
    \end{align*}
    Sei \(M/K\) normal. Für \(g\in G\) ist \(g(M)=M\). Somit haben wir einen Homomorphismus von Gruppen 
    \(\psi\colon \Gal(L/K)\to\Gal(M/K), \sigma\mapsto \sigma|_M\)
    Jedes \(\tau\in\Gal(M/K)=\bar G\) hat Fortsetzung zu \(\bar\tau\in\Gal(L/K)\). Somit ist \(\psi\) surjektiv. \(\psi\) induziert \(\Gal(L/K)/\ker(\psi)\stackrel{\sim}{\to}\Gal(M/K)\) und 
    \[\Ker(\psi)=\set{g\in G}{ g|_M=\id}=\set{g\in\Aut(L)}{ g|_M=\id}=\Gal(L/M)=H.\]
\end{proof}
\begin{Kor}\label{Kor:SepZwischen}
    Wenn \(L/K\) endlich separable ist, dann hat \(L/K\) nur endlich viele Zwischenkörper.
\end{Kor}
\begin{proof}
    Sei \(\bar L/K\) eine normale Hülle von \(L/K\). \(\bar L/K\) ist Galoisch. Zwischenkörper von \(L/K\) sind Zwischenkörper von \(\bar L/K\). Diese korrespondieren zu Untergruppen von \(G=\Gal(\bar L/K)\). Die letzte Menge ist endlich.
\end{proof}
\begin{Bsp}
    Sei \(L\) Zerfällungskörper von \(X^3-2\) über \(K=\QQ\). Die Abbildung 
    \(\psi\colon G\to S_3\) sei definiert durch Formel \(\sigma(a_i)=a_{\psi(\sigma)(i)}\) für \(a_1=\sqrt[3]{2}\), \(a_2=\zeta a_1\), \(a_3=\zeta^2a_1\).
    Untergruppen von \(G\) sind 
    % https://tikzcd.yichuanshen.de/#N4Igdg9gJgpgziAXAbVABwnAlgFyxMJZARgBoAGAXVJADcBDAGwFcYkQAdD4GLgXxB9S6TLnyEU5UsWp0mrdgB4AFGQBMASgB8g4SAzY8BImRk0GLNohAqyAZm26Rh8UTXTZFhdZXuHOoWcxYxQ7D3N5KxtVUj9HQP1RIwlkdzVPSPYAcUFZGCgAc3giUAAzACcIAFskKRAcCCQyEAALGHoodkgwNgSK6tqaBqR3VvbO627evX6axDrhxDCxjq6CabLKuYXGxAAWGjbVyfWnEFmmod2AVkPxtZ6zi8RRxduVifBTvq2kZbe7scvo8fgN9lckO8jp8prk+EA
\begin{tikzfigure}
                               & \Set{e} \arrow[ld, no head] \arrow[d, no head] \arrow[rd, no head] \arrow[rrd, no head] &                              &                                 \\
{\anglebr{(1,2)}} \arrow[rrd, no head] & {\anglebr{(1,3)}} \arrow[rd, no head]                                                         & {\anglebr{(2,3)}} \arrow[d, no head] & {\anglebr{(1,2,3)}} \arrow[ld, no head] \\
                               &                                                                                       & G                            &                                
\end{tikzfigure}
Das entspricht den Teilkörpern 
% https://tikzcd.yichuanshen.de/#N4Igdg9gJgpgziAXAbVABwnAlgFyxMJZARgBoAGAXVJADcBDAGwFcYkQAZEAX1PU1z5CKAEykR1Ok1bsA0jz4gM2PASLlSxSQxZtEIADoGAisYAURgF4wc9AHoijcAI4AnHMgDMlYCO4BKBX4VISIyLRodGX0jUwsDa1snNw9vXwCgpQFVYWQxCKlddljzZPcvHz9A3mDBNRRPTW1pPUMTUoSbemrJGCgAc3giUAAzVwgAWyQNEBwIJDEQAAsYeih2SDA2SJb2ERAaRnoAIxhGAAVs0P0sMGxYTLHJ6Zo5pEbl1fX9Te3C6JA+0OJzOlxC9RAt3ubBqICeU0QMzeiAALDQVmsNgQ-lFWkCQEdThcrhCoVgHrD4S9ZvNEABWdFfLFbA7-VqeVmE0Ek4SQu7kmGKKn0160sifTE-bGs3F7Tkg4ng3lkilC8YItE0pDijHfcDSnZFfQc4FEsF1ZX81WjdXvUXaxmS-Usw0Ak0EhXmnLsFWCm3PRCLZE6plSl1s9jurmKi0+q0wyjcIA
\begin{tikzfigure}
                                                                & L \arrow[ld, "2" description, no head] \arrow[d, "2" description, no head] \arrow[rd, "2" description, no head] \arrow[rrd, "3" description, no head] &                                                        &                                                 \\
{\QQ(\zeta^2\sqrt[3]{2})} \arrow[rrd, "3" description, no head] & {\QQ(\zeta\sqrt[3]{2})} \arrow[rd, "3" description, no head]                                                                                          & {\QQ(\sqrt[3]{2})} \arrow[d, "3" description, no head] & \QQ(\zeta) \arrow[ld, "2" description, no head] \\
                                                                &                                                                                                                                                       & K                                                      &                                                
\end{tikzfigure}
\end{Bsp}
\section{Satz vom Primitiven Element}
\begin{Satz}[Satz vom primitiven Element]\label{Satz:PrimElt}
Sei \(L/K\) eine endliche separable Erweiterung. Dann gibt es ein \(a\in L\) mit \(L=K(a)\).
    
\end{Satz}
\begin{proof}
    Wenn \(K\) endlich ist, dann ist \(L=\FF_q\) endlich.
    \(L^*=\ZZ/(q-1)\ZZ\) ist zyklisch. Sei \(a\in L^*\) Erzeuger als Gruppe. Dann ist 
    \(L=\Set{0}\cup \set{a^n}{ n\in\NN}\) und somit \(L=K(a)\). Wenn \(K\) unendlich. wähle \(a\in L\) sodass \([K(a):K]\) maximal ist und sei \(b\in L\). Für \(c\in K\) sei \(M_C=K(a+cb)\). Da es nach \Cref{Kor:SepZwischen} nur endlich viele Zwischenkörper gibt, gibt es nur endlich viele Möglichkeiten für \(M_C\). Somit gibt es \(c_1\neq c_2\) mit \(M_{c_1}=M_{c_2}\). Dann ist \(a+c_1b,a+c_2b\in K(a+c_1b)=K(a+c_2b)\)
    Also ist \((c_1-c_2)b\in M_{c_1}\) also ist \(b\in M_{c_1}\) und damit \(a\in M_{c_1}\)
    Somit \(K(a)\subseteq K(a+c_1b)\) und wegen Maximalität von \([K(a):K]\) ist \(K(a)=M_{c_1}\). Somit \(b\in K(a)\) also \(L=K(a).\)
\end{proof}
\section{Kompositum und Translationssatz}
\begin{Def}
    Sei \(L/K\) eine Körpererweiterung und \(M_1,M_2\) zwei Zwischenkörper. Die Komposition \(M_1M_2\) ist der kleinste Teilkörper von \(L\), der \(M_1\) und \(M_2\) enthält.
\end{Def}
\begin{Satz}
    Sei \(L/K\) eine endliche Galoiserweiterung und \(G=\Gal(L/K)\). Wenn \(M_1,M_2\) jeweils zu \(H_1,H_2\) korrespondieren, dann korrespondiert \(M_1M_2\) zu \(H_1\cap H_2\) und \(M_1\cap M_2\) korrepondiert zu \(\langle H_1,H_2\rangle\).
\end{Satz}
\begin{proof}
    Der Kleinste Teilkörper der \(M_1\) und \(M_2\) enthält korrespondiert zu größten Untergruppe von \(G\), die in \(H_1\) und \(H_2\) liegt. Analog folgt die andere Aussage.
\end{proof}
\begin{Satz}[Translationssatz]\label{Satz:Translat}
    Sei \(L/K\) eine endliche Galoiserweiterung und \(K\subseteq M\subseteq L\) sodass \(M/K\) Galoisch. Sei \(K\subseteq K'\subseteq L\) ein Zwischenkörper und \(M'=MK'\) Kompositum. Dann ist \(M'/K'\) Galoisch und \(\Gal(M'/K')\cong \Gal(M/M\cap K')\). Insbesondere ist \([M':K']=[M:K'\cap M]\)
    % https://tikzcd.yichuanshen.de/#N4Igdg9gJgpgziAXAbVABwnAlgFyxMJZARgBoAGAXVJADcBDAGwFcYkQBZEAX1PU1z5CKAEwVqdJq3YcA5Dz4gM2PASJjiEhizaIQAaXm9+KoUTKaa26XsMAdOwGN6aAARdjSgauHIAzOJWUrogADIKJoJqKOSklpI67Po8EjBQAObwRKAAZgBOEAC2SACsNDgQSAEgABYw9FDskGBsnvlFVeWViGK19Y16za2K7cWI1RVIsX0NTQTDuQVjvZOIZDMD4PMgNIz0AEYwjAAK3mZ6jDA5OBEgo1NdSOt1s4PbbUtPj4gALDQvmyGKW4QA
\begin{tikzfigure}
                     & M \arrow[r, no head]                           & MK' \arrow[r, no head] & L \\
K \arrow[r, no head] & K'\cap M \arrow[r, no head] \arrow[u, no head] & K' \arrow[u, no head] &  
\end{tikzfigure}
\end{Satz}
\begin{proof}
    \(M/K\) ist der Zerfällungskörper von separablen Polynomen \(f\in K[X]\). Dann ist \(M'/K'\) der Zerfällungskörper von den selben Polynomen aufgefasst in \(K'[X]\). Somit ist \(M'/K'\) Galoisch.
    Für \(\sigma\in\Gal(M'/K')\) ist \(\sigma|_K=id_K\).
    Da \(M/K\) normal ist, folgt \(\sigma(M)=M\) das heißt \(\sigma|_M\in \Gal(M/K)\). Das gibt Gruppenhomomorphismus
    \[\psi\colon \Gal(M'/K')\to\Gal(M/K),\sigma\mapsto \sigma|_M\].
    Angenommen \(\psi(\sigma)=id\) Dann ist \(\sigma|_{MK'}=id\) also \(\sigma=id\). Also ist \(\psi\) injektiv. Sei \(H\subseteq \Gal(M/K)\) das Bild.
    Dann ist 
    \begin{align*}
        M^H&=M^{\psi(\Gal(M'/K')}\\
        &= \set{ x\in M}{ \sigma(x)=x\ \forall\sigma\in\Gal(M'/K')}\\
        &= M\cap\set{ x\in M'}{ \sigma(x)=x\ \forall\sigma\in\Gal(M'/K')}\\
        &= M\cap K'
    \end{align*}
    Also ist \(H=\Gal(M/M\cap K')\)
\end{proof}
