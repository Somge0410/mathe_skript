\section{Gruppen}
\begin{Satz}[Gruppenordnung]
    Sei \(G\) eine Gruppe und \(H\subseteq G\) eine Untergruppe. Dann ist \(|G|=|H|\cdot [G:H]\)
\end{Satz}
\begin{proof}
    Es ist \([G:H]=\abs{G/H}=\set{aH}{a\in G}\) und \[G=\bigcup_{aH\in G/H}aH\] disjunkte Vereinigung, denn wenn \(g\in G\) dann ist \(g\in gH\) und \(aH\cap bH=\empty \iff aH=bH\) Es ist \(\abs{H}\not=\abs{aH}\) für alle \(a\in G\).
\end{proof}
\begin{Kor}[Satz von Lagrange]\label{Kor:Lagrange}
    Es gilt \(\abs{H}\mid\abs{G}\)
\end{Kor}
\begin{Kor}\label{Kor:Lagrange2}
Für \(a\in G\) ist \(\ord(a)\mid \abs{G}\). Also \(a^{\abs{G}}=e\).
\end{Kor}
\begin{Kor}\label{Kleiner Satz von Fermat}
    Sei \(p\) eine Primzahl und \(a\in\ZZ\) sodass \(p\not\mid a\). Dann ist \(a^{p-1}\equiv 1\mod p\).
\end{Kor}
\begin{proof}
    Sei \(G=\FF_p^*\). Dann ist \(\abs{G}=p-1\) also gilt \(\bar a^{p-1}=1\).
\end{proof}
\begin{Satz}[Struktursatz endlicher abelscher Gruppen] \label{Satz:StuktEndlAb}
Jede endliche abelsche Gruppe \(G\) mit \(\abs{G}\) lässt sich schreiben als 
\[\abs{G}\cong \ZZ/m_1\ZZ\times\dots\times\ZZ/m_r\ZZ\] mit \(m_1\mid\dots\mid m_r\)
\end{Satz}
\subsection{Bahnenformel und Klassengleichung}
\begin{Def}
    Sei \(G\) eine Gruppe, \(X\) eine Meng und \(G\times X\to X\) eine Gruppenwirkung.
    \begin{enumerate}
        \item Für \(x\in X\) ist \(Gx=\set{ax}{a\in G}\) die Bahn von \(x\).
        \item Für ist \(x\in X\) ist \(\Stab_G(x)=G_x=\set{a\in G}{ax=x}\) der Stabilisator von \(x\).
        \item Die Wirkung heißt transitiv, wenn \(X=Gx\) für ein \(x\in X\) ist.
        \item \(x\in X\) heißt Fixpunkt der Wirkung, wenn \(\Stab_G(X)=G\) ist.
        \item \(X^G\) sie die Menge der Fixpunkte von \(G\) auf \(X\).
        \item Die Operation heißt treu, wenn der Homomorphismus \(G\to S(X)\) injektiv ist.
    \end{enumerate}
\end{Def}
\begin{Bem}
    Sei \(G\times X\to X\) eine Gruppenwirkung. Dann ist \(X=\coprod _{i\in I}Gx_i\) für \(x_i\) ein Vertretersystem aller Bahnen.
\end{Bem}
\begin{Lemma}\label{Lem:Bahnengleichung1}
    Für \(x\in X\) gibt es eine natürliche Bijektion 
    \[G/\Stab_G(X)\to G_x, a\Stab_G(X)\mapsto ax\]
\end{Lemma}
\begin{proof}
    Man prüft, dass das wohldefiniert ist. Die Abbildung ist surjektiv nach Definition und wenn \(ax=bx\) ist, dann ist \(a^{-1}b\in\Stab_G(X)\) also \(a\Stab_G(X)=b=\Stab_G(X)\).
\end{proof}
\begin{Kor}
    Sei \(G\times X\to X\) eine Gruppenwirkung. Dann ist 
    \[\abs{Gx}=[G:\Stab_G(X)].\]
\end{Kor}
\begin{Satz}[Bahnengleichung]\label{Satz:Bahnengleichung}
    Sei \(G\times X\to X\) eine Gruppenwirkung wobei \(X\) eine endliche Menge ist. Seien \(x_1,\dots,x_r\) Verterter der Bahnen. Dann ist 
    \[\abs X=\sum_{i=1}^r[G\colon\Stab_G(X_i)].\]
\end{Satz}
\begin{proof}
    Da \(X=\coprod_{i=1}^rGx_i\) eine disjunkte Vereinigung ist, folgt 
    \[\abs X=\sum_{i=1}^r\abs{Gx_i}=\sum_{i=1}^r[G\colon\Stab_G(x_i)].\]
\end{proof}
\begin{Def}
    Sei \(G\) eine Gruppe. Dann ist das Zentrum von \(G\) definiert als
    \[Z(G)=\set{a\in G}{ab=ba\ \forall b\in G}.\] Das ist ein abelscher Normalteiler. \(Z(G)\) sind Fixpunkte der Operation \[G\times G\to G, (a,b)\mapsto aba^{-1}.\]
\end{Def}
\begin{Def}
    Der Zentralisator von \(a\in G\) ist \[C_G(a)=\set{b\in G}{ba=ab}.\] Das ist eine Untergruppe von \(G\) und \(a\in Z(C_G(a))\). Es ist \(a\in Z(G)\) genau dann wenn \(C_G(a)=G\). Es ist \(C_G(a)=\Stab_G(a)\) für die Wirkung \(G\times G\to G,\ (a,b)\mapsto aba^{-1}\)
\end{Def}
\begin{Satz}[Klassengleichung]\label{Satz:Klassengleichung}
    Sei \(G\) eine endliche Gruppe und \(a_1,\dots,a_r\) Verteter der Konjugationsklassen von nicht-zentralen Elementen. Dann ist 
    \[\abs G=\abs{Z(G)}+\sum_{i=1}^r[G:C_G(a_i)\]
\end{Satz}
\begin{proof}
    Das ist die \nameref{Satz:Bahnengleichung}} für die Wirkung \(G\times G\to G,\ (a,b)\mapsto aba^{-1}\), denn wenn \(x_i\) ein Vertreter einer Bahn mit zentralen Element, dann ist \(\Stab_G(x_i)=C_G(x_i)=G\)
\end{proof}
\begin{Def}
    Sei \(p\) eine Primzahl. Eine \(p\)-Gruppe ist eine Gruppe \(G\) mit \(\abs G=p^r\) für \(r\geq 0\). 
\end{Def}
\begin{Satz}\label{Satz:FixpPGrp}
    Sei \(G\times X\to X\) eine Wirkung einer \(p\)-Gruppe auf eine endliche Menge \(X\). Dann ist 
    \[\abs X\equiv \abs{X^G}\mod p.\]
\end{Satz}
\begin{proof}
    Nach \Cref{Satz:Bahnengleichung}} ist 
    \[\abs X=\abs{X^G}+\sum_{i=1}^r[G\colon \Stab_G(x_i)]\] wobei \(x_i\) Vetreter von Bahnen mit mindestens zwei Elementen sind. Da \(G\) eine \(p\)-Gruppe ist, gilt 
    \(p\mid [G:\Stab_G(x_i)]=\abs X-\abs{X^G}.\)
\end{proof}
\begin{Satz}\label{Satz:pGrpZentrum}
    Eine nicht-triviale \(p\)-Gruppe \(G\) hat ein nicht-triviales Zentrum.
\end{Satz}
\begin{proof}
    Betrachte Wirkung \(G\times G\to G, (a,b)\mapsto aba{-1}.\). Nach \Cref{Satz:FixpPGrp}} gilt \(\abs G=\abs{Z(G)}\mod p\). Also \(\abs{Z(G)}\geq p\).
\end{proof}

\begin{proof}
    Das ist der \nameref{Satz:HauptEndlMod} zusammen mit dem Fakt, dass endliche Gruppen Torsionsmoduln sind.
\end{proof}
\subsection{Sylowsätze}
Sei \(G\) eine endliche Gruppe und \(p\) eine Primzahl.
\begin{Def}
    Eine \(p\)-Sylowgruppe von \(G\) ist eine \(p\)-Gruppe \(P\subseteq G\) mit \(p\not\mid[G:P]\). Das heißt wenn \(\abs{G}=p^r\cdot m\) mit \(p\not\mid m\) dann ist \(P\subseteq G\) eine \(p\)-Sylowgruppe von \(G\) genau dann wenn \(\abs P=p^r\).
\end{Def}
\begin{Lemma}\label{Lem:CauchyGrp}
    Sei \(G\) eine endliche Gruppe und \(p\) eine Primzahl mit \(p\mid \abs G\). Dann gibt es ein \(a\in G\) mit \(\ord(a)=p\).
\end{Lemma}
\begin{proof}
    \(\ZZ/p\ZZ\) operiert auf der Menge \(M=\set{(g_1,\dots,g_p)\in G^p}{g_1\cdot g_2\dots \cdot g_p=e}\) durch \(k\cdot (g_1,\dots,g_p)=(g_{1+k\mod p},\dots,g_{p+k\mod p})\). Es ist \(\abs M=\abs G ^{p-1}\).
    Nach \Cref{Satz:FixpPGrp}} ist \[0\equiv \abs G^{p-1}\equiv \abs{M^{\ZZ/p\ZZ}}\mod p\]. Das heißt es gibt Fixpunkt \((g_1,\dots,g_p)\) wobei \(g_i\neq e\).
    Dann ist aber \(g_i=g_j\) für alle \(i,j\) also ist dieses Tupel \((g,\dots ,g)\) für ein \(g\in G\). Dann ist \(g^p=e\).
\end{proof}
\begin{Kor}\label{Kor:NormIndex}
    Sei \(G\) eine \(p\)-Gruppe, \(\abs{G}=p^n\). Dann gibt es für jedes \(k\leq n\) eine normale Untergruppe \(H\subseteq G\) der Ordnung \(p^k\).
\end{Kor}
\begin{proof}
    Da \(\abs{Z(G)}\mid p^n\) und \(Z(G)\) nach \Cref{Satz:pGrpZentrum}} nicht trivial ist, gibt es nach \Cref{Lem:CauchyGrp}} ein \(a\in Z(G)\) von der Ordnung \(p\). Wenn \(k=1\) dann ist \(N=\anglebr{a}\) die Lösung denn \(N\) ist normal. Wenn \(k>1\) ist, dann hat \(G/N\) hat Ordnung \(p^{n-1}\). Nach Induktion hat \(\bar G=G/N\) eine normale Untergruppe \(\bar H\) der Ordnung \(p^{k-1}\). Sei \(H=\pi^{-1}\bar H\) wobei \(\pi\colon G\to N\) die Projektion ist. Dann ist \(H/N\cong \bar H\) also ist \(H\) normal von Ordnung \(p^{k-1}\).
\end{proof}
\begin{Satz}[1. Sylowsatz]
    \(G\) hat eine \(p\)-Sylowgruppe.
\end{Satz}
\begin{proof}
    Sei \(\abs G=p^r\cdot m\) mit \(p\not\mid m\).
    Sei ohne Einschränkung \(r\geq 1\).
    Angenommen \(p\mid \abs{Z(G)}\) wobei \(Z(G)\) das Zentrum von \(G\) ist. Wähle \(a\in Z(G)\) mit \(\ord(a)=p\). Dann ist \(N=\anglebr{a}\subseteq G\) ein Normalteiler, da \(a\) zentral ist. Es ist \(\abs{G/N}=p^{r-1}\cdot m\) und nach Induktion gibt es eine \(p\)-Sylowgruppe \(Q\subseteq G/N\). Sei \(P=\pi^{-1}(Q)\) wobei \(\pi\colon G\to G/N\) die Projektion ist.
    Dann ist \(P/N\cong Q\) also \(\abs P=p^r\). Somit ist \(P\subseteq G\) eine \(p\)-Sylwogruppe.\\
    Angenommen \(p\not\mid \abs{Z(G)}\). Dann gibt es die \nameref{Satz:Klassengleichung}}
    \[\abs G=\abs{Z(G)}+\sum_{i=1}^r[G\colon C_G(a_i)].\]
    Also gibt es ein \(i\) sodass \(p\not\mid [G:C_G(a_i)]\).
    Da \[\abs G=\abs{C_G(a_i)}\cdot [G\colon C_G(a_i)]\] ist \(p^r\mid\abs{C_G(a_i)}\). Da \(a_i\) nicht zentral ist, ist \(C_G(a_i)\neq G\). Nach Induktion gibt es eine \(P\)-Sylowgruppe \(P\subseteq C_G(a_i)\).
\end{proof} 
\begin{Satz}[2. Sylowsatz]\label{Satz:2Sylow}
Sei \(P\subseteq G\) eine \(P\)-Sylowgruppe und \(H\subseteq G\) eine \(p\)-Gruppe. Dann gibt es ein \(a\in G\) sodass \(aHa^{-1}\subseteq P\).    
\end{Satz}
\begin{proof}
    \(H\) operiert auf \(G/P\) duch \(h\cdot (aP)=haP\). Nach \Cref{Satz:FixpPGrp}} ist 
    \[\abs{G/P}\equiv \abs{(G/P)^H}\mod p.\]
    Also ist \(p\not\mid \abs{(G/P)^H}\) und somit gibt es einen Fixpunkt \(aP\). Das heißt \(HaP=aP\) und somit \(a^{-1}HaP=P\) also \(a^{-1}Ha\subseteq P\).
\end{proof}
\begin{Kor}
    Wenn \(H\) eine weitere \(p\)-Sylowgruppe ist, dann folgt \(aHa^{-1}=P\) wegen Anzahl der Elemente. somit sind alle \(p\)-Sylowgruppen konjugiert.
\end{Kor}
\begin{Bem}
    Eine \(p\)-Sylowgruppe \(P\subseteq G\) ist normal \(\iff P\) die einzige \(p\)-Sylowgruppe ist.
\end{Bem}
\begin{Satz}[3. Sylowsatz]\label{Satz:3Sylow}
    Sei \(G=p^r\cdot m\) mit \(p\not \mid m\). Sei \(s\) die Anzahl der \(p\)-Sylowgruppen von \(G\). Dann gilt
    \begin{align}
        s&\equiv 1\mod p\\
        s&\mid m
    \end{align}
\end{Satz}
\begin{proof}
    Zeige 1). Sei \(P\) eine \(p\)-sylowgruppe. \(P\) operiert auf der Menge \(X\) der \(p\)-Sylowgruppen von \(G\) durch Konjugation \(P\times X\to X, (a,Q)\mapsto aQa^{-1}\).
    \nameref{Satz:Bahnengleichung}} liefert
    \[\abs X=\sum_{i=1}^r[P:\Stab_P(P_i)]\] wobei \(P_i\) Verteter der Bahnen sind. Sei \(P=P_1\). Dann ist \([P\colon\Stab_P(P_1)]=1\) und für \(i\geq 2\) gilt, dass \(P_i\) normal ist in \(H_i=\set{a\in G}{aP_ia^{-1}=P_i}=\Stab_G(P_i)\). Also ist \(P_i\) die einzige \(p\)-Sylowgruppe von \(H_i\). Also ist \(P\subsetneq H_i\) und damit \(P\neq \Stab_P(P_i)\) also \(p\mid [P\colon\Stab_P(P_i)]\). Damit ist \[S=\abs X=1+\sum_{i\geq 2}^k[P\colon \Stab_G(P_i)]\equiv 1\mod p.\]\\
    Zeige 2). Sei \(X\) wie oben. Wegen \nameref{Satz:2Sylow}} operiert \(G\) transitiv auf \(X\) durch Konjugation.
    Sei \(H=\Stab_G(P)\) der Normalisator von \(P\). Es ist \(P\subseteq H\) und somit \(p^r\mid H\). Es ist \(\abs H=p^r\cdot \ell\) mit \(p\not\mid \ell\). Nach \nameref{Lem:Bahnengleichung1}} ist
    Dann ist \(s=\abs X=\text{Bahn}(P)=\abs G/\abs H=\frac{m}{\ell}.\) Also ist \(s\) ein Teiler von \(m\).
\end{proof}
\begin{Satz}
    Seien \(p,q\) Primzahlen und \(p<q\)  und \(p\not\mid q-1\). Dann ist jede Gruppe der Ordnung \(p\cdot q\) zyklisch.
\end{Satz}
\begin{proof}
    Sei \(s\) die Anzahl der \(p\)-Sylwogruppen und \(t\) die Anzahl der \(q\)-Sylowgruppen. Nach \nameref{Satz:3Sylow} gilt \(s=t=1.\) Seien also \(P\) eine \(p\)-Sylowgruppe und \(Q\) eine \(q\)-Sylowgruppen.
    Es ist \(P\cap Q=\Set e\) denn die Ordnung von \(P\cap Q\) teilt \(p\) und \(q\) nach \Cref{Kor:Lagrange2}. Also gilt für \(a\in P, b\in Q\) dass \(ab=ba\), denn \(b^{-1}aba^{-1}\in P\cap Q\) wegen Normalität von \(P\) und \(Q\). Es sind \(P,Q\) zyklisch. Seien also \(a\in P\) und \(b\in Q\) Erzeuger. Es ist 
    \[(ab)^n=a^nb^n=e\iff a^n=b^{-n}\iff a^n=e=b^n\iff pq\mid n\]
    Also ist \(\ord(ab)=pq=\abs G\) und somit \(G=\anglebr{ab}.\)
\end{proof}
\begin{Lemma}
    Jede endliche abelsche Gruppe ist isomorph zum Produkt ihrer Sylowgruppen.
\end{Lemma}
\begin{proof}
    Sei \(\abs G=\prod_{i=1}^rp_i^{e_i}\) wobei \(p_i\) verschiedene Primzahlen sind und \(e_i\geq 1\).
    Da \(G\) abelsch ist, sind alle Untergruppen normal somit gibt es genau eine \(p_i\)-Sylowgruppe \(P_i\subseteq G\). Die Abbildung 
    \[f\colon\prod_{i=1}^rP_i\to G,\ (a_1,\dots,a_r)\mapsto a_1\cdot\dots\cdot a_r\] 
    ist Gruppenhomomorphismus. Wegen der Gruppenordnung ist \(f\) bijektiv wenn \(f\) 
    injektiv ist. Angenommen \(a_i\in P_i\) sodass \(a_1\cdot\dots\cdot a_r=e\).
     Es ist \(a_1=(a_2\cdots a_r)^{-1}\) und 
     \(\ord(a_1)=\ord(a_2\cdots a_r)\mid\prod_{i=2}^rp_i^{e_i}\).
    Somit \(\ord(a_1)\mid \ggT(p_1^{e_1},\prod_{i=2}^rp_i^{e_i})=1\). Also \(e_1=e\) und genauso \(a_i=e\) für alle \(i\). Also ist \(f\) injektiv.
\end{proof}
\begin{Lemma}[Chinesischer Restsatz]\label{Lem:ChinRest1}
    Seien \(n,m\in\ZZ\) und \(d=\ggT(n,m)\) und \(k=\kgV(n,m)\).
    Dann ist \(\ZZ/n\ZZ\times\ZZ/m\ZZ\cong \ZZ/k\ZZ\times\ZZ/d\ZZ.\)
\end{Lemma}
\begin{proof}
    Sei \(xn+ym=d\) für \(x,y\in \ZZ\).
    Definiere \[f\colon \ZZ\times \ZZ\to\ZZ/n\ZZ\times\ZZ/m\ZZ, (a,b)\mapsto (a-b\frac{xn}{d}+n\ZZ,a+b\frac{ym}{d})+m\ZZ.\]
    Angenommen \(f(a,b)=(0,0)\). Dann ist \[b\frac{xn}{d}\equiv a \mod n\] und \[-b\frac{ym}{d}\equiv a\mod m.\] Also gilt \[b\frac{xn}{d}\equiv -b\frac{ym}{d} \mod d\] und somit \[b(\frac{xn}{d}+\frac{ym}{d})\equiv 0 \mod d.\] Da \[\frac{xn}{d}+\frac{ym}{d}\equiv 1\mod d\] gilt, ist \[b\equiv 0 \mod d.\] Sei also \(b=s\cdot d\). Dann ist \[a\equiv \frac{xn}{d}\cdot b=xns\equiv 0\mod n\] und analog \(a\equiv 0 \mod n\). Also ist \(a\equiv 0\mod k\). Das zeigt \(\ker(f)=k\ZZ\times d\ZZ\) und somit ist \(\bar f\colon \ZZ/k\ZZ\times \ZZ/d\ZZ\to \ZZ/n\ZZ\times \ZZ/m\ZZ\) injektiv und damit surjektiv wegen gleicher Anzahl der Elemente.
\end{proof}

\subsection{Semidirekte Produkte}
\begin{Def}
    Sei \(G\) eine Gruppe und \(N,H\subseteq  G\) Untergruppen sodass \(N\) normal ist und \(H\cap N=\Set e\) und \(G=NH\). Dannn heißt \(G\) das innere semidirekte Produkt von \(N\) und \(H\).
\end{Def}
\begin{Bem}
    Wenn \(G\) das semidirekte Produkt von \(N\) und \(H\) ist, dann ist die Abbildung \(N\times H\to G,\ (n,h)\mapsto n\cdot h\) bijektiv. Das Verleiht \(N\times H\) mit der Gruppenstruktur \((n,h)\cdot (n',h')=(n\cdot hn'h^{-1},h\cdot h')\). Das heißt wenn \[\alpha\colon H\to\Aut(N),\ \alpha(h)=(N\to N, n\mapsto hn'h^{-1})\] dann ist \((n,h)\cdot(n',h')=(n\cdot\alpha(h)(n'),hh')\)
\end{Bem}
\begin{proof}
    Wenn \(nh=n'h'\) dann ist \(n^{-1}n=h^{'-1}\in N\cap H=\Set e\).
\end{proof}
\begin{Def}
    Seien \(N,H\) Gruppen und \(\alpha\colon H\to\Aut(N)\) ein Gruppenhomomorphismus. Definiere \(N\rtimes H=N\rtimes_\alpha H\) als Menge \(N\times H\) und Gruppenstuktur \((n,h)(n',h')=(n\alpha(h)(n'),hh')\). Das ist eine Gruppe. (Z.B. ist \((n,h)^{-1}=(\alpha(h^{-1})(n^{-1}),h^{-1})\)). \(N\rtimes_\alpha H\) heißt semidirektes Produkt von \(N\) und \(H\) bezüglich \(\alpha\).
    Es ist \(N\subseteq N\rtimes H, n\mapsto (n,e)\) und \(H\subseteq N\times H,\ h\mapsto (e,h)\).
\end{Def}
\begin{Lemma}
    \(N\subseteq N\rtimes H=G\) ist normale Untergruppe, \(G=NH\) und \(N\cap H=\Set e\).
\end{Lemma}
\begin{proof}
    Es sei \(\tau=(\star,g)\in G\). Dann ist \(\tau^{-1}=(\star,g^{-1})\).
    Es ist \(\tau(n,e)\tau^{-1}=\tau\cdot(\star,g^{-1})=(\star,e)\in N\) also ist \(N\) normal. 
    Sei \(\tau=(n,h)\in G\). Dann ist \((n,e)\cdot (e, h)=\tau\).
\end{proof}
\begin{Def}
    Sei \(\Grpact\) die Kategorie mit Objekten \((H,N,\alpha)\) wobei \(H,N\) Gruppen sind und \(\alpha\colon H\to\Aut(N)\) ein Gruppenhomomorphismus ist. Morphismen \[(H,N,\alpha)\to (H',N',\alpha')\] in \(\Grpact\) sind \((f_H,f_N)\) wobei \(f_H\colon H\to H'\)  und \(f_N\colon N\to N'\) \(H\)-equivariante Gruppenhomomorphismen sind. Das heißt
    \(f_N(\alpha(h)(n))=\alpha'(f_H(h))(f_N(n))\) for all \(h\in H\) and \(n\in N\).
\end{Def}
\begin{Lemma}\label{Lem:AdjSemDir}
    Sei \(\Arr(\Grp)\) die \Cref{Arrow Kategorie} von \(\Grp\). Der 
    \[R\colon \Arr(\Grp)\to\Grpact,\ (\phi\colon H'\to N')\mapsto (H',N',\alpha')\] wobei \(\alpha'\) die Konjugation in \(N\) ist, das heißt \(\alpha'(h')(n')=\phi(h')n'\phi(h')^{-1}\) für alle \(h'\in H'\) und \(n'\in N'\).\\
    Der Funktor \[L\colon\Grpact\to\Arr(\Grp), (H,N,\alpha)\mapsto (H\subseteq N\rtimes_\alpha H)\] ist linksadjungiert zu \(U\).
\end{Lemma}
\begin{proof}
     Sei % https://tikzcd.yichuanshen.de/#N4Igdg9gJgpgziAXAbVABwnAlgFyxMJZABgBoBGAXVJADcBDAGwFcYkQAJEAX1PU1z5CKchWp0mrdgDkAOrIBOeALbwA+vKZoAFvQAEXXv2x4CRMgCZxDFm0ScA5Dz4gMJoUVFWaNqfelORq4CpsLIosTWknYg0jziMFAA5vBEoABmChDKSBY0OBBIAMw+0ezyOljOGVk5iGQgBbmltuzpaoYumdlIDU2Ioo30WIzs2hAQANYgLX4gWB3VIN11g-0lEq327cByiirwBtxLK0gALPmFA-nDo-bjUzObcwtx3JTcQA
\begin{tikzcd}
                                          & N \arrow[d, "i_N", hook,dashed]                      \\
H \arrow[d, "f_H"] \arrow[r, "i_H", hook] & N\rtimes_\alpha H \arrow[d, "f_{{N\rtimes H}}"] \\
H' \arrow[r, "\phi"]                      & N'                                           
\end{tikzcd} ein kommutatives Diagramm von Gruppen.
     Definiere \(f_N=i_N\circ f_{N\rtimes H}\colon N\to N'\) und sei \((H',N',\alpha')=R(\phi)\).
     Dann ist \[(f_H,f_{N\rtimes H}\circ i_N)\colon (H,N,\alpha)\to (H',N',\alpha')\] ein Morphismus in \(\Grpact\), denn
     \[\alpha'(f_H(h))(f_{N\rtimes H}(n,e))=f_{N\rtimes H}((e,h)\cdot (n,e)\cdot (e,h)^{-1})\]
      und da \[(e,h)(n,e)(e,h)^{-1}=(\alpha(h)(n),h)(e,h^{-1})=(\alpha(h)(n),e)\]
      ist \[\alpha'(f_H(h))(f_{N}(n))=f_N(\alpha(h)(n))\] für alle \(h\in H\) und \(N\in N\).\\
      Wenn andersrum \(\phi\colon H'\to N'\) Gruppenhomomorphismus ist und \[(f_H,f_N)\colon(H,N,\alpha)\to R(\phi)=(H',N',\alpha')\] ein Morphismus in \(\Grpact\) ist, dann definiere \(f_{N\rtimes H}\colon N\rtimes_\alpha H\to N'\) durch \[f_{N\rtimes H}(n,h)=f_N(n)\phi(f_H(h)).\]
      Man prüft dass das folgende Diagramm kommutiert, das heißt wir haben einen Morphismus in \(\Arr(Grp)\):
      % https://tikzcd.yichuanshen.de/#N4Igdg9gJgpgziAXAbVABwnAlgFyxMJZABgBpiBdUkANwEMAbAVxiRAAkQBfU9TXfIRQBGclVqMWbAHIAdWQCc8AW3gB9eYzQALOgAJOPPtjwEiZYePrNWiDgHJuvEBhOCioy9WtS70x1ziMFAA5vBEoABmChDKSABM1DgQSADM3pK2IPI6WE5RMXGIZCDJCRk2bJFqhs7RsUglZYiipXRYDGzaEBAA1iAVviBYNfkg9UWtzekSlXbVwHKKKvAGXNwUXEA
\begin{tikzcd}
H \arrow[d, "f_H"] \arrow[r, "i_H", hook] & N\rtimes_\alpha H \arrow[d, "f_{N\rtimes H}"] \\
H' \arrow[r, "\phi"]                      & N'                                           
\end{tikzcd} Man prüft das diese Zuordnungen einen natürlichen Isomorphismus der \(\Hom\)-Funktoren definieren.
      
     \end{proof}
\begin{Lemma}
    Sei \(\codom\colon \Arr(\Grp)\to\Grp\) der Codomain Funktor aus \Cref{Lem:CodomAdj}} mit Rechtadjungiertem \(R'\colon\Grp\to\Arr(\Grp)\).
    Seien \(L,R\) wie in \Cref{Lem:AdjSemDir}}. Dann ist \[\codom\circ L\colon \Grpact\to \Grp,\ (H,N,\alpha)\mapsto N\rtimes_\alpha H\] linksadjungiert zu 
    \[R\circ R'\colon \Grp\to\Grpact,\ G\mapsto (G,G,\alpha')\] wobei \(\alpha'(h)(g)=hgh^{-1}\) für alle \(h,g\in G\).
\end{Lemma}
\begin{proof}
    Klar.
\end{proof}
    
\subsection{Einfache und auflösbare Gruppen}
Sei \(G\) eine Gruppe.
\begin{Def}
    \begin{enumerate}
        \item []
        \item Eine Normalreihe in \(G\) ist eine Folge \[\Set e=G_0\subseteq G_1\subseteq \dots\subseteq G_n=G\] sodass \(G_i\subseteq G\) eine normale Untergruppe ist.
        \item Eine abelsche Normalreihe ist eine Normalreihe sodass \(G_i/G_{i-1}\) abelsch ist für alle \(i\).
        \item \(G\) ist auflösbar, wenn \(G\) eine abelsche Normalreihe hat.
    \end{enumerate}
\end{Def}
\begin{Bsp}
    Wenn \(G\) abelsch ist, dann ist \(G\) auflösbar denn \(G_0=\Set e\subseteq G_1=G\) ist abelsche Normalreihe.
\end{Bsp}
\begin{Lemma}\label{Lem:AuflUntQuot}
    Sei \(H\subseteq G\) normal. Es gilt 
    \[H \text{ auflösbar und } G/H \text{ auflösbar}\implies G \text{ auflösbar}\]
\end{Lemma}
\begin{proof}
    Sei \[\Set e=H_0\subseteq H_1\subseteq \dots\subseteq H_n=H\] eine abelsche Normalreihe und \[\Set e=\bar G_n\subseteq\bar G_{n+1}\subseteq\dots \subseteq \bar G_m=G/H\] eine abelsche Normalreihe. Sei \(\pi\colon G\to G/H\) Projektion und setze \(G_i=H_i\) für \(0\leq i\leq n\) und \(G_i=\pi^{-1}\bar G_i\)  für \(n\leq i\leq n\).
    Dann bilden die \(G_i\) eine abelsche Normalreihe.
\end{proof}
\begin{Satz}
    Seien \(p<q\) Primzahlen. Jede Gruppe der Ordnung \(p\cdot q\) ist auflösbar.
\end{Satz}
\begin{proof}
    Sei \(s\) die Anzahl der \(q\)-Sylowgrupen. Dann ist \(s\equiv 1\mod q\) und \(s\mid p\). Also ist \(s=1\) denn \(p<q\). Somit ist eine \(q\)-Sylowgruppe \(Q\subseteq G\) normal und \(G/Q\cong \ZZ/p\ZZ\) und \(Q\cong\ZZ/q\ZZ\). Dann ist \(\Set e\subseteq Q\subseteq G\) eine abelsche Normalreihe.
\end{proof}
\begin{Satz}
    Jede \(p\)-Gruppe \(G\) ist auflösbar
\end{Satz}
\begin{proof}
    Wenn \(G\neq \Set e\) ist, dann ist \(Z(G)\neq \Set e\). \(Z(G)\) ist abelscher Normalteiler und somit auflösbar. Nach Induktion ist \(G/Z(G)\) auflösbar. Nach \Cref{Lem:AuflUntQuot}} folgt die Aussage.
\end{proof}
\begin{Def}
    Sei \(G\) eine Gruppe. Der Kommutatot von \(a,b\in G\) ist \(aba^{-1}b^{-1}=[a,b]\).
    Die Kommutatorgruppe oder derigierte Gruppe von \(G\) ist \(D(G)=[G,G]=\anglebr{\set{[a,b]}{a,b\in G}}\). Die Abelisierung von \(G\) ist \(G^{\ab}=G/D(G)\).
\end{Def}
\begin{Lemma}
    \(D(G)\) ist normaler Untergruppe und \(G^{\ab}\) ist abelsch.
\end{Lemma}
\begin{proof}
    Klar, Rechnung
\end{proof}
\begin{Satz}[Universelle Eigenschaft der Abelisierung]
    Abelisierung wird zu einem Funktor \((-)^\ab\colon\Gr\to\Ab\) der linksadjungiert ist zum Vergiss-Funktor \(U\colon \Ab\to \Grp\)
\end{Satz}
\begin{proof}
    Sei \(f\colon G\to U(H)\) Gruppenhomomorphismus. Da \(U(H)\) abelsch ist, ist \(D(G)\) im Kern von \(f\). Das induziert also \(\bar f\colon G/D(G)\to H\).
\end{proof}
\begin{Def}
    Sei \(D^n(G)=D(D^{n-1}(G))\) und \(D^0(G)=G.\)
\end{Def}
\begin{Bem}
    Wenn \(G=G_0\supseteq G_1\supseteq\dots\) abelsche Normalreihe ist, dann gilt \(D^n(G)\subseteq G_n\) für alle \(n\).
\end{Bem}
\begin{proof}
    Da \(G_n/G_{n+1}\) abelsch ist, ist \(D(G_n)\subseteq G_{n+1}\). Damit folgt die Aussage per Induktion.
\end{proof}
\begin{Satz}
    \(G\) ist auflösbar \(\iff D^n(G)=\Set e\) für ein \(n\in\NN\).
\end{Satz}
\begin{proof}
    Wenn \(G\) auflösbar ist, dann ist \(G=G_0\supseteq\dots\supseteq G_n=\Set e\) und somit \(D^n(G)\subseteq \Set e\).
    Wenn \(D^n(G)=\Set e\), dann ist \(G\supseteq D(G)\supseteq\dots\supseteq D^n(G)\) eine abelsche Normalreihe.
\end{proof}
\begin{Bsp}
    \(S_3\) ist auflösbar, denn für \(p=\anglebr{(1,2,3)}\subseteq S_3\) ist \(G/P\cong \ZZ/2\ZZ\) und somit \(\Set e, P,S_3\) eine abelsche Normalreihe.
    Es gibt surjektiven Homomorphismus \(\psi\colon S_4\to S_3\) mit \(\ker(\psi)\cong \ZZ/2\ZZ\times\ZZ/2\ZZ\). Das ist abelsch und da \(S_4/\ker(\psi)=S_3\) auflösbar ist, ist \(S_4\) auflösbar.
\end{Bsp}
\begin{Satz}
    \(A_n\) ist nicht auflösbar für \(n\geq 5\).
\end{Satz}
\begin{proof}
    Es ist \[[(1,2,3),(3,4,5)]=(1,4,3).\] Somit enthält \(D(A_n)\) alle 3-Zykel, also \(D(A_n)=A_n\) und \(D^m(A_n)\neq\Set e\) für alle \(m\).
\end{proof}
\begin{Def}
    Eine Gruppe \(G\) heißt einfach, wenn \(G\neq \Set e\) und \(G\) keine Normalteler außer \(\Set e\) und \(G\) hat.
\end{Def}
\begin{Bem}
    Wenn \(G\) abelsch und einfach ist, dann ist \(G\cong \ZZ/p\ZZ\) für eine Primzahl \(p.\)
\end{Bem}
\begin{proof}
    Es gibt \(a\in G\) sodas \(\ord(a)=p\) prim ist. Sei \(H=\anglebr{a}\subseteq G\). Dann ist \(H\) normal und da \(H\neq \Set e\) ist \(H=G\).
\end{proof}
\begin{Bem} Wenn \(G\) einfach ist und nicht abelsch, dann ist \(G\) nicht auflösbar.
\end{Bem}
\begin{Bem}
    In \(A_n\) mit \(n\geq 5\) sind alle 3-Zykel konjugiert. Denn sei \(\tau_1=(a \ b\ c)\) und \(\tau_2=(1\ 2\ 3)\). Sei \(\sigma(1)=a, \sigma(2)=b, \sigma(3)=c\) und setzte \(\sigma\) fort zu \(\sigma\in A_n\). Dann ist \(\sigma\tau_2\sigma^{-1}=\tau_1\).
\end{Bem}
\begin{Satz}
    Für \(n\geq 5\) ist \(A_n\) einfach.
\end{Satz}
\begin{proof}
    Sei \(N\subseteq A_n\) normale Untergruppe und \(N\neq\Set e\).
    Sei \(\sigma\in N\) sodass die Anzahl der Fixpunkte von \(\sigma\) maximal ist und \(\sigma\neq e\).
    Behauptung: Alle Zykel in der Zykeldarstellung von \(\sigma\) haben die gleiche Länge \(d\). Denn wenn Längen \(m<d\) vorkommen, dann hat \(\sigma^m\) mehr Fixpunkte und \(\sigma^m\neq e\).
    Behauptung: \(\sigma\) ist ein 3-Zykel. Dann enthält \(N\) als normale Untergruppe alle \(3\)-Zykel, somit \(N=A_n\).
    Sei \(\sigma=(a\ b \ c\dots)(\dots)\dots\). 
    Wenn \(d\geq 3\) bilde \(s=(\tau \sigma\tau^{-1})\sigma^{-1}\) für \(\tau=(a\ b)(d\ e)\). Dann ist \(s=(c\ d\ e)\) ein 3-Zykel. Wegen Maximalität ist dann \(\sigma\) dieser 3-Zykel.
    Wenn \(d=2\) dann ist \(\sigma=(a\ b)(c \ d)\dots=\sigma^{-1}\).
    Sei \(\tau=(b\ c)(e\ f)\). Dann ist \(\tau\sigma\tau^{-1}\sigma^{-1}=(a\ d)(b\ c)\). Wegen der Maximalität von \(\sigma\) ist \(\sigma=(a\ b)(c\ d)\)
    Dann ist aber für \(\tau=(d\ e)(a\ b)\) der Folgende 3-Zykel \(\tau\sigma\tau^{-1}\sigma^{-1}=(c\ d\ e)\) was ein Widerspruch zur Maximalität ist.
\end{proof}
\begin{Satz}[Burnside 1911]
    Jede endliche Gruppe der Ordnung \(p^aq^b\) wobei \(p,q\) Primzahlen sind ist auflösbar.
\end{Satz}
\begin{Satz}[Feit-Thomson 1963]
    Jede Gruppe ungerader Ordnung ist auflösbar.
\end{Satz}