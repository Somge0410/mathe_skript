\chapter{Grundlagen}
\section{Quadratzahlen \( \mod p\)}
\begin{Bem}
    Sei \(n=k\cdot \ell\) natürliche Zahlen. Dann ist 
    % https://tikzcd.yichuanshen.de/#N4Igdg9gJgpgziAXAbVABwnAlgFyxMJZABgBpiBdUkANwEMAbAVxiRAB12AtLgejE48QAX1LpMufIRQBGclVqMWbQXwHcuIsSAzY8BIgCZ51es1aIOG-qpEKYUAObwioAGYAnCAFskZEDgQSHKK5irsAMZQEDgABADWWu5evoghgUjGocqWnFExsQx2wkA
\begin{tikzfigure} 
\ZZ/n\ZZ \arrow[r, "\cdot k"] & \ZZ/n\ZZ \arrow[r, "\cdot \ell"] & \ZZ/n\ZZ
\end{tikzfigure} exakt, das heißt
    \[\{a\in\ZZ/n\mid \ell\dot a=0\} =\{k\cdot b\mid b\in\ZZ/n\ZZ\}\]
    denn für \(x\in \ZZ\) ist \(n\mid \ell\dot x \iff k\mid x\).
\end{Bem}
\begin{Bem}
    Es ist \((\FF_p^*,\cdot)\cong (\ZZ/(p-1),+),\, 1\mapsto 0\) nach Satz ???.
    Wenn also \(p-1=n=k\dot \ell\) dann ist
    \(\{a\in\FF_p^*\mid a^k=1\}=\{b^\ell\mid b\in\FF_p^*\}\).
    Wenn also \(p\geq 3\) dann ist \(p-1=2\cdot \frac{p-1}{2}\).
    Also ist \[\{b^2\mid b\in\FF_p^*\}=\{a\in\FF_p^*\mid a^{\frac{p-1}{2}}=1\}.\]
\end{Bem}
\begin{Def}[Legendre Symbol]
    Sei \(p\) prim und \(a\in \ZZ\). Dann ist das Legendre Symbol gegeben durch 
    \[\left( \frac a p\right)=\begin{cases}
        0 & p\mid a\\
        1 & p\not\mid a \text{ und } a \text{ Quadratzahl in} \FF_p\\
        -1 & p\not\mid a \text{ und } a \text{ keine Quadratzahl in} \FF_p\\
    \end{cases}\]
\end{Def}
\begin{Lemma} \[\left(\frac a p\right)\equiv a^{\frac{p-1}{2}}\mod p\]
\end{Lemma}
\begin{proof}
    Wenn \(\left(\frac a p\right)=-1\) dann ist \((a^{\frac{p-1}{2}})^2=a^{p-1}=1\) in \(\FF_p\).
    also \(a^{\frac{p-1}{2}}\in \{\pm 1\}\) aber \(a^{\frac{p-1}{2}}\neq 1\) da sonst \(a\) ein 
    Quadrat. 
\end{proof}
\begin{Lemma}
    \[\left(\frac{-1}{p}\right)=\begin{cases}
        1 & 4\mid p-1 \iff p\equiv 1 \mod 4\\
        -1 & 4\not\mid p-1 \iff p\equiv 3\mod 4
    \end{cases}\]
\end{Lemma}

\begin{Lemma}
    Wenn \(p\equiv 1 \mod 4\) dann ist \(-1\in \FF_p\) eine Quadratzahl.
\end{Lemma}
\begin{proof}
    
\end{proof}
\section{Primzahlen in \(\ZZ[i]\)}
\begin{Bem}
    Für \(z\in\ZZ[i]\) definiere \(N(z)=z\cdot\bar{z}\). Damit wird \(\ZZ[i]\) ein Euklidischer 
    Ring bzgl \(N\).
    Es ist \(N(zw)=N(z)=N(w)\) und 
    \(\ZZ[i]^*=\{z\in \ZZ[i]\mid  N(z)=1\}=\{\pm 1,\pm i\}.\)
\end{Bem}
\begin{Satz} Eine Primzahl \(p\in \NN\) hat eine Darstellung \(p=a^2+b^2\) mit \(a,b\in \NN\) genau dann wenn
    \(p=2\) oder \(p \equiv 1 \mod 4\).
\end{Satz}
\begin{proof}
    Für alle \(x\in \NN\) gilt \(x^2\equiv 0 \mod 4\) oder \(x^2\equiv 1\mod 4\).
    wenn also \(p=x^2+y^2\) ist, dann ist \(p \equiv\begin{cases}
        0 \\ 
        1\\
        2\\
    \end{cases} \mod 4\).
    Also ist eine Richtung gezeigt.
    Sei andererseits \(p\equiv 1 \mod 4\). Wähle \(x\) sodass \(x^2\equiv -1 \mod p\) nach ??
    Das heißt \(p\mid x^2+1=(x+i)(x-i)\) in \(\ZZ[i]\).
    Aber \(p\not \mid x\pm i\) da \(\frac{x\pm i}{p}=\frac x p \pm \frac i p\not\in\ZZ[i]\).
    Somit ist \(p\) nicht prim in \(\ZZ[i]\).
    Da \(\ZZ[i]\) Euklidisch, ist es faktoriell nach ?? somit ist \(p\) nicht irreduzibel.
    das heißt \(p=z\cdot w\) für \(z,w\in\ZZ[i]\) keine Einheiten.
    Dann ist \(p^2=N(p)=N(Z)N(w)\) und \(N(Z),N(w)\neq 1\) also \(p=N(z)=N(w)\).
    Wenn \(z=a+bi\) dann ist \(p=N(z)=a^2+b^2\).
\end{proof}
\begin{Satz}
    Primelemnte in \(\ZZ[i]\) sind 
    \begin{enumerate}
        \item Primzahlen \(p\in\ZZ\) mit \(p\equiv 3\mod 4\)
        \item \(a+bi\in\ZZ[i]\) mit \(a^2+b^2=p\) und \(p=2\) oder \(p\equiv 1 \mod 4\).
    \end{enumerate}
\end{Satz}
\begin{proof}
    Ähnliche Rechnung.
\end{proof}
\begin{Bsp}[Pellsche Gleichung]
    Fixiere \(N\in \NN\) und \(N\) sei keine Quadratzahl.
    Suche \(a,b\in\NN\) sodass \(a^2-Nb^2=1\).
    
\end{Bsp}
\begin{Lemma} Sei \(R=\ZZ[\sqrt{N}]=\{a+b\sqrt{N}\mid a,b\in\ZZ\}\).
    \(z=a+b\sqrt{N}\in R^*\iff a^2-Nb^2=\pm 1\)
\end{Lemma}
\begin{Bsp}
    Sei \(R=\ZZ[\sqrt{2}]\). Die Norm von \(x=a+b\sqrt{2}\) ist definiert als 
    \(N(x)=(a+b\sqrt{2})(a-b\sqrt{2})=a^2-2b^2\).
    Es ist \(x\in R\) eine Einheit genau dann wenn \(N(x)=\pm 1\).
    Denn wenn \(x\cdot \bar x=N(x)=\pm 1\) dann ist \(\pm \bar x\) das Inverse zu \(x\), wobei 
    \(\bar x=a-b\sqrt{2}\) ist.
    Es ist jede Einheit \(u=a+b\sqrt{2}\) von \(R\) von der Form \(\pm (1+\sqrt{2})^n\) für \(n\in \ZZ\).
    
    Induktion über \(n=|a+b|\).
    Wenn \(n=0\) dann ist \(a=-b\) sodass \(u=b(-1+\sqrt{2})\). Da \(-1+\sqrt{2}=(1+\sqrt{2})^{-1}\) ist muss
    \(b=\pm 1\) sein also hat es die gewünschte Form.
    Sei \(n=|a+b|>0\)
    Wenn \(a,b\) verschiedene Vorzeichen haben sei \(u'=\frac{a-b\sqrt{2}}{N(u)}=\pm (a-b\sqrt{2})\)
    und \(uu'=\pm 1\) und ersetze \(u\) durch \(u'\).
    Wenn \(a,b\) beide negativ sind, ersetze \(u\) durch \(-u\).
    Also ohne Einschränkung \(a,b>0\). 
    Dann ist \(u'=u(-1+\sqrt{2})=(-a+2b)+(a-b)\sqrt{2}\) eine Einheit, da \(-1+\sqrt{2}=(1+\sqrt{2})^{-1}\)
    Sei \(c=-a+2b\) und \(d=a-b\). Dann ist \(|c+d|=|b|<|a+b|\).
    Nach Induktion hat \(u'\) die gewünschte Form und damit auch \(u\).
    Das zeigt, dass \(\ZZ/2\ZZ\times \ZZ\cong R^*,\, (n+2\ZZ,m)\mapsto (-1)^n(1+\sqrt{2})^m\).
\end{Bsp}
\section{Zahlkörper}
\begin{Def}
    Ein Zahlkörper ist eine endliche Erweiterung \(K/\QQ\). Wenn \(K\) ein Zahlkörper ist, dann
    ist der Ring der ganzen Zahlen von \(K\) gegeben durch
    \[\OO_K=\{z\in K\mid z \text{ ganz über }\ZZ\}\]
    Also \(\OO_K\) ist der ganze Abschluss von \(\ZZ\) in \(K\).
\end{Def}
\begin{Bsp}
    Sei \(K\) ein quadratischer Zahlkörper, \(K=\QQ(\sqrt{d})\) für ein quadratfreies \(d\in \ZZ\).
    Sei \(x=a+b\sqrt{d}\in K\) mit \(a,b\in \QQ\)
    Wenn \(b=0\) dann ist \(x\) ganz über \(ZZ\) genau dann wenn \(a\in \ZZ\).
    Wennn \(b\neq 0\) dann ist das Minimalpolynom von \(x\) gegeben durch 
    \(X^2-2aX+(a^2-db^2)\).
    \(x\) ist nach ??? ganz über \(\ZZ\) genau dann wenn \(-2a, a^2+-db^2\in \ZZ\) also genau dann
    wenn \(a,b\in \ZZ\) oder \(a,b\in \frac 1 2+\ZZ\) und \(d\equiv 1 \mod 4\).
    Das heißt \[\OO_K=\begin{cases}
        \ZZ[\sqrt{d}] & d\not\equiv \mod 4\\
        \ZZ[\frac{\sqrt{d}+1}{2}] & d\equiv 1 \mod 4
    \end{cases}\]
\end{Bsp}
\begin{Bem}
    Aus ??? folgt, dass \(\OO_K\) ein endlicher \(\ZZ\)-Modul ist.
    Da \(\OO_k\) torsionsfrei ist folgt mit ???
    \(\O_K=\ZZ^n\) als \(\ZZ\)-Modul.
\end{Bem}
\begin{Def} Eine \(\ZZ\)-Basis von \(\OO_K\) heißt Ganzheitsbasis von \(K\).
\end{Def}
\begin{Lemma}
    Jede Ganzheitsbasis ist eine \(\\Q\)-Basis von \(K\) und \(\OO_K\cong \ZZ^n\) mit 
    \(n=[L:K]\).
\end{Lemma}
\begin{proof}
    Seien \(a_1,\dots,a_n\in\OO_K\) eine Ganzheitsbasis.
    Dann sind sie auch \(\\Q\)-linear unabhängig.
    Sei \(y\in K\) mit \(f\in \QQ[X]\) Minimalpolynom.
    Sei \(b\) der gemeinsame Nenner aller Koeffienten von \(f\).
    Dann ist \(by\) ganz also in \(\OO_K\).
    somit ist \(y\in \QQ\cdot \OO_K\).
\end{proof}
\begin{Bsp}
    Wenn \(K=\QQ(\sqrt{d})\) mit \(d\) quadratfrei, dann ist
    falls \(d\equiv 1 \mod 4\) eine Ganzheitsbasis gegeben durch 
    \(1,\frac{\sqrt{d}+1}{2}\) und falls \(d\not\equiv \mod 4\) gegeben durch 
    \(1,\sqrt{d}\).
\end{Bsp}
\begin{Def}
    Sei \(L/K\) eine Körpererweiterung.
    Definiere die Spur von \(b\in L\) durch
    \[Tr_{L/K}(b)=Tr(b\colon L\to L)\]
    und die Norm von \(b\in L\) durch 
    \[N_{L/K}(b)=\det(b\colon L\to L)\]
\end{Def}
\begin{Lemma}
    Wenn \(L/K\) separabel, dann ist für \(b\in L\) und \(\Sigma=\{\sigma\colon L\to \bar K\mid \sigma|_K=\id\}\):
    \[Tr_{L/K}(b)=\sum_{\sigma\in\Sigma}\sigma(b)\]
    \[N_{L/K}(b)=\prod_{\sigma\in\Sigma}\sigma(b)\]
\end{Lemma}
\begin{proof}
    Wenn \(L=K(b)\) und \(f\) das Minimalpolynom von \(b\) ist, dann ist
    \(f=\mu_b\mid \chi_b\) und \(n=[L:K]=\deg(f)=\deg(\chi_b)\) 
    also ist \(\chi_b=f\).
    Die \(\sigma(b)\) sind paarweise verschiedene Nullstellen von \(f\) da \(b\) separabel.
    Da Separabilitätsgrad gleich Körpergrad, ist \(|\Sigma|=n\).
    Somit teilt \(\prod_{\sigma\in \Sigma}(X-\sigma(b))\mid f\) und weil beide gleiche Gerade haben
    sind sie gleich und Aussage folgt.
    Wenn \(L\) allgemein, sei \(M=K(b)\).
    Jedes \(\sigma'\colon M\to \bar K\) hat genau \(r=[L:M]\) Fortsetzungen zu \(\sigma\colon L\to \bar K\) 
    nach ????.
    Also ist \(\prod_{\sigma\in \Sigma}(X-\sigma(b))=\prod_{\sigma'\in \Sigma'}(X-\sigma'(b))^r\) mit 
    \(\Sigma'=\{\sigma'\colon M\to \bar K\).
    Es ist \(L=M^r\) als \(M\) Vektorraum und \(b\in M\) respektiert die Summenzerlegung.
    Also ist \(\chi_{b\colon L\o L}=(\chi_{b\colon M\to M})^r\).

\end{proof}
\begin{Kor}
Seien \(M//L/K\) separable Körpererweiterungen und \(b\in M\).
Dann \[Tr_{M/K}(b)=Tr_{L/K}(Tr_{M/L}(b))\]
\[N_{M/K}(b)=N_{L/K}(N_{M/L}(b))\]
\end{Kor}
\begin{proof}
    Genau wie in ???
\end{proof}
\begin{Def}
    Sei \(L/K\) endliche separable Körpererweiterung und \(a_1,\dots,a_n\) eine \(K\)-Basis.
    Die Diskrimminante von \(a_1,\dots,a_n\) sei 
    \[d(a_1,\dots,a_n)=\det(Tr_{L/K}(a_ia_j)_{ij})\]
    das heißt die Determinante der Darstellenden Matrix bzgl der Bilinearform
    \(L\times L\to K, (a,b)\mapsto Tr_{L/K}(ab)\).
\end{Def}
\begin{Bem}[Basiswechsel]
    Wenn \(a_1,\dots,a_n\) und \(b_1,\dots,b_n\) zwei \(K\)-Basen und \(S\) die Übergangsmatrix ist,
    dann ist\[d(b_1,\dots,b_n)=det(S^TTr(a_ia_j)_{ij}S)=\det(S)^2d(a_1,\dots,a_n)\] 
\end{Bem}
\begin{Lemma}
    Wenn \(A=Tr(a_ia_j)_{ij})\) Dann ist \(A=B^tB\) für \(B=\sigma_i(a_j)_{i,j}\in M_n(\bar K)\) mit
    wobei \(\sigma_i\) die verscchiedenen \(K\)-Homomorphismen \(L\to \bar K\).
\end{Lemma}
\begin{proof}
    Nach ?? ist
    \[(B^tB)_{ik}=\sum_{j=1}^n\sigma_j(a_i)\sigma_j(a_k)=\sum_{j=1}^n\sigma_j(a_ia_k)=Tr_{L/K}(a_ia_k)=A_{ik}\]
\end{proof}
\begin{Kor}
    \(d(a_1,\dots,a_n)=det(A)=\det(B)^2\)
\end{Kor}
\begin{Satz}
\(d(a_1,\dots,a_n)\neq 0\) also ist \((a,b)\to Tr_{L/K}(ab)\) nicht ausgeartet.
\end{Satz}
\begin{proof}
    Das folgt eigentlich schon direkt aus ???
    Alternativ:
    Nach \nameref{Satz:PrimElt} ist \(L=K(b)\).
    Das heißt eine \(K\) basis von \(L\) ist \(1,b,b^2,\dots,b^{n-1}\) mit \(n=[L:K]\).
    Sei \(B=\begin{pmatrix}
1 & \dots & 1 \\
\sigma_1(b) &  &  \\
\vdots &  & \vdots \\
\sigma_1(b)^{n-1} &  & \sigma_n(b)^{n-1} 
\end{pmatrix} \) 
Nach ??? ist \(\det(B)=\prod_{i<j}(\sigma_i(b)-\sigma_j(b))\neq 0\).
\end{proof}
\begin{Bem}
    Angenommen \(A\subset K\) Teilring und \(a_1,\dots,a_n\in L\) ganz über \(A\).
    Dann ist \(\sigma_i(a_j)\) ganz über \(A\) und damit \(\det(B)\) ganz über \(A\).
    Also ist Diskrimminante \(d(a_1,\dots,a_n\) ganz über \(A\).
\end{Bem}
\begin{Def}
    Sei \(K/\QQ\) endlich und \(a_1,\dots,a_n\in\OO_K\) Ganzheitsbasis.
    Diskrimminante von \(K\) ist 
    \(d_K=d(a_1,\dots,a_n)=\det(Tr_{K/\QQ}(a_ia_j))\).
    Eine Basiswechselmatrix \(S\) zu einer anderen Ganzheitsbasis ist über \(\ZZ\) 
    also ist hat Determinante \(\pm 1\) also ist nach ??? \(d_K\) wohldefiniert.
\end{Def}
\begin{Bsp}
    Sei \(K=\QQ(\sqrt{d})\) und \(d\) quadratfrei.
    Wenn \(d\not\equiv 1 \mod 4\), dann ist \(1,\sqrt{d}\) eine Ganzheitsbasis nach ???
    Also ist \(B=(\sigma_i(a_j))_{ij}=\begin{pmatrix}
        1 & 1 \\
        \sqrt{d} & -\sqrt{d}
    \end{pmatrix}\) 
    und \(d_K=\det(B)^2=4d\).
    Wenn \(d\equiv 1 \mod 4\), dann ist \(1,\frac{\sqrt{d}+1}{2}\) eine Ganzheitsbasis nach ???
    Also ist \(B=(\sigma_i(a_j))_{ij}=\begin{pmatrix}
        1 & 1 \\
        \frac{\sqrt{d}+1}{2} & \frac{-\sqrt{d}+1}{2}
    \end{pmatrix}\) 
    und \(d_K=\det(B)^2=d\).
    
\end{Bsp}
\begin{Lemma}
    sei \((0)\neq I\subseteq \OO_K\) ein Ideal.
    Dann ist \(I=\cong \ZZ^n\) als \(\ZZ\)-Modul und \(n=[L:K]\).
\end{Lemma}
\begin{proof}
    Wähle \(a\in I\) mit \(a\neq 0\).
    Dann ist \(a\OO_K\subseteq I\subseteq \OO_K\).
    Da \(a\OO_K\cong \ZZ^n\) ist \(I\cong \ZZ^n\).
\end{proof}
\begin{Def}
    Definiere \(d(I)=d(b_1,\dots,b_n\) für eine \(\ZZ\)-Basis \(b_1,\dots,b_n\) von \(I\).
    Wenn \(a_1,\dots,a_n\) Ganzheitsbasis und \(S\) Übergangsmatrix,
    dann ist \(d(I)=\det(S)^2d_K\) mit \(\det(S)\neq 0\) über \(\QQ\) invertierbar.
\end{Def}
\begin{Lemma}
    \[\mid \det(S)\mid =\mid \OO_K/I\mid\] und somit 
    \(d(I)=\mid \O_K/I \mid^2d_K\).

\end{Lemma}
\begin{Bsp}
    Sei \(K=\QQ(\zeta_p)\) für eine primitive \(p\)-te Einheitswurzel \(\zeta_p\).
    Es ist \(\Sigma=\{\sigma\colon K\to \bar\QQ\}=\{\sigma_i\mid \sigma_i(\zeta_p)=\zeta_p^i, i=1,\dots,p-1\}\).
    Also ist 
    \[Tr_{K/\QQ}(\zeta_p)=\sum_{i=1}^{n-1}b^i=-1+\sum_{i=0}^{n-1}b^i=-1+\dfrac{\zeta_p^p-1}{\zeta_p-1}=-1\] 
    und 
    Es ist \(K=\QQ(1-\zeta_p)\) daher hat das Minimalpolynom von \(1-\zeta_p\) Grad \(p-1\).
    Dann ist \(f(X)=\frac{(1-X)^p-1}{X}=-\dfrac{(1-X)^p-1}{(1-X)-1}=-(1+(1-X)^1+\dots+(1-X)^{p-1})\in \QQ[X]\) das Minimalpolynom denn \(f(1-\zeta_p)=0\) und 
    \(f\) hat Grad \(p-1\) und ist normiert.
    Somit ist \(f\) auch das Charakteristische Polynom \(\chi_{\zeta_p}\).
    Es ist \(f(0)=-p\). Somit ist \(N_{L/K}(1-\zeta_p)=p\).
\end{Bsp}
\begin{Bsp}
    Sei \(K=\QQ(\sqrt[3]{2})\) mit Ganzheitsbasis \(1,\sqrt[3]{2},\sqrt[3]{4}\).
    Es ist \(\Sigma=\{\sigma_k\mid b\mapsto b\cdot e^{\frac{2\pi i k}{3}}, k=0,1,2\}\)
    Dann ist 
    \[B=(\sigma_i(b^j))_{ij}=\begin{pmatrix}
        1 & 1 & 1\\
        \sqrt[3]{2} & \sqrt[3]{2}e^{\frac{2\pi i 1}{3}} & \sqrt[3]{2}e^{\frac{2\pi i 2}{3}}\\
        \sqrt[3]{4} & \sqrt[3]{4}e^{\frac{2\pi i 1}{3}} & \sqrt[3]{4}e^{\frac{2\pi i 2}{3}}
    \end{pmatrix}\]
    und \(d_K=\det(B)^2=(-3i2\sqrt{3})^2=-3^3\cdot 2^2\).
    Alternativ:
    Wenn \(a_1=1, a_2=b, a_3=b^2\) Ganzheitsbasis ist,
    Dann ist \(\{a_ia_j\mid i,j=1,2,3\}=\{1,b,b^2,2\}\).
    Es ist \(b\colon K\to K\) ist dargestellt durch 
    \[\begin{pmatrix}
       0 & 0& 2\\
       1 & 0 & 0\\
       0 & 1& 0\\
    \end{pmatrix}\] Also ist \(Tr(b)=0\).
    \(b^2\colon K\to K\) ist dargestellt durch \[\begin{pmatrix}
       0 & 2& 0\\
       0 & 0 & 2\\
       1 & 0& 0\\
    \end{pmatrix}\] hat also auch \(Tr(b^2)=0\).
    Es ist \(Tr(1)=3\) und \(Tr(2)=2\dot 3\).
    Also ist \[A=
        \begin{pmatrix}
       3 & 0& 0\\
       0 & 0 & 6\\
       0 & 6 & 0\\
    \end{pmatrix}\] 
    und somit \(d_K=\det(A)=-3^3\cdot 2^2\).
\end{Bsp}
% Das hier sollte eher zu Ganzheit gehören
\begin{Bsp}
    Seien \[a=\sqrt[3]{1+\sqrt[4]{2+\sqrt{7}}},\, b=\sqrt[3]{1+\sqrt[4]{2+\frac 1 2\sqrt{7}}},\,
    c=\dfrac{3+2\sqrt{6}}{1-\sqrt{6}}\] 
    Es ist \(f(X)=\left((X^3-1)^4-2\right)^2-7\in \ZZ[X]\) und \(f(a)=0\) 
    also ist \(a\) ganz über \(\ZZ\).
    Angenommen \(b\) wäre ganz über \(\ZZ\)).
    Dann auch \(\left((b^3-1)^4-2\right)^2=\frac 7 4\) was nicht richtig ist da \(\ZZ\) ganzabgeschlossen in \(\QQ\).
    Es ist \(c=-(\sqrt{6}+3)\) und \(\sqrt{6}\) ist ganz da \(\sqrt{6}^2-6=0\) und \(3\) ist ganz.
    Also ist auch \(c\) ganz.
\end{Bsp}
% Das hier sollte eher zu Topologie oder Gruppen gehören
\begin{Lemma}
    Sei \(G\) eine topologische Gruppe. Dann gilt:
    \begin{enumerate}
        \item Für eine normale Untergruppe \(H\subseteq G\) ist \(G/H\) eine topologische Gruppe.
        \item Wenn \(K\subseteq H\subseteq G\) normale Untergruppen sind, dann ist die 
        Teilraumtopologie \(H/K\subseteq G/K\) die gleiche wie die Quotiententopologie.
        \item Sei \(G\) Hausdorff. Dann \[G/H\text{ Hausdorff } \iff H\subseteq G \text{abgeschlossen}\] 
        \item Sei \(H\subseteq \RR^n\) abgeschlossen. Es gilt 
        \[\RR^n/H \text{ kompakt } \iff \exits B\subseteq \RR^n \text{ beschränkte Menge sd.} B+H=\RR^n\] 
    \end{enumerate}
\end{Lemma}
\begin{proof}
    \begin{enumerate}
        \item Klar.
        \item Die Abbildung \(H\to H/K\) wobei \(H/K\) Unterraumtopologie hat ist stetig.
        Wenn also \(U\) offen in Unterraumtopologie dann auch in Quotiententopologie.
        Sei \(U\) offen in Quotiententopologie, \(\pi_H\colon H\to H/K\), \(\pi_G\colon G\to G/K\) Projektion.
        Das heißt es gibt ein \(V\subseteq G\) offen sodass \(V\cap H=\pi_H^{-1}(U)\).
        Es ist \[\pi_G^{-1}(V/K)=V\cdot K=\bigcup_{k\in K}V\cdot k\] offen, also
        \(V/K\subseteq G/K\) offen.
        Es ist \(V/K \cap H/K=(V\cap H)/K=U\). 
        Also ist \(U\) offen in Unterraumtopologie.
        \item Sei \(G/H\) Hausdorff, \(\pi\colon G\to G/H\). Es ist \(eH\) abgeschlossen in \(G/H\) da Einpunktmengen immer
        abgeschlossen sind in Hausdorff. Also ist \(H=\pi^{-1}(eH)=H\) abgeschlossen.
        Sei andererseits \(H\) abgeschlossen.
        Betrachte \(G\times G\to G/H\times G/H\). Das ist offen da \(\pi\) offen ist(Topologische Gruppen).
        Sei \(W=\{(x,y)\in G\times G\mid x^{-1}y\in G\setminus H\}\). Das ist offen da \(G\) topologische Gruppe.
        Es ist \((\pi\times\pi)(W)=G/H\times G/H \setminus\{\text{Diagonale}\}\).
        Also ist die Diagonale abgeschlossen in \(G/H\times G/H\) also \(G/H\) Hausdorff.
        \item Sei \(H\subseteq \RR^n\) abgeschlossen und \(\pi\colon \RR^n\to \RR^n/H\) die Projektion.
        Angenommen \(R/H\) kompakt das heißt \(\RR^n/\setminus H=\bigcup_{i=1}^m\pi(B_k(0))\) für ein \(m\).
        Dann ist \(\RR^n=\bigcup_{k=1}^mB_k(0)+H\).
        wenn andererseits \(\RR^n=B+H\) wobei \(B\) beschränkt, dann ist ohne Einschränkung \(B\) abgeschlossen also kompakt.
        Da \(\pi(B)=R^n/H\) ist ist auch \(R^n/H\) kompakt.
    \end{enumerate}
\end{proof}
