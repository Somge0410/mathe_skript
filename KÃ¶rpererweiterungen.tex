\chapter{Körpererweiterungen}
\section{Grundlagen}
\begin{Def}
    Sei \(K\) ein Körper. Der Primkörper von \(K\) ist der Schnitt aller Teilkörper von \(K\).
    Es gibt \(\ZZ/(p)\subseteq K\) für \(p=\chara K\).
Wenn \(p=0\), dann ist \(\QQ\subseteq K\) und \(\QQ\) ist der Primkörper. Wenn \(p\neq 0\), dann ist \(\mathbb F_p\subset K\) der Primkörper.
\end{Def}
\begin{Def}
    Seien \(L,K\) Körper sodass \(K\subseteq L\) Teilring ist. Dann heißt \(K\) Teilkörper von \(L\) und \(L\) eine Erweiterung von \(K\). Wir setzen \([L:K]=\dim_K(L)\).
\end{Def}
\begin{Satz}
    Sei \(L/K\) eine Körpererweiterung und \(V\) ein \(L\)-Vektorraum. Dann ist \[\dim_K(V)=\dim_L(V)\cdot [L:K].\]
\end{Satz}
\begin{proof}
    Wenn \(\dim_L(V)=\infty\) dann ist alles klar. Sonst wähle Isomorphismus \(V=L^n\). Das ist Isomorphismus von \(K\)-Vektorräumen. Also gilt die Aussage.
\end{proof}
\begin{Kor}
    Wenn \(M/L/K\) Körpererweiterungen sind, dann ist 
    \[[M:K]=[M:L]\cdot[L:K]\]
\end{Kor}
\section{Algebraische und transzendente Erweiterungen}
\begin{Def}Sei \(L/K\) eine Körpererweiterung und \(a\in L\) und \(\phi\colon K[X]\to L\) der Ringhomomorphismus gegeben durch \(\phi(f)=f(a)\).
    \begin{enumerate}
        \item []
        \item  Das Element \(a\) heißt algebraisch, falls \(\ker(\phi)\neq 0\). Andereseits nennen wir \(a\) transzendental über \(K\).
        \item \(L/K\) heißt algebraisch, wenn jedes \(a\in L\) algebraisch über \(K\) ist.
        \item  \(K[X]\) ist Euklidisch, somit Hauptidealring. Also ist \(\ker(\phi)=(f)\) für ein normiertes Polynom \(f\). Das Polynom \(f\) heißt das Minimalpolynom von \(a\).
    \end{enumerate}
\end{Def}
\begin{Def}
    Sei \(L/K\) eine Körpererweiterung und \(a_1,\dots,a_r\in L\).
    \begin{enumerate}
        \item Die Algebra erzeugt von \(a_1,\dots,a_r\) ist \begin{align*} 
        K[a_1,\dots,a_r]&=\bigcap\limits_{R\in M}R\\
        &=\Image(\phi\colon K[X_1,\dots,X_R]\to L,\ X_i\mapsto a_i)
        \end{align*} wobei \(M\) die Menge aller Unterringe von \(L\) ist, die \(K\) und die Elemente \(a_1,\dots,a_r\) enthalten.
        \item Der Körper erzeugt von \(a_1,\dots,a_r\) ist \begin{align*} 
        K(a_1,\dots,a_r)&=\bigcap\limits_{K'\in M'}K'\\
        &=\Quot(K[a_1,\dots,a_r])
        \end{align*} wobei \(M'\) die Menge aller Teilkörper von \(L\) ist, die \(K\) und die Elemente \(a_1,\dots,a_r\) enthalten.
    \end{enumerate}
\end{Def}
\begin{Satz}
    Sei \(L/K\) Körpererweiterung und \(a\in L\).
    \begin{enumerate}
        \item \(a \text{ ist algebraisch über }K \iff K[a]=K(a)\iff \dim_K(K[a])<\infty\)
        \item Wenn \(a\) algebraisch über \(K\) ist, dann ist \(K[a]=K(a)\cong K[X]/(f)\) wobei \(f\) (irreduzibel) das Minimalpolynom von \(a\) ist und \[\deg(f)=[K(a):K].\]
    \end{enumerate}
\end{Satz}
\begin{proof}
    Sei \(a\) algebraisch. Dann ist das Minimalpoylon \(f\) irreduzibel sodass 
        \[(f)=Ker(\phi\colon K[X]\to L, x\mapsto a).\] Dann ist \(K[a]\cong K[X]/(f)\) ein Körper, also ist auch \(K[a]=K(a)\). 
        Wenn \(K[a]\cong K[X]/(f)\) ein Körper ist, dann ist \(f\) irreduzibel also \(f\neq 0\) und \(f(a)=0\). Also ist \(a\) algebraisch.
        In dem Fall ist \(\infty>deg(f)=[K[a]:K]\).
        Wenn \(a\) nicht algebraisch ist, dann ist \(K[a]\cong K[X]\) und \(\dim_K(K[X])=\infty.\)
\end{proof}
\begin{Bsp}
    \(\CC=\RR[i]\cong \RR[X]/(X^2+1)\)
\end{Bsp}
\begin{Satz}\label{Satz:EndlAlg}
    Sei \(L/K\) eine Körpererweiterung.
    \begin{align*}
        L/K\text{ ist endlich } &\iff L/K\text{ ist algebraisch und } L=K(a_1,\dots,a_n)\\
        &\iff L=K(a_1,\dots,a_n) \text{ für \(K\)-algebraische \(a_i\in L\)}
    \end{align*}
    In dem Fall gilt \(K(a_1,\dots,a_n)=K[a_1,\dots,a_n]\)
\end{Satz}
\begin{proof}
    Sei \(L/K\) endlich. Dann ist \(L\) als \(K\)-Vektorraum erzeugt und insbesondere als Körpererweiterung. Für \(a\in L\) gilt \(K[a]\subseteq L\) und \[\dim_K(K[a])\leq\dim_KL<\infty\] also ist \(a\) algebraisch.
    Sei \(L=K(a_1,\dots,a_n)\) sodass \(a_i\) algebraisch über \(K\) ist.
    Wenn \(n=0\) ist, dann ist \(L=K\) also ist \(L/K\) endlich.
    Sei \(K[a_1]=K'\). Das ist ein Körper und somit endlich über \(K\) und \(L=K'[a_1,\dots,a_n]\). Nach Induktion folgt \(K'[a_2,\dots,a_n]=K'(a_2,\dots,a_n)\).
\end{proof}
\begin{Kor} Sei \(L/K\) eine Körpererweiterung und \(a_1,\dots,a_n\in L\). Es gilt
    \[a_1,\dots a_n \text{ algebraisch über }K \iff K(a_1,\dots,a_n)=K[a_1,\dots,a_n]\]
\end{Kor}
\begin{proof}
    Die eine Richtung folgt aus \Cref{Satz:EndlAlg}, die andere aus dem \Cref{Satz:WeakNst}.
\end{proof}
\begin{Kor}
    Seien \(M/L/K\) Körpererweiterungen. Es gilt \[M/L \text{ und }L/K \text{ algebraisch }\iff M/K\text{ algebraisch}\]
\end{Kor}
\begin{proof}
    Sei \(a\in M\) algebraisch über \(L\) und \(f\in L[X]\) ein Polynom mit \(f(a)=0\). Sei \[f=\sum\limits_{i=0}^nb_iX^i\] für \(b_i\in L\) und \(b_n=1\).
    Dann ist \(L'=K[b_0,\dots,b_{n-1}]\) ein Körper sodass \(L'/K\) endlich ist.
    Da \(a\) algebraisch ist über \(L'\), ist \(L'(a)\) endlich über \(L'\). Also ist \(L'(a)\) endlich über \(K\) und somit algebraisch.
\end{proof}
\begin{Satz}
    Sei \(K\) Körper und \(f\in K[X]\) irreduzibel. Dann gibt es eine Körpererweiterung \(L/K\) mit \([L:K]=\deg(f)\) und \(a\in L\) mit \(f(a)=0\).
    
\end{Satz}
\begin{proof}
    Klar, \(L=K[X]/(f)\).
\end{proof}
\begin{Kor}\label{Kor:ExZerfall}
    Zu endlich vielen \(f_1,\dots,f_r\in K[X]\) mit \(deg(f_i)\geq 1\) gibt es eine endliche Erweiterung \(L/K\), sodass \(f_i\in L[X]\) in Linearfaktoren zerfällt.
\end{Kor}
\begin{proof}
    Angenommen es gibt \(L/K\) sodass \(f_1\) in \(L[X]\) zerfällt und \(L_r/L_1\) sodass \(f_2,\dots,f_r\) in \(L_r[X]\) zerfällt. Dann zerfällt \(f_1,\dots,f_r\) in \(L_r/K\).
    Also sei nach Induktion \(r=1\).
    Sei \(f=g_1,\dots,g_s\) mit \(g_i\) irreduzibel.
    Nach letztem Satz gibt es \(L'/K\) endlich und \(a\in L'\) mit \(g_1(a)=0\).
    In \(L'[X]\) gilt \(f=(X-a)f_1\).
    Induktion über \(\deg(f)\) gibt die Aussage.
\end{proof}
\begin{Bsp}
    \(f=X^3-2\in\QQ[X]\) ist irreduzibel nach Eisenstein.
    \[L_1=\QQ[X]/(f)\cong \QQ(\sqrt[3]{2}).\] In \(L_1[X]\) ist \(f=(X-a)\cdot g\) wobei \(a=\sqrt[3]{2}\) und \(g\) irreduzibel ist (Irreduzibel in \(\RR\) also in \(L_1\)).
    Wenn \(L=L_1[X]/(g)\) dann hat \(L/\QQ\) hat Grad \(6\) und \(x^3-2\) zerfällt in Linearfaktoren in \(L\).
\end{Bsp}
\begin{Lemma}[Algebraisch Abgeschlossen]

 Für einen Körper \(K\) ist äquivalent:

\begin{enumerate}

 \item Jedes nicht-konstante Polynom \(f\in K[X]\) hat eine Nullstelle

\item Jedes irreduzible Polynom \(f\in K[X]\) hat Grad 1

\item Für jede algebraische Erweiterung \(L/K\) gilt \(L=K\)

\end{enumerate}

In dem Fall heißt \(K\) algebraisch abgeschlossen.

\end{Lemma}

\begin{proof}

\(1\implies 2\) ist klar.

Gelte \(2\). Dann sei \(f\) nicht-konstant. Also gibt es ein irreduzibles Polynom \(P\) mit \(P|f\). Da \(P=aX+b\) für ein \(a\neq 0\), ist \(P(\frac{-b}{a})=0\) also hat \(f\) eine Nullstelle.

Gelte 2 und sei \(a\in L\) algebraisch mit Minimalpolynom \(f\in K[X]\). \(f\) ist irreduzibel, also ist \(f=X-a\) und somit \(a\in K\).

Gelte 3. und sei \(f\) irreduzible. Dann ist \(L=K[X]/(f)\cong K\). Also ist \(\deg(f)=1\) und \(f\) linear.

\end{proof}

\begin{Def}
    Sei \(K\) ein Körper. Ein algebraischer Abschluss von \(K\) ist eine algebraische Körpererweiterung \(L/K\) sodass \(L\) algebraisch abgeschlossen ist. Notation \(L=\bar K\)
\end{Def}

\section{Körperhomomorphismen}
\begin{Def}
    Ein Körperhomomorphismus ist ein Ringhomomorphismus zwischen Körpern. Seien \(L/K\) und \(M/K\) zwei Körpererweiterungen. Ein \(K\)-Homomorphismus ist ein Homomorphismus von \(K\)-Algebren.
    \(\Aut_K(L)=\Aut(L/K)\) sei die Menge der invertierbaren \(K\)-Homomorphismen \(f\colon L\to L\)
\end{Def}
\begin{Lemma}
 Seien \(L=K(a)/K\) eine Körpererweiterungen und sei \(f\) das Minimalpolynom von \(a\). Sei \(M\) ein Körper mit einem Homomorphismus \(\sigma\colon K\to M\). Sei \[\Sigma=\Set{K-\text{Homomorphismen } \sigma'\colon L\to M}.\] Dann ist die Abbildung 
    \[\Sigma\to\set{b\in M}{ f(b)=0},\ \sigma'\mapsto \sigma'(a)=b\] bijektiv
\end{Lemma}
\begin{proof}
    Wir haben die Abbildung \(\phi\colon K[X]\to M, X\mapsto b\). Es gilt \(\phi\) lässt sich eindeutig fortsetzen zu \(\sigma'\colon K[X]/(f)\to M\) genau dann, wenn \(f\in \Ker(\phi)\) ist, das heißt wenn \(f(b)=0\). Dann ist \(\sigma'(a)=\sigma'(\bar X)=b\).
\end{proof}
\begin{Bsp}
    \(L=M=\CC\) und \(K=\RR\). Dann ist
    \[\{\RR-\Hom \sigma'\colon\CC\to\CC\} \stackrel{\sim}{\to}\set{b\in\CC}{b^2+1=0},\ \sigma'\mapsto \sigma'(i)\]
    also \(\Aut(\CC/\RR)=\{\id,\sigma'\}\) wobei \(\sigma'\) komplexe Konjugation ist.
\end{Bsp}
\begin{Satz}\label{Satz:AlgAbMor}
    Sei \(L/K\) eine algebraische Erweiterung und \(M\) ein algebraisch abgeschlossener Körper. Sei weiter \(\sigma\colon K\to M\) ein Körperhomomorphismus. Dann existiert eine Fortsetzung von \(\sigma\) zu einem Körperhomomorphismus \(\sigma'\colon L\to M\) sodass
     % https://tikzcd.yichuanshen.de/#N4Igdg9gJgpgziAXAbVABwnAlgFyxMJZABgBpiBdUkANwEMAbAVxiRABkQBfU9TXfIRQAmclVqMWbALLdeIDNjwEiARlKrx9Zq0QgA0t3EwoAc3hFQAMwBOEALZJRIHBCTqJOtgB1v2U-Z0INQMdABGMAwACvzKQiA2WKYAFjhy1naOiM6uSGQudFgMbMkQEADW6SC2DnnUuYge2lJ6vv6BAORGXEA
\begin{tikzfigure}
L \arrow[rr, "\sigma'"] &                                          & M \\
                        & K \arrow[ru, "\sigma"'] \arrow[lu, hook] &  
\end{tikzfigure} kommutiert
\end{Satz}
\begin{proof}
    Fall 1: Sei \(L=K(a)=K[a]=K[X]/(f)\). Die Menge der \(\sigma'\) ist bijektiv zur Menge der Nullstellen von \(f\) in \(M\). Also existiert \(\sigma'\).
    
    Fall 2: Sei \(L/K\) allgemein. Sei \(X\) die Menge der Paare \((L',\sigma')\) wobei \(L'\) ein Körper ist mit \(K\subseteq L'\subseteq L\) und \(\sigma'\) eine Fortsetzung von \(\sigma\).
    Definiere partielle Ordnung \((L',\sigma')\leq (L'',\sigma'')\) durch \(L'\subseteq L''\) und \(\sigma''|_{L'}=\sigma'\).
    Sei \(X'\subseteq X\) total geordnet. Dann ist \[\tilde L=\bigcup\limits_{(L',\sigma')\in X'}L'\] ein Körper und zusammen mit \(\tilde\sigma\colon\tilde L\to M\) definiert durch \(\tilde\sigma(b)=\sigma'(b)\) für \(b\in L'\) eine obere Schranke von \(X'\).
    Nach \nameref{Satz:Zorn} hat \(X\) ein maximales Element \((L',\sigma')\).
    Angenommen \(L'\neq L\). Dann wähle \(a\in L\setminus L'\) und setze \(L''=L'(a)\). Nach Fall 1 existiert eine Fortsetzung \(\sigma''\colon L''\to M\) was ein Widerspruch ist. Also ist \(L'=L\). 
\end{proof}
\begin{Kor}
    Seien \(L/K\) und \(M/K\) zwei algebraische Abschlüsse von \(K\). Dann gibt es einen \(K\)-Isomorphismus \(L\to M\)
\end{Kor}
\begin{proof}
    Nach \Cref{Satz:AlgAbMor} gibt es \(K\)-Homomorphismus \(\sigma\colon L\to M\). Dadurch wird \(M\) eine algebraische Erweiterung von \(L\). Also ist \(M/L\) triviale Erweiterung, dh. \(\sigma\) ist bijektiv.
\end{proof}
\section{Zerfällungskörper und Algebraischer Abschluss}
\begin{Satz}[Existenz algebraischer Abschluss]
    Jeder Körper hat einen algebraischen Abschluss
\end{Satz}
\begin{proof}
    Sei \(I\) die Menge aller irreduziblen Polynome \(f\in K[X]\) und sei \(R=K\polring{X_f}{f\in I}\). Sei \(J\subseteq R\) das ideal, das von allen Elementen der Form \(f(X_f)\) mit \(f\in I\) erzeugt wird.
    
    Es gilt die Behauptung \(J\subsetneq R\). Denn angenommen \(J=R\), dann ist \(1\in J\) also gibt es Darstellung 
    \begin{align} 1=\sum_{j=1}^rg_jf_j(X_{f_j})\label{Eq:1}
    \end{align} wobei \(g_j\in R\) und \(f_1,\dots,f_r\in I\). In \(g_1,\dots,g_r\) kommen nur endlich viele \(X_{f_i}\) vor, somit gibt es eine endliche Menge \(I'\subseteq I\), sodass die Gleichung \ref{Eq:1} in \(R'=K[X_f| f\in I']\) stattfindet.
    Nach \Cref{Kor:ExZerfall} gibt es eine endliche Erweiterung \(M/K\), sodass jedes \(f\in I'\) in \(M\) eine Nullstelle \(a_f\) hat. Betrachte die Abbildung \[\phi\colon R'\to M,\ \phi(X_f)=a_f.\]
    Dann ist \(\phi(f(X_f))=f(a_f)=0\) also \(f(X_f)\in \Ker(\phi)\). Die Gleichung \ref{Eq:1} würde zeigen, dass \(1\in \Ker(\phi)\), was ein Widerspruch ist zu \(1\neq 0\) in \(M\).
    Also ist gilt die Behauptung \(J\subsetneq R\).
    
    Also ist \(\bar R=R/J\neq 0\) und nach \Cref{Satz:ExMaxId} gibt es maximales Ideal \(\frakm\subseteq \bar R\). Für den Quotienten \(L=\bar R/m\), ist \(L/K\) eine Körpererweiterung und \(L\) ist erzeugt von \(\bar X_f\) für \(f\in I\). Es ist \(f(X_f)=0\) in \(\bar R\) und somit auch in \(L\). Also ist \(\bar X_f\in L\) algebraisch über \(K\) und damit ist auch \(L/K\) algebraisch und jedes \(f\in I\) hat in \(L\) eine Nullstelle, nämlich \(\bar X_f\).
    Sei \(L_1=L\). \(L_1/K\) ist algebraisch, sodass jedes irreduzible Polynom \(f\in K[X]\) in \(L_1\) eine Nullstelle hat.
    Konstruiere Analog \(K\subseteq L_1\subseteq L_2\subseteq L_3\dots\) sodass jedes irreduzible Polynom in \(L_i\) in \(L_{i+1}\) eine Nullstelle hat. Sei \(\tilde L= \bigcup\limits_{i\geq 1}L_i\). Das ist  ein Körper und \(\tilde L/K\) ist algebraisch.
    Der Körper \(\tilde L\) ist algebraisch abgeschlossen, denn wenn \(f\in \tilde L[X]\) irreduzibel dann gibt es \(i\in\NN\) sodass \(f\in L_i[X]\) und dann hat \(f\) in \(L_{i+1}\) eine Nullstelle. Also hat \(f\) in \(\tilde L\) eine Nullstelle.
\end{proof}
\begin{Def}
    Sei \(K\) ein Körper und \(\mathcal F\subseteq K[X]\) eine Menge von nicht-konstanten Polynomen. Ein Zerfällungskörper ist eine Körpererweiterung \(L/K\) sodass jedes \(f\in\mathcal F\) in \(L[X]\) in Linearfaktoren zerfällt und \(L=K(a\in\bar K\mid f(a)=0 \text{ für ein } f\in\mathcal F)\)
\end{Def}
\begin{Lemma}
    Für eine Menge \(\mathcal F\subseteq K[X]\) nicht-konstanter Polynome existiert ein Zer\-fäl\-lungs\-kör\-per und ein Zer\-fäl\-lungs\-kör\-per ist eindeutig bist auf \(K\)-Isomorphismus.
\end{Lemma}
\begin{proof}
    Sei \(L=K(a\in\bar K\mid f(a)=0 \text{ für ein } f\in\mathcal F)\subseteq \bar K\). Dann ist \(L\) ein Zerfällungskörper.
    Sei \(M\) ein weiterer Zerfällungskörper. Dann gibt es nach \Cref{Satz:AlgAbMor} einen \(K\)-Homomorphismus \(\sigma\colon M\to\bar K\). Seien \(f\in\mathcal F\) und \(a_1,\dots, a_n\) die Nullstellen in \(\bar K\) von \(f\) und \(b_1,\dots,b_n\) die Bilder der Nullstellen von \(f\) in \(M\) unter \(\sigma\).
    Da \(\prod_{i=1}^n(X-a_i)=f=\prod_{i=1}^n(X-b_i)\) in \(\bar K[X]\) ist ohne Einschränkung \(a_i=b_i\) für alle \(i\). Somit ist \(\sigma(M)=L\) und \(\sigma\) ist ein Isomorphismus \(M\to L\).
\end{proof}
\begin{Bem}
    Das zeigt: Alle \(K\)-Homomorphismen \(\sigma\colon M\to \bar K\) haben Bild \(L\).
\end{Bem}
\begin{Bsp}
    Sei \(K=\QQ\) und \(f=X^3-2\). In \(\bar\QQ[X]\) gilt \(f=(X-a)(X-\zeta a)(X-\zeta^2a)\) für \(a=\sqrt[3]{2}\) und \(\zeta=e^{2\pi i/3}\).
    Also ist der Zerfällungskörper \(L=\QQ(a,\zeta a,\zeta^2 a)=\QQ(a,\zeta)\)
    Es ist \(\QQ\subsetneq \QQ(a)\subsetneq \QQ(a,\zeta a)\) und \([\QQ(a):\QQ]=3\) da \(f\) das Minimalpolynom ist. Es ist \(g=(X-\zeta a)(X-\zeta^2 a)\) das Minimalpolynom von \(\zeta a\) über \(\QQ(a)\), also ist \([\QQ(a,\zeta a):\QQ]=6\)
\end{Bsp}
\section{Normale und Separable Erweiterungen}
\subsection{Normale Erweiterungen}
\begin{Def}
    Eine algebraische Körpererweiterung \(L/K\) ist normal, wenn jedes irreduzible Polynom in \(K[X]\), das in \(L\) eine Nullstelle hat in \(L[X]\) in Linearfaktoren zerfällt.
\end{Def}
\begin{Lemma}
    Sei \(L/K\) algebraisch und \(\varphi\colon L\to L\) ein \(K\)-Homomorphismus. Dann ist \(\phi\) bijektiv.
\end{Lemma}
\begin{proof}
    Immer ist \(\phi\) injektiv.
    Sei \(a\in L\). Dann gibt es \(f\in K[X]\) sodass \(K(a)=K[X]/(f)\). Seien \(a=a_1,\dots,a_n\) die Nullstellen von \(f\) in \(L\).
    \(\varphi|_{K(a_i)}\colon K(a_i)\to L\) gibt \(n\) verschiedene \(K\)-Homomorphismen \(K[X]/(f)\to L\) da \(\varphi(a_i)\neq \varphi(a_j)\) für \(i\neq j\). Da diese in Bijektion zu der Menge der \(\{a_1,\dots,a_n\}\) stehen gilt für ein \(i\): \(\varphi|_{a_i}(a_i)=a_1=a\). Also ist \(\varphi\) bijektiv.
\end{proof}
\begin{Satz}
    Sei \(L/K\) algebraisch. Dann ist äquivalent:
    \begin{enumerate}
        \item \(L/K\) ist normal
        \item \(L/K\) ist Zerfällungskörper einer Menge \(\mathcal F\subseteq K[X]\).
        \item Für jede Körpererweiterung \(M/L\) und jeden \(K\)-Homomorphismus \(\varphi\colon L\to M\) gilt \(\phi(L)=L\)
        \item Für Jeden \(K\)-Homomorphismus \(\varphi\colon L\to\bar L\) gilt \(\phi(L)=L\).
    \end{enumerate}
\end{Satz}
\begin{proof}
    Gelte 1. Dann sei \[\mathcal F=\set{f\in K[X]}{ f\text{ ist irreduzibel und hat eine Nullstelle in \(L\)}}.\]
    Jedes \(f\in \mathcal F\) zerfällt in \(L[X]\) in Linearfaktoren.
    Sei \(a\in L\) mit Minimalpolynom \(f\in K[X]\). Dann ist \(f\in\mathcal F\) also ist \(L\) von allen Nullstellen erzeugt. Also ist \(L\) Zerfällungskörper.
    Gelte 2. Sei \(L/K\) Zerfällungskörper von \(\mathcal F\) und \(\varphi\colon L\to M\) ein \(K\)-Homomorphismus. Für \(f\in \mathcal F\) und \(a\in L\) mit \(f(a)=0\) gilt \(f(\varphi(a))=0\) also ist \(\phi(a)\) Nullstelle von \(f\) in \(M\).
    Dann ist \(L=K(a\in M\mid f(a)=0 \text{ für ein } f\in\mathcal F)\) und somit \(\phi(L)\subseteq L\). Da \(L/K\) algebraisch ist, ist \(\varphi\) bijektiv und \(\varphi(L)=L.\)\\
    3. \(\implies 4\) ist klar. Gelte \(4.\) Sei \(f\in K[X]\) irreduzibel mit \(f(a)=0\) für \(a\in L\).
    Die Menge der \(K\)-Homomorphismen \(\sigma\colon K(a)\to \bar L\) ist bijektiv zur Menge \(\set{b\in\bar L}{ f(b)=0}\). Zu \(b\in \bar L\) wähle also \(\sigma\colon K(a)\to \bar L\).
    Nach \Cref{Satz:AlgAbMor} gibt es ein \(\varphi\colon L\to \bar L\), das \(\sigma\) fortsetzt. Dann ist \(\varphi(L)=L\). Also ist \(b=\sigma(a)=\varphi(a)\in L\). Also zerfällt \(f\) in \(L[X]\).
\end{proof}
\begin{Lemma}[Normalität in Türmen]
    Seien \(M/L/K\) Körpererweiterungen. Es gilt \[M/K \text{ normal }\implies M/L \text{ normal.}\]
\end{Lemma}
\begin{proof}
    Sei \(M\) ein Zerfällungskörper von \(\mathcal{F}\subseteq K[X]\). Dann ist \(M/L\) ein Zerfällungskörper von \(\mathcal F\) als Teilmenge von \(L[X].\)
\end{proof}
\begin{Def}
    Sei \(L/K\) algebraisch. Eine normale Hülle von \(L/K\) ist eine Erweiterung \(M/L\) sodass \(M/K\) normal ist und für jede andere Erweiterung \(M'/L\) mit \(M'/K\) normal gibt es einen \(L\)-Homomorphismus \(M\to M'\). Das zeigt: Eine normale Hülle ist eindeutig bis auf Isomorphismus
\end{Def}
\begin{Satz}
    Die normale Hülle einer algebraischen Erweiterung \(L/K\) existiert
\end{Satz}
\begin{proof}
    Sei \[\mathcal F=\Set{f\in K[X] \text{ irreduzibel sodass \(f\) eine Nullstelle in \(L\) hat}}.\] Sei \(M/L\) Zerfällungskörper von \(\mathcal F\subseteq L[X]\). Dann ist \(M/K\) Zerfällungskörper der selben Menge und somit \(M/K\) normal. Sei \(M'/L\) mit \(M'/L\) normal. Wähle \(\bar M'/M'\) algebraischen Abschluss. Nach \Cref{Satz:AlgAbMor} gibt es einen \(L\)-Hom \(\varphi\colon M\to\bar M'\). Jedes \(f\in\mathcal F\) zerfällt in \(M'[X]\) und da \(M\) von allen Nullstellen erzeugt gilt \(\phi(M)\subseteq M'\).
\end{proof}
\begin{Bsp}
    Eine normale Hülle von \(\QQ(\sqrt[3]{2})/\QQ\) ist \(\QQ(\sqrt[3]{2},\zeta a)\), der Zerfallskörper von \(x^2-2.\)
\end{Bsp}
\subsection{Separable Erweiterungen}
\begin{Def}
    \(f\in K[X]\) heißt separable, wenn \(f\in \bar K[X]\) keine mehrfachen Nullstellen hat.
\end{Def}
\begin{Def}
    Die formale Ableitung von \(f\in K[X]\) ist \( f=\sum_{i=0}^na_iX^i\) ist \(f'=\sum_{i=0}^ni\cdot a_iX^{i-1}\)
\end{Def}
\begin{Lemma}[Leibniz-Regel]
    \[(fg)'=f'g+fg'\]
\end{Lemma}
\begin{proof}
    Reduziere auf \(f=X^n\) und \(g=X^m\). Dann ist alles klar.
\end{proof}
\begin{Lemma}
    Sei \(f\in K[X]\).
    \begin{enumerate}
        \item \(b\in \bar K\) ist mehrfache Nullstelle von \(f\iff f(b)=0=f'(b)\iff (X-b)| \ggT(f,f')\)
        \item \(f\) ist separable \(\iff \ggT(f,f')=1\)
    \end{enumerate}
\end{Lemma}
\begin{proof}
    Der \(\ggT\) ändert sich nicht bei Übergang zu einer Erweiterung \(L/K\), also auch nicht bei \(\bar K/K\). 1. ist eine Rechnung und 2. folgt aus \(1\).
\end{proof}
\begin{Satz}
    Sei \(f\in K[X]\) irreduzibel.
    \begin{enumerate}
        \item \(f\) ist separable \(\iff f'\neq 0\) in \(K[X]\).
        \item Wenn \(K\) Charakteristik \(0\) hat, ist \(f\) separable.
    \end{enumerate}
\end{Satz}
\begin{proof}
    \begin{align*}
        f \text{ separable}\iff\ggT(f,f')=1\iff f'\neq 0
    \end{align*} da \(f'\neq f\).
    Wenn \(K\) Charakteristik 0 hat, dann ist \(f'\neq 0\) also ist \(f\) separable.
\end{proof}
\begin{Bem} Für \(f\) irreduzibel gilt:
    \(f'=0\iff \chara K=p>0\) und \(f=\sum_{i=0}^nb_iX^{p^i}\) für \(b_i\in K\)
\end{Bem}
\begin{Bem}
    Sei \(p\) Primzahl und \(K=\Quot(\FF_p[Y])\). Das Polynom \(f=X^p-Y\in K[X]\) ist irreduzibel nach Eisenstein für das Primelement \(Y\in\FF_p[X]\) und in-separable.
\end{Bem}
\begin{Def}
    Sei \(L/K\) eine algebraische Körpererweiterung. 
    \begin{enumerate}
        \item \(a\in L\) ist separable über \(K\), wenn das Minimalpolynom von \(a\) separable ist.
        \item \(L/K\) ist separable, wenn jedes \(a\in L\) separable ist.
        \item Der Separabilitätsgrad von \(L/K\) ist \([L:K]_s= \abs{\Set{K-\Hom.\ L\to\bar K}}\)
    \end{enumerate}
\end{Def}
\begin{Lemma}
    Sei \(L=K(a)\) algebraisch über \(K\).
    Dann gilt 
    \begin{enumerate}
        \item \([L:K]_s\leq [L:K]\)
        \item \([L:K]_s=[L:K]\iff a\) separable über \(K\)
    \end{enumerate}
\end{Lemma}
\begin{proof}
    Sei \(f\) Minimalpolynom von \(a\) sodass \(L\cong K[X]/(f)\).
    Es gilt 
    \[\abs{\Set{K-\Hom L\to\bar K}}=\abs{{\set{b\in\bar K}{ f(b)=0}}}\leq \deg(f)\] und "\(=\)" genau dann, wenn \(f\) separable. 
\end{proof}
\begin{Lemma}\label{Lem:SeparabilitätsgradKörpererw}
    Sei \(M/L/K\) endlich. Dann ist \([M:L]_s\cdot [L:K]_s=[M:K]_S\)
\end{Lemma}
\begin{proof}
    \(\bar K\) kann als algebraischer Abschluss von \(M\) und \(L\) aufgefasst werden.
    Dann gilt \[[M:L]_s=\abs{\set{\psi\colon M\to \bar K}{ \psi|_L=\varphi}}\] für jeden \(\phi\colon L\to\bar K\)
    erfüllt die Abbildung \[R\colon\set{\psi\colon M\to\bar K}{ \psi|_K=id}\to\set{\varphi\colon L\to\bar K}{ \varphi|_K=id},\ \psi\mapsto \psi|_L\]  \[ \abs{R^{-1}\Set{\varphi}}=[M\colon L]_s.\] Also
    \[[M:K]_s=\abs{\set{\psi\colon M\to\bar K}{ \psi|_K=id}}=[M:L]_s\cdot[L:K]_s\]
\end{proof}
\begin{Satz}
    Sei \(L/K\) endlich.
    \begin{enumerate}
        \item \([L:K]_s\leq [L:K]\)
        \item Es ist äquivalent:
        \begin{enumerate}
            \item \([L:K]_s=[L:K]\)
            \item \(L/K\) ist separable
            \item \(L/K\) ist von separablen Elementen erzeugt
        \end{enumerate}
    \end{enumerate}
\end{Satz}
\begin{proof}
   \begin{enumerate}
       \item  Wähle \(a_1,\dots,a_r\) sodass \(L=K(a_1,\dots,a_r)\). 
    Es gilt \[[L:K]=[L:K(a_1)]\cdot[K(a_1):K]\] und 
    \[[L:K]_s=[L:K(a_1)]_s\cdot [K(a_1):K]_s\]
    Jetzt folgt 1. mit Induktion.
    \item \(b)\implies c)\) ist klar. Gelte c). \(L=K(a_1,\dots,a_r)\) mit \(a_i\) separable.
    Dann ist \[[K(a_1):K]_s=[K(a_1):K]\] und nach Induktion \([L:K(a_1)]_s=[L:K(a_1)]\). Also gilt a).
    Gelte a) und sei \(a\in L\). Dann ist 
    \begin{align*}
        [L:K]&=[L:K(a_1)]\cdot[K(a_1):K]\\
        & \geq [L:K(a_1)]_s\cdot[K(a_1):K]_s\\
        &=[L:K]_s=[L:K]
    \end{align*}
    Also ist \([K(a_1):K]=[K(a_1):K]_s\) und somit ist \(a\) separable. Also gilt b).
   \end{enumerate}
\end{proof}
\begin{Kor}
    Seien \(M/L/K\) algebraisch.
    \[M/K \text{ separable}\iff M/L \text{ und } L/K\text{ separable }\]
\end{Kor}
\begin{proof}
    Sei \(M/K\) separable und \(a\in M\). Sei \(f\in K[X]\) das Minimalpolynom über \(K\) und \(g\in L[X]\) das Minimalpolynom über \(L\). Dann \(g|f\) in \(L[X]\) also ist \(g\) separable und damit auch \(a\) separable über \(L\).\\
    Seien \(M/L\) und \(L/K\) separable. Sei \(a\in M\) und \(g\in L[X]\) das Minimalpolynom. Wähle \(K\subseteq L'\subseteq L\) sodass \(L'/K\) endlich ist und \(g\in L'[X]\). Sei \(M'=L'(a)\). Dann ist \(M'/L'/K\) endlich. Minimalpolynom von \(a\) über \(L'\) ist \(g\) und das ist separable, da \(M/L\) separable. Somit ist \(M'/L'\) separable. \(L'/K\) ist separable das \(L/K\) separable.
    Somit ist ohne Einschränkung \(M/L/K\) endlich.
    Dann zeigt eine Rechnung mit den Graden der Erweiterungen, dass \(M/K\) separable ist.
\end{proof}
\begin{Kor}
    Der Zerfällungskörper einer Menge von separablen Polynomen ist eine separable Erweiterung.
\end{Kor}
\begin{proof}
    Sei \(L/K\) Zerfällungskörper von \(\mathcal F\in K[X]\) bestehend aus separablen Polynomen. Zu \(a\in L\) gibt es endliche Teilmenge \(\mathcal F'\subseteq \mathcal F\) sodass \(a\in L'\) wobei \(L'\subseteq L\) Zerfällungskörper von \(\mathcal F'.\) \(L'/K\) ist endlich und von separablen Elementen erzeugt. Somit ist \(L'/K\) separable. Insbesondere ist \(a\) separable über \(K\).
\end{proof}
\begin{Kor}
    Die normale Hülle einer separablen Erweiterung ist ebenfalls separable
\end{Kor}
\begin{proof}
    Folgt aus der Konstruktion der normalen Hülle.
\end{proof}
\section{Spur und Norm}
\begin{Def}
	Sei \(L/K\) eine endliche Körpererweiterung und \(b\in K\).
	Dann ist \[Tr_{L/K}(b)=Tr(b\cdot \colon L\to L)\] die Spur und
	\[N_{L/K}(b)=\det(b\cdot\colon L\to L)\] die Norm von \(b\).
\end{Def}
\begin{Bem}
	Es ist für \(a\in K\)
	\[Tr_{L/K}(a)=[L:K]a\] und
	\[N_{L/K}(a)=a^{[L:K]}\]
\end{Bem}
\begin{Lemma}\label{Lem:CharMinPot}
	Sei \(L/K\) eine endliche Körpererweiterung und \(b\in L\). Sei \(P\) das Minimalpolynom von \(b\) über \(K\),
	\(\deg(P)=d\) und \(e=[L:K(b)]=\frac{[L:K]}{d}\). Dann ist das charackteristische Polynom \[\chi_{b\colon L\to L}=P^e.\]
\end{Lemma}
\begin{proof}
	Wenn \(L=K(b)\) dann ist \(P\mid \chi_b\) und da beide den gleichen Grad haben sind sie gleich.
	Allgemein ist \[L=K(b)^e\] als Vektorraum. Multiplikation mit \(b\) respektiert diese Zerlegung also ist
	\(\chi_{b\colon L\to L}=\chi_{b\colon K(b)\to K(b)}^e=P^e\)
\end{proof}
\begin{Kor}
	Sei \(L/K\) endliche Körpererweiterung und \(P=X^d+a_{d-1}X^{d-1}+\dots+a_0\) das Minimalpolynom von \(b\in L\). Dann ist 
	\[Tr_{L/K}(b)=-ea_{d-1}\] und
	\[N_{L/K}(b)=(-1)^{[L:K]}a_0\] wobei \(ed=[L:K]\)
\end{Kor}
\begin{proof} Klar nach \cref{Lem:CharMinPot}.
\end{proof}
\begin{Satz}\label{Satz:MipolCharSep} Wenn \(L/K\) eine endliche separable Erweiterung ist, dann ist für \(b\in L\) das Minimalpolynom
	\[\mu_b(X)=\prod_{\sigma\in\Sigma}(X-\sigma(b))\] und das charackteristische Polynom
	\[\chi_b(X)=\prod_{\sigma\in\Sigma'}(X-\sigma(b))\]
	wobei \(\Sigma=\{\sigma\colon K(b)\to \bar K\mid \sigma|_K=\id\}\) und \(\Sigma'=\{\sigma\colon L\to \bar K\mid \sigma|_K=\id\}\).
	\end{Satz}
\begin{proof}
	Da \(b\) separabel ist, sind \(\sigma(b)\) die \(deg(\mu_b)\) vielen verschiedenen Nullstellen für \(\sigma\in\Sigma\). Somit \(\mu_b=prod_{\sigma\in\Sigma}(X-\sigma(b))\).
	Jedes \(\sigma\in\Sigma\) hat genau \(e=[L:K(b)]\) viele
	Fortsetzungen zu \(\sigma'\in \Sigma'\), denn \(L/K(b)\) ist separabel und \ref{Lem:SeparabilitätsgradKörpererw}.
	Daher ist 
	\[\prod_{\sigma\in\Sigma'}(X-\sigma(b))=\prod_{\sigma\in\Sigma}(X-\sigma(b))^e=\mu_b^e=\chi_b\] nach ??.
\end{proof}
\begin{Kor}
	Wenn \(L/K\) endlich separabel ist und \(b\in L\), dann ist 
	\[Tr_{L/K}(b)=\sum_{\sigma\in\Sigma}\sigma(b)\] und
	\[N_{L/K}(b)=\prod_{\sigma\in\Sigma}\sigma(b)\] wobei
	\(\Sigma=\{\sigma\colon L\to\bar K\mid \sigma|_K=\id\}\).
\end{Kor}
\begin{proof}
	Klar nach \cref{Satz:MipolCharSep}.
\end{proof}
\begin{Lemma} Sei \(L\) ein Körper und \(a_1,\dots,a_n\) paarweise verschieden. Wenn \(\lambda_1,\dots,\lambda_n\) nicht alle \(0\) sind, dann gibt es ein \(e\geq 0\) sodass \(\sum_{i=1}^n\lambda_i a_i^e\neq 0\).
\end{Lemma}
\begin{proof}
	Wenn \(n=1\) dann ist die Aussage klar.
	Angenommen die Aussage gilt für \(n-1\). Angenommen die Aussage ist falsch für \(\lambda_1,\dots,\lambda_n\). Ohne Einschränkung \(\lambda_n=-1\) sodass \(a_n^e=\sum_{i=1}^{n-1}\lambda_i a_i ^e\) für alle \(e\).
	
	\[a_na_n^e=a_n^{1+e}=\sum_{i=1}^{n-1}\lambda_i a_i^{1+e}=\sum_{i=1}^{n-1}\lambda_i a_ia_i^e\].
	Die erste Gleichung mit \(a_n\) Multiplizieren und beide Gleichungen voneinander abziehen liefer 
	\[0=\sum_{i=1}^{n-1}\lambda_i (a_i-a_n)\]
	Da nicht alle \(\lambda_i=0\) ist \(a_i=a_n\) für ein \(i\) nach Induktion für ein \(e\) was ein Widerspruch ist.
\end{proof}
	
\begin{Satz}
	Sei \(L/K\) eine endliche Körpererweiterung. Es ist äquivalent:
	\begin{enumerate}
		\item \(L/K\) ist separable
		\item \(Tr_{L/K}\colon L\to K\) ist nicht die Nullabbildung
		\item Die Abbildung \((x,y)\mapsto Tr_{L/K}(xy)\) ist nicht-ausgeartete Bilinearform \(L\times L\o K\).
	\end{enumerate}
\end{Satz}
\begin{proof}
	Die Äquivalenz von (2) und (3) ist klar.
	Ohne Einschränkung hat \(K\) die Charakteristik \(p\) denn in Charakteristik \(0\) Fall sind alle
	Erweiterungen separabel und \(Tr_{L/K}(1)=[L:K]\neq 0\).
	Es gilt \(Tr\neq 0\) genau dann wenn \(Tr\) surjektiv, denn \(Tr\) ist \(K\)-linear.
	Wenn \(L/K\) separabel, dann ist \(L=K(\alpha)\) für ein \(\alpha\in L\) nach ????.
	Wenn \(L/K\) inseperabel, dann gibt es \(\alpha\in L\) inseperabel über \(K\).
	Da \(Tr_{L/K}=Tr_{K(\alpha)/K}\circ Tr_{L/K(\alpha)}\) ist ohne Einschränkung \(L=K(\alpha)\).
	Sei \(P\) das Minimalpolynom von \(\alpha\) in \(K[X]\).
	Dann ist \(P(X)=\tilde P(X^{p^m})\) für ein maximales \(m\) sodass \(\tilde P\) separabel da irreduzibel und 
	\(P\) ist genau dann separabel, wenn \(m=0\).
	Sei \(n=\deg(P)=p^md\) with \(d=\deg(\tilde P)\).
	In \(\bar K\) gilt \(\tilde P(X)=(X-\beta_1)\dots (X-\beta_d)\) für paarweise verschiedene 
	\(\beta_i\). 
	Dann ist 
	\[P(X)=\tilde P(X^{p^m})=\prod (X^{p^m}-\beta_i)=\prod (X-\gamma_i)^{p^m}\]
	for \(\gamma_i\in\bar K\).
	Da Körpererweiterungen treuflach sind nach ??? ist \(Tr_{L/K}\colon L\to K\) surjektiv genau dann wenn 
	\(\bar{Tr}=\id_{\bar K}\otimes Tr_{L/K}\) surjektiv ist.
	Da \(L=K(\alpha)=K[X]/P\) ist \(\bar K\otimes_KL=\prod \bar{K[X]}/(X^{p^m}-\beta_i)\) und die Spur ist die Summe
	der Spuren auf \(\bar K[X]/(X^{p^m}-\beta_i)\).
	Es ist \(\bar K[X]/(X^{p^m}-\beta_i)\cong \bar K[Y]/(Y^{p^m})\) unter der Substitution
	\(Y=X-\gamma_i\).
	Wenn \(m=0\) dann ist \(\bar K[Y]/(Y^{p^m})=\bar K\) und die SPur ist die Identität.
	Wenn \(m>0\) dann ist jedes Element in \(\bar K[Y]/(Y^{p^m})\) Summe von etwas Konstantem und etwas nilpotenten.
	Jede Konstante hat Spur \(0\) denn \(p^m=0\) in \(\bar K\).
	Jedes nilpotente Element hat auch Spur \(0\). Also hat jedes Element in \(\bar K[Y]/(Y^{p^m})\) Spur \(0\).
\end{proof}
\begin{Satz}
	Sei \(L/K\) eine endliche Galoiserweiterung und \(A\subseteq K\) normal.
	Sei \(B\) der ganze Abschluss von \(A\) in \(L\).
	Wenn \(b\in B\) dann ist \(Tr(b)\in A\).
\end{Satz}
\begin{proof}
	Sei \(\varphi(x)=\prod_{g\in G}(X-g(b))\) mit \(G=\Gal(L/K)\). Das ist das Charakterisitsche Polynom nach ?? und ??.
	Alle Koefffizienten von \(\varphi\) sind in \(K\). Da \(B\) ganz ist über \(A\) ist auch \(g(b)\) ganz.
	Also sind alle Koeffizienten ganz über \(A\) und damit in \(A\) da \(A\) normal.
	Also ist \(Tr(b)\in A\) da \(Tr(b)\) einer der Koeffienten ist.
\end{proof}
\begin{Def}
	Sei \(K\) ein Körper und \(C\) eine endlich-dimensionale \(K\)-algebra. \(C\) heißt separabel falls die symmetrische Bilinearform \((x,y)\mapsto Tr_{C/K}(xy)\) nicht ausgeartet ist.
\end{Def}
\begin{Lemma}
	Sei \(C\) eine endlich-dimensionale \(K\)Algebra.
	\(C\) ist separabel, genau dann wenn \(C\cong L_1\times\dots\times L_r\) mit \(L_i/K\) separable Körpererweiterung.
\end{Lemma}
\begin{proof}
	Sei \(C=L_1\times\dots\times L_r\) wie im Satz.
	Sei \(B_i\) die Matrix der Spurform eingeschränkt auf \(L_i\).
	Dann ist \(B=diag(B_1,\dots,B_r)\) die Matrix der Spurform für \(C\).
	Nach ??? ist \(B_i\) invertierbar und somit \(B\) invertierbar also \(C\) separabel.
	Andererseits sei \(C\) separabel.
	Da \(C\) artinsch ist, folgt mit ??? dass 
	\(C=\prod_{i=1}^rC_i\) wobei \(C_i\) lokal ist mit maximalen Ideal \(\frakm_i\).
	Sei \(B_i\) die Matrix der Spurform eingeschränkt auf \(C_i\).
	Da \(C\) separabel, ist \(B=diag(B_1,\dots,B_r)\) die Matrix der Spurform invertierbar. Somit ist \(B_i\) invertierbar und damit \(C_i/K\) separabel.
	Also ist ohne Einschränkung \(C\) lokal mit maximalem Ideal \(\frakm\).
	Für \(x\in\frakm\) gilt \(xy\in \frakm\) für alle \(y\in C\). Da \(C\) artinsch und lokal, ist \(\frakm\) das einzige Primideal somit gleich dem Nilradikal.
	Also ist \(xy\) nilpotent und damit \(Tr(xy)=0\) da die Spur eines nilpotenten Endomorphismus immer Spur 0 hat.
	Also ist \(x=0\) und somit \(\frakm=0\) und \(C\) ein Körper.
	
\end{proof}
