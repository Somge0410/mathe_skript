\chapter{Minkowski Theorie}
\begin{Def} Sei \(V\) ein endlich-dimensionaler \(\RR\)-Vektorraum.
	\begin{enumerate}
		\item Eine Untergruppe \(\Lambda\subseteq V\) ist ein Gitter, wenn jede \(\ZZ\)-lineare unabhängige Menge in 
		\(\Lambda\) in \(V\) \(\RR\)-linear unabhängig ist.
		\item Ein Gitter \(\Lambda\subseteq V\) ist vollständig, wenn \(\Lambda\) \(V\) als \(\RR\)-Vektorraum erzeugt.
	\end{enumerate}
	
\end{Def}
\begin{Bsp}
	\(\ZZ^n\subseteq \RR^n\) ist vollständiges Gitter. \(\ZZ[i]\subseteq\CC\) ist vollständiges Gitter mit Basis \(1,i\). Aber \(\ZZ[\sqrt 2]\subseteq \RR\) ist kein Gitter.
\end{Bsp}
\begin{Bem}
	Sei \(\Lambda\subseteq V\) und \(\alpha\colon\Lambda \otimes_\ZZ\RR\to V,(x,v)\mapsto xv\).
	Es ist \(\Lambda\) ein Gitter \(\iff \alpha\) injektiv ist.
	\(Lambda\) ist vollständiges Gitter \(\iff \alpha\). bijektiv.
	
	Wenn \(\Lambda\subseteq V\) vollständiges Gitter ist, dann gibt eine Wahl von Basis von \(\Lambda\) und 
	% https://tikzcd.yichuanshen.de/#N4Igdg9gJgpgziAXAbVABwnAlgFyxMJZABgBpiBdUkANwEMAbAVxiRAB12AZOgWwCModEAF9S6TLnyEUARnJVajFmwBqo8SAzY8BImVmL6zVog7sAWhYB6hMRJ3Si8w9WMqznAEpfboxTBQAObwRKAAZgBOELxIAEzUOBBIAMyJdFgMbAAWEBAA1hoR0bGIZCBJ8dQMWGCmIFAQTPwMrNTZMHRQbJB1IOmZPQRtSiZsnNE4dDgw-BAAHsAAnMQiwJwAFJzYvJwAlCJFIFExSPIVyYhpIB1dQ33VtfWNza39FRlZZr0j7vUTECmMzmixWa022ywu3YByOJ1K5UqiHOU0GZlyBX8IiAA
	\begin{tikzfigure}
		\Lambda \arrow[d, "\rotatebox{90}{\(\sim\)}", no head, equal] \arrow[r, hook] & V \arrow[d, "\rotatebox{90}{\(\sim\)}", no head, equal] \\
		\ZZ^n \arrow[r, hook]                                                              & \RR^n                                                       
	\end{tikzfigure}
\end{Bem}
\begin{Satz} Sei \(V\) ein \(\RR\)-Vektorraum.
	\begin{enumerate}
		\item Eine Untergruppe \(\Lambda\subseteq V\) ist Gitter \(iff \Lambda\) ist diskret in \(V\).
		\item Ein Gitter \(\Lambda\subseteq V\) ist vollständig \(iff V/\Lambda\) ist kompakt.
	\end{enumerate}
\end{Satz}
\begin{proof}
	\begin{enumerate}
		\item Wenn \(Lambda\) Gitter dann ist es diskret da \(\ZZ^n\subseteq \RR^n\) diskret ist.
		Sei also \(\Lambda\subseteq V\) diskret.
		Ersetze \(V\) durch \(RR\cdot \Lambda\) und wähle \(RR\)-Basis \(v_1,\dots,v_n\in \Lambda\).
		Sei \(Lambda_0=\langle v_1,\dots,v_n\rangle_\ZZ\).
		Dann ist \(\Lambda_0\) Gitter in \(V\).
		Es ist \(\Lambda/\Lambda_0\subseteq V/\Lambda_0\) wobei die Quotiententopologie gleich der Teilraumtopologie ist nach ???.
		Da \(\Lambda\subseteq V\) diskret, ist \(\Lambda/\Lambda_0\subseteq V/\Lambda_0\) diskret.
		Da \(\Lambda_0\subseteq\) vollständiges Gitter ist, ist \(V/\Lambda_0\) kompakt nach (2).
		Also ist \(\Lambda/Lambda_0\) auch kompakt da \(\Lambda/\Lambda_0\subseteq V/\Lambda_0\) abgeschlossen denn \(\Lambda\subseteq V\) ist diskrete Untergruppe einer Hausdorff Gruppe.
		Also ist zusammen \(\Lambda/\Lambda_0\) kompakt und diskret also endlich.
		Also ist \(\Lambda\) endlich erzeugte abelsche Gruppe.
		Nach ??? Struktursatz ist \(\Lambda\cong ZZ^r\). Da \(v_1,\dots,v_n\) linear unabhängig sind, ist \(\Lambda_0=\Lambda\) und somit \(\Lambda\) Gitter.
		\item Wenn \(\Lambda\) vollständig, dann ist \(\RR^n/\ZZ^n=(S^1)^n\) kompakt.
		Wenn \(\Lambda\subseteq V\) Gitter sodass \(V/\Lambda\) kompakt ist, dann sei \(V_0=\RR\Lambda\).
		Es ist \(V=V_0\oplus V_1\) nach ergänzen einer Basis von \(V_0\).
		Also ist \(V/\Lambda\cong V_0/\Lambda \oplus V_1=\text{kompakt}\times \RR^m\) kompakt. Also ist \(m=0\) und damit \(V_0=V\).
	\end{enumerate}
\end{proof}
\begin{Def}
	Sei \(\Lambda\subseteq \RR^n\) ein Gitter. Eine Grundmasche ist eine Teilmenge \(F\subseteq \RR^n\)
	sodass jedes \(x\in\RR^n\) genau eine Darstellung
	\(x=\lambda+\gamma\) hat mit \(\lambda\in \Lambda\) und \(\gamma\in F\).
\end{Def}
\begin{Lemma}
	Ein Gitter \(\Lambda\) ist vollständig genau dann wenn eine beschränkte Grundmasche existiert.
\end{Lemma}
\begin{proof}
	Sei \(\Lambda\) vollständig. Wähle \(\ZZ\)-Basis \(v_1,\dots,v_n\) von \(\Lambda\) und setze 
	\(F=\{\sum_{i=1}^na_iv_i\mid 0\leq a_i<1\}\). Dann ist \(F\) eine Fundamentalmasche.
	Wenn es \(F\) beschränkte Fundamentalmasche gibt, dann ist \(\RR^n/\Lambda\) kompakt nach ??? da \(\Lambda\) diskret.
	Nach ??? ist \(\Lambda\) vollständig.
	
\end{proof}
\begin{Lemma}
	Seien \(F_1,F_2\) zwei Fundamentalmaschen in \(\RR^n\) für ein vollständiges Gitter \(\Lambda\subseteq \RR^n\).
	Dann gilt \(\Vol(F_1)=\Vol(F_2)\).
\end{Lemma}
\begin{proof}
	Sei \(\mathbbm{1}_{F_1}\) die charakteristische Funktion von \(F_1\).
	Es ist 
	\begin{align*}
	\Vol(F_1)&=\int_{\RR^n}\mathbbm{1}_{F_1}(x)\text{d}x\\
	&=\sum_{g\in\Lambda}\int_{F_2+g}\mathbbm{1}_{F_1}(x)\text{d}x\\
	&=\sum_{g\in\Lambda}\int_{F_2}\mathbbm{1}_{F_1}(x+g)\text{d}x\\
	&=\int_{F_2}\sum_{g\in\Lambda}\mathbbm{1}_{F_1}(x+g)\text{d}x\\
	&=\int_{F_2}1\text{d}x\\
	&=\Vol(F_2)
	\end{align*} wobei man Summe und Integral nach Fubini-Tonelli ??? vertauschen darf.
\end{proof}
Sei ab jetzt \(V\) ein endlich-dimensionaler eukldischer Vektorraum. Wähle Isometrie \(\varphi\colon (V,\langle , \rangle)\to (\RR^n,\cdot)\)
und definiere ein Maß \(\mu\) auf \(V\) durch \(\mu(A)=\lambda(\phi(A))\) für Messbare Mengen \(A\) wobei \(\lambda\) das Lebesgueq Maß auf \(\RR^n\) ist. Das ist wohldefiniert denn für eine andere Isometrie sind die Orthonormalbasen verbunden durch eine orthonormale Matrix und das Lebesgue Maß ist invariant unter diesen Transformationen nach ???.
\begin{Lemma}
	Sei \(V\) ein Euklidischer Vektorraum und \(v_1,\dots,v_n\in V\) und sei \(\Gamma\) das Parallelopid das aufgespannt wird von \(v_1,\dots,v_n\).
	Dann ist \(\mu(\Gamma)=|\det(A)|\) wobei \(A\) die Darstellungsmatrix ist der \(v_1\) bezüglich einer Orthonormalbasis.
\end{Lemma}
\begin{proof}
	Eine Orthonormalbasis gibt Isometrie zu \(\RR^n\) mit Standardskalarprodukt. Somit ist ohne Einschränkung \(V=\RR^n\) und die Orthonormalbasis die Standardbasis.
	Sei \(A\) die Abbildungsmatrix. Nach QR-Zerlegung ???
	ist \(A=QR\) wobei \(Q\) Orthogonal ist und \(R\) obere Dreiecksmatrix. Wegen Invarianz des Lebesgue Maß von Orthogonalen Transformationen ist ohne Einschränkung \(A\)
	obere Dreiecksmatrix. Dann ist sowohl das Volumen als auch der Betrag der Determinante gleich dem Produkt der Diagoaleinträge.
\end{proof}
\begin{Lemma}
	Seien \(M\subseteq N\) zwei vollständige Gitter in \(\RR^n\) und sei \(A\in M_n(\ZZ)\) die Übergangsmatrix von einer Basis von \(N\) zu einer Basis von \(M\).
	Es gilt
	\begin{enumerate}
		\item \(\Vol(M)=|\det(A)|\Vol(N)\)
		\item \(|N/M|=|det(A)|\)
	\end{enumerate}
\end{Lemma}
\begin{proof}
	Sei \(v_1,\dots,v_n\) eine Basis von \(N\) und \(w_1,\dots,w_n\) eine Basis von \(M\) und \(A\) die Übergangsmatrix.
	Dann ist \((w_1,\dots,w_n)=(v_1,\dots,v_n)\cdot A\) und somit
	\[\Vol(M)=|\det(w_1,\dots,w_n)|=|\det(A)|\cdot |\det(v_1,\dots,v_n)|=|\det(A)|\Vol(N).\]
	Die zweite Aussage kann man rein algebraisch Zeigen. Oder aber so:
	Sei \(x\in N\). Es gibt Darstellung \(x=\sum_i \beta_iw_i\) mit \(\beta_i\in\RR\). Sei \(y=\sum_i\lfloor \beta_i \rfloor w_i\in M\).
	Dann ist \(x-y\in \Gamma_M\cap N\).
	Wenn \(x_1-y_1=x_2-y_2\in \Gamma_M\cap N\) für \(x_1,x_2\in N\) und \(y_1,y_2\in M\) dann ist 
	sind die Koeffizienten von \(x_1-x_2\) ganze Zahlen aber auch im Betrag \(<1\) also 0.
	Das heißt es gibt Bijektion \(N/M\to N\cap \Gamma_M\).
	Sei \(\Gamma_M\cap N=\{s_1,\dots,s_k\}\). 
	Es ist \(s_i+\Gamma_N\) disjunkt für verschiedene \(i\).
	Es ist \(F=\bigcup_i s_i+\Gamma_N\) eine Grundmasche von \(M\), denn für \(x\in \RR^n\) gibt es eine eindeutige Darstellung \(x=n+\gamma\) wobei \(n\in N\) und \(\gamma\in \Gamma_N\). Sei \(s_i\) gegeben durch \(n\equiv s_i \mod M\). Dann ist \(x=m+s_i+\gamma\) und die Darstellung ist eindeutig.
	Also ist \(F\) eine Grundmasche und \(\Vol(\Gamma_M)=\Vol(F)=k\cdot\Vol(\Gamma_N)\).
	Nach 1. folgt, dass \(|N/M|=|\det(A)|\)
	
\end{proof}
\begin{Def}
	Das Volumen \(\Vol(\Lambda)\) von Gitter \(\Lambda\) ist definiert als \(\mu(\Gamma)\) wobei \(\Gamma\) eine Grundmasche ist.Das ist wohldefiniert.
	
\end{Def}
\begin{Bem}
	Es ist \(A^t\cdot A=\langle v_i,v_j\rangle_{i,j}=B\)
	und somit \(\det(B)=\det(A)^2\).
	Also ist \(\Vol(A)=\sqrt{\det(B)}\)
\end{Bem}
\section{Miskowskysche Gitterpunktsatz}
\begin{Satz}[Gitterpunktsatz] \label{Satz:Gitterpunktsatz}
	
	Sei \(V\) ein Euklidischer Vektorraum und \(\Lambda\subseteq V\) ein vollständiges Gitter und \(X\subseteq V\) eine konvexe, zentralsymmetrische (dh. \(x\in X\implies -x\in X\)) Menge.
	Wenn \[\mu(X)>2^n\Vol(\Lambda)\] dann ist \(X\cap \Lambda\neq\{0\}\)
\end{Satz}
\begin{proof}
	Es reicht zu zeigen, dass es \(\gamma_1,\gamma_2\in \Lambda\) gibt mit \(\gamma_1\neq\gamma_2\) sodass
	\[(\frac 1 2 X+\gamma_1)\cap(\frac 1 2 X+\gamma_2)\neq \emptyset\] denn wenn
	\[\frac 1 2 x_1+\gamma_1=\frac 1 2 x_2+\gamma_2\] mit \(x_1,x_2\in X\) dann ist 
	\[\gamma=\gamma_1-\gamma_2=\frac 1 2 x_2-\frac 1 2 x_1\in X\cap \Lambda\]
	Also angenommen \(\frac 1 2 X+\gamma\) für \(\gamma\in\Lambda\) sind paarweise disjunkt.
	Dann sind auch \(\Gamma\cap (\frac 1 2 X+\gamma)\) paarweise disjunkt wobei \(\Gamma\) eine Grundmasche ist von \(\Lambda\)
	Dann ist 
	\begin{align*}
		\Vol(\Lambda)&\geq \sum_{\gamma\in \Lambda}\Vol(\Gamma\cap() \frac 1 2 +\gamma))\\
		&=\sum_{\gamma\in\Lambda}\Vol((\Gamma-\gamma)\cap \frac 1 2 X)\\
		&= \Vol(\frac 1 2 X)\\
		&= \frac{1}{2^n}\Vol(X)
	\end{align*} was ein Widerspruch ist.
\end{proof}
\begin{Kor}
	Jede natürliche Zahl lässt sich schreiben als Summe von vier Quadraten.
\end{Kor}
\begin{proof}
	Es ist 
	\begin{align*}
	(a^2+b^2+c^2+d^2)(A^2+B^2+C^2+D^2)&=(aA-bB-cC-dD)^2+
	(aB+bA+cD-dC)^2\\&+(aC-bD+cA+dB)^2+(aD+bC-cB+dA)^2
	\end{align*}
	Also reicht es die Aussage für eine Primzahl \(p\) zu zeigen.
	Es ist \(2=1^2+1^2+0^2+0^2\) also ist ohne Einschränkung \(p\geq 3\).
	Es ist \[|\{m^2 \mod p\mid m=0,\dots p-1\}|=\frac{p+1}{2}\] und
	\[|\{-1-n^2 \mod p\mid n=0,\dots p-1\}|=\frac{p+1}{2}\]
	Also gibt es \(m,n\in\ZZ\) sodass \(m^2+n^2+1\equiv 0 \mod p\).
	Sei \[\Lambda=\{(a,b,c,d)\in\ZZ^4\mid c\equiv ma+nb,\, d\equiv mb-na \mod p\}\] Gitter.
	Dann ist \(p\ZZ^4\subseteq \Lambda\subseteq ZZ^4\)
	und \(\Lambda/p\ZZ^4\subseteq \FF_p^4\) Unterraum der Dimension 2.
	Also ist der Index \([\ZZ^4\colon \Lambda]=p^2\)
	und somit ist \(\Vol(\Lambda)=p^2\).
	Sei \(T=\bar{B_r(0)}\) und \(r\) so gewählt, dass \(2p>r^2>\frac{4\sqrt{2}}{\pi}p\).
	Dann ist
	\[\Vol(T)=\frac{\pi^2r^4}{2}>16p^2=\Vol(\Lambda)\]
	Nach \nameref{Satz:Gitterpunktsatz} gibt es \((a,b,c,d)\in \Lambda\setminus\{0\}\cap T\) das heißt
	\(a^2+b^2+c^2+d^2\equiv a^2(1+m^2+n^2)+b^2(1+n^2+m^2)\equiv 0 \mod p\).
	Da \((a,b,c,d)\in T\) ist \(a^2+b^2+ c^2+ d^2<2p\).
	Also ist \(a^2+b^2+c^2+d^2=p\).
\end{proof}
\begin{Bem}
	Sei \(K\) ein Zahlkörper. \(\calO_K\cong \ZZ^n\) soll ein Gitter werden in einem passenden \(\RR\)-Vektorraum.
	Da jede \(\ZZ\)-Basis von \(\calO_K\) eine \(\QQ\)-Basis ist, ist die natürliche Abbildung
	\(\calO_K\otimes_\ZZ\QQ\to K\) bijektiv. 
	Betrachte Diagramm
	% https://tikzcd.yichuanshen.de/#N4Igdg9gJgpgziAXAbVABwnAlgFyxMJZABgBpiBdUkANwEMAbAVxiRAB12AtLkAX1LpMufIRQBGclVqMWbTgEUF-QSAzY8BIgCYp1es1aIO7AEqmVQjaKIBmPTMPz2AYReW1wzWJKlx0gzljTgB5EIB9AGkPdREtCT8A2SMTMKjOCDwAW3hwzh5FZQErOJ9df31k5zTIjOzc-K46rBy4AAJOcxivGxR7Cscg1Ija9kyWhu4m13c+aRgoAHN4IlAAMwAnCCykMhAcCCRJfbosBjYACwgIAGsPTe2j6gOkXROzy+u74pAHncQ3i9EPZ3udjFdbvctv89kCACw-P5PfaHRAAVkR0Nez1RADZMY9gTikAB2An-OHE9HPU5gkAQ76qJHUlFIXE0j7gr5Qwns1mIEkcukM-gUPhAA
	\begin{tikzfigure}
		\ZZ \arrow[r, hook] \arrow[d] & \QQ \arrow[r, hook] \arrow[d]       & \RR \arrow[r, hook] \arrow[d]               & \CC \arrow[d]       \\
		\calO_K \arrow[r, hook]         & \calO_K\otimes_\ZZ\QQ \arrow[r, hook] & \calO_K\otimes_\ZZ\otimes \RR \arrow[r, hook] & \calO_K\otimes_\ZZ\CC
	\end{tikzfigure}
	Sei \(K_\RR=\calO_K\otimes_\ZZ\RR\cong (\calO_K\otimes_\ZZ \QQ)\otimes_\QQ\RR\cong K\otimes_\QQ\RR\) der Minkowski Raum. Es sei \(K_\CC=\calO_K\otimes_\ZZ\CC\).
	Es ist \(\calO_K\) ein vollständiges Gitter in \(K_\RR\) denn die Abbildung 
	\(\calO_K\otimes_\ZZ\RR\to K_\RR, (x,a)\mapsto a\cdot x=(1\otimes a)(x\otimes 1)=x\otimes a\) ist bijektiv.
\end{Bem}
\begin{Lemma}
	Die Komplexe Konjugation \(\tau\colon\CC\to\CC\) induziert einen Homomorphismus 
	\[\tilde\tau=\id\otimes \tau\colon K\otimes_\QQ\CC\to K\otimes_\QQ\CC\].
	Die Invarianten davon sind 
	\[(K\otimes_\QQ\CC)^{\tilde\tau}=\{x\in K\otimes\CC\mid \tilde\tau(x)=x\}=K_\RR\]
\end{Lemma}
\begin{proof}
	Wähle eine \(\QQ\)-Basis von \(K\cong \QQ^n\).
	Dann ist \(K\otimes_\QQ\RR\cong \RR^n\) und \(K_\CC\cong \CC^n\).
	Also ist
	\((K_\CC)^{\tilde\tau}\cong (\CC^n)^{\tilde\tau}=\RR^n\cong K\otimes_\QQ\RR\).
\end{proof}
\begin{Lemma}
	Es gibt Isomorphismus von \(\CC\)-Algebren \(\psi\colon K_\CC\to \prod_{\tau}\CC,\psi(a\otimes b)=(\tau(a)\cdot b)_\tau\) wobei \(\tau\) alle Körperhomomorphismen \(\tau\colon K\to\CC\) durchläuft.
\end{Lemma}
\begin{proof}
	\(K_\CC\) ist endlich-dimensionale \(\CC\)-algebra, also ein Modul endlicher Länge und damit Artinscher Ring.
	Nach \ref{Kor:StruktArtinring} ist \(K_\CC\cong \prod R_i\) mit \((R_i,\frakm_i)\) lokale artinsche Ringe. Es gilt dass
	\(R_i/\frakm_i=\CC\) da \(\CC\) algebraisch abgeschlossen ist.
	Seien \(\tau_1,\dots,\tau_n\) die Homomorphismen \(K\to \CC\). Es ist \(n=[K:\QQ]_S=[K:\QQ]\).
	Also sind auch \(\tilde\tau_1,\dots,\tilde\tau_n\) verschiedene Abbildungen \(K_\CC\to \CC\).
	Jeder Homomorphismus von \(\CC\)-Algebren \(\prod R_i\to \CC\) hat die Form \(\prod R_i\to R_i\to R_i/\frakm_i \cong \CC\) siehe ???.
	Also ist Anzahl \(r\) der Homomorphismen \(K_\CC\to\CC\) gleich Anzahl der Faktoren.
	Da jede Algebra \(R_i\) als Vektorraum mindestens Dimension 1 hat, muss \(n\geq r\) sein. Aber oben haben wir gesehen, dass \(r\geq n\). Also ist \(r=n\) und \(\dim_\CC(R_i)=1\). Somit ist \(R_i\cong \CC\).\\
	Alternativ kann man auch wie folgt argumentieren:
	Sei \(b\in K\) sodass \(K=\QQ[b]\cong \QQ[X]/(f)\).
	Dann ist \(K_\CC\cong \CC[X]/(f)\).
	Wenn \(f=\prod_{i=0}^r(X-b_i)\) mit paarweisen verschiedenen Nullstellen in \(\CC\) dann ist 
	\[K_\CC\cong \prod_{i=1}^r\CC[X]/(X-b_i)\cong \prod\CC_i\]
	Es ist \(\{b_1,\dots,b_n\}=\{\tau(b)\colon \tau\colon K\to \CC\}\).
	Sei \(\tau_i\colon K\to \CC\) sodass \(\tau(b)=b_i\).
	dann ist die Abbildung gegeben durch
	% https://tikzcd.yichuanshen.de/#N4Igdg9gJgpgziAXAbVABwnAlgFyxMJZABgBpiBdUkANwEMAbAVxiRAGkB9AHW4GE+IAL6l0mXPkIoAjOSq1GLNrzQAnaJ2BYAvNKEA9MLwHDRIDNjwEiAJjnV6zVohDHBIsZclEy0+Y6UXACNeCDwAW3gAAmlTTwlrGVI-B0VnEAAKIM4sAEpNWV4oMLhSMCE483ErKWQ7FIUnNmydXhw6Jhys3OF5GCgAc3giUAAzdXCkMhAcCCRZRsDXbmxJjxBxiEnEBdmkO0X0tRzKze2AZmo9xAAWK7osBjZwujQ4PfWzpDuZucQAVnuj2er3ecyEFCEQA
	\begin{tikzfigure}
		K_\CC \arrow[r, "\sim"]       & \prod_{i=1}^n\CC \arrow[r, "pr_i"]     & \CC           \\
		b\otimes 1 \arrow[r, maps to] & {(b_i)_{1,\dots,n}} \arrow[r, maps to] & b_i=\tau_i(b)
	\end{tikzfigure}
	folglich ist die Abbildung wie im Satz auch für alle Elemente in \(K_\CC\).
\end{proof}
\begin{Bsp}
	Sei \(d>1\) quadratfrei und \(K=\QQ[\sqrt{d}]\cong \QQ[X]/(X^2-d)\)
	Dann ist \(K\otimes_\QQ\RR\cong \RR[X]/(X-\sqrt{d})\times \RR[X]/(X+\sqrt{d}\cong \RR\times\RR\) wobei \(1\otimes a\) auf \(a,a\) geschickt wird und \(\sqrt{d}\) auf \(\sqrt{d},-\sqrt{d}\).
		Wenn \(d\not\equiv 1 \mod 4\) dann ist \(d_K=d\) und \(1,\sqrt{d}\) eine Ganzheitsbasis.
		Es ist \(\Vol(\calO_K)=|\det(\begin{pmatrix} 1 & \sqrt d\\ 1 & -\sqrt d\end{pmatrix})|=\sqrt{d}=\sqrt{d_K}\)
		Analog wenn \(d\equiv 1 \mod 4\) dann 
		\(\Vol(\calO_K)=\sqrt{(d_K)}\)
\end{Bsp}
\begin{Bem} Sei \(K\) ein Zahlkörper.
	Sei \(c\colon \CC\to\CC\) komplexe Konjugation und \(\tilde c=\id_K\). Sei weiter \(\tau\colon K\to \CC\) Körperhom und \(\tilde\tau\colon K_\CC\to \CC,\, a\otimes b\mapsto \tau(a)\cdot b\).
	Dann kommutiert
	% https://tikzcd.yichuanshen.de/#N4Igdg9gJgpgziAXAbVABwnAlgFyxMJZABgBpiBdUkANwEMAbAVxiRAGkB9AHW4GE+IAL6l0mXPkIoAjOSq1GLNrwHDRIDNjwEiZafPrNWiDj36CRYrZKKz91Q0pMqL8mFADm8IqABmAJwgAWyQyEBwIJAAmB0VjEF48BlgAAgBjNT9AkMQwiKRZBSNlbiTYRLomTJAA4ILqfMQAZljikwzLGuzohsjm1qcE0qxkmGA03jSsfzSUiqYhYQohIA
	\begin{tikzfigure}
		K_\CC \arrow[d, "\tilde c"] \arrow[r, "\tilde\tau"] & \CC \arrow[d, "c"] \\
		K_\CC \arrow[r, "\tilde{c\circ \tau}"]              & \CC               
	\end{tikzfigure}.
	Wenn \(\rho_1,\dots,\rho_r\) die Homomorphismen \(K\to\RR\cdot \RR\) sind und \(\sigma_1,\dots,\sigma_s, c\sigma_1,\dots,c\sigma_s\) die Verbleibenden Homomorphismen \(K\to \CC\) mit \(n=[K:\QQ]=r+2s\)
	Dann schreibe \(x\in\prod_\tau \CC\) als 
	\(a_1,\dots,a_r,b_1,\dots,b_s,c_1,\dots,c_s)\).
	Dann ist 
	\[\tilde c (a_1,\dots,c_s)=(\bar{a_1},\dots,\bar{a_r},\bar{c_1},\dots,\bar{c_2},\bar{b_1},\dots,\bar{b_s}).\]
	Es folgt, dass \(K_\RR=K_\CC^{\tilde c=\id}\cong \RR^r\times \CC^s\).
\end{Bem}
\begin{Def}
	Definiere bilineare Abbildung 
	\[\beta\colon K\times K\to \RR,\, \beta(x,y)=\sum_{\tau\colon K\to\CC}\tau(x)\bar{\tau(y)}\]
	Das Bild ist in \(\RR\) denn wenn \(\tau=\rho_i\)
	Dann ist \(\rho_i(x)\bar{\rho_i(y)}\in \RR\)
	und wenn \(\tau=\sigma_i\) dann ist
	\[\sigma_i(x)\bar{\sigma_i(y)}+c\sigma_i(x)\bar{c\sigma_i(y)}=2\real(\sigma_i(x)\bar{\sigma_i(y)})\in \RR.\]
	Definiere also 
	\[\tilde\beta\colon K_\RR\to K_\RR\to \RR,\ \tilde \beta(a\otimes x,b\otimes y)=ab\beta(x,y)\]
\end{Def}
\begin{Lemma}
	\(tilde b\) ist symmetrische, positive Bilinearform gegeben durch 
	\[\tilde\beta(a_1,\dots,a_r,b_1,\dots,b_s ),a'_1,\dots,a'_n,b'_1,\dots,b'_n)=\sum_{i=1}^ra_ia'_i+\sum_{i=1}^s\real(b_ib'_i)\]
\end{Lemma}
\begin{proof}
	Symmetrie ist klar da \(\beta\) symmetrisch.
	Dass \(\tilde\beta\) positiv definit ist folgt aus der Formel da erste Summe aus dem Standardskalarprodukt auf \(\RR^n\) kommt und zweite Summe das zweifache des Standardskalarpoduktes ist auf \(\CC=\RR^2\).
	Zeige also die Formel.
	Das durch die Formel gegebene Sklarprodukt ist \(\RR\)-bilinear, also reicht die Gleichheit zu überprüfen für eine \(\RR\)-Basis von \(K_\RR\).
	Für \(a\in K\) ist entspricht \(a\otimes 1\) dem Tupel
	\((\rho_1(a),\dots,\rho_r(a),\sigma_1(a),\dots,\sigma_s(a))\)
	Damit folgt die Formel wie in der Rechnung von der Definition ???.
\end{proof}
\begin{Bem}
	Man kann genauso eine hermitesche Form \(\tilde\beta'\colon K_\CC\times K_\CC\to \CC\) definieren durch \(\tilde \beta'(a\otimes x,b\otimes y)=a\bar{b}\beta(x,y)\). Das ist dann auch positiv definit.
\end{Bem}
\begin{Def}
	wenn \(I\subseteq \calO_K\) ein Ideal ist, dann ist 
	\(N(I)=|\calO_K/I|\) die Norm des Ideals.
\end{Def}
\begin{Bem}
	Wenn \(I\subseteq \calO_K\) ein Ideal ungleich 0 ist, dann ist \(I\subseteq K_\RR\) ein vollständiges Gitter da \(I\cong \ZZ^n\) nach ???.
\end{Bem}
\begin{Satz}
	Sei \(K\) ein Zahlkörper und \(I\subseteq \calO_K\) ein Ideal ungleich \(0\). Dann ist
	\(\Vol(I)=N(I)\cdot\sqrt{|d_K|}\) wobei \(d_K\) die Diskrimminante ist.
\end{Satz}
\begin{proof}
	Es ist nach ???
	\[\Vol(I)=N(I)\cdot \Vol(\calO_K)\]
	Also ist ohne Einschränkung \(I=\calO_K\). Sei
	\(v_1,\dots,v_n\in\calO_K\) eine \(\ZZ\)-Basis und
	\(A=(Tr(v_i,v_j))_{ij}\) sowie \(A'=((\tilde beta(v_i,v_j)))\). Es \(d_K=\det(A)\) und 
	\(\Vol(\calO_K)=\sqrt{|\det(A')|}\) nach ???.
	Sei \(B=(\tau_i(v_j))_{ij}\in M_n(\CC)\).
	Dann ist \(A=B^tB\) und \(A'=B^t\bar B\).
	Somit ist \(\det(A)=\det(B)^2\) und 
	\[\det(A')=|\det(B)|^2\].
	Somit ist \(\Vol(\calO_K)=\sqrt{|d_K|}\)
\end{proof}
\begin{Satz}
	Sei \(A=(\frac 2 \pi)^s\sqrt{|d_K}\). Wenn \(I\subseteq \calO_K\) ein Ideal ungleich 0 ist und 
	\(c_\tau\in\RR_{>0}\) sodass für alle \(\tau\colon K\to\CC\) gilt: \(c_\tau=c_{c\circ\tau}\) und
	\(\prod_\tau c_\tau\geq N(I)\cdot A\).
	Dann gibt es ein \(a\in I\) mit \(a\neq 0\) sodass
	\(|\tau(a)\leq c_\tau\) für alle \(\tau\).
\end{Satz}
\begin{proof}
	Definiere \(X\subseteq K_\RR\) durch
	\[X=\{(a_1,\dots,a_r,b_1,\dots,b_s)\mid |a_i|\leq c_{\rho_i}, |b_i|\leq c_{\sigma_i}\}\]
	\(X\) ist konvex als Produkt von abgeschlossenen Bällen
	\[B_{c_{\rho_1}}(0)\times\dots \times B_{c_{\sigma_1}}\times\dots\times B_{c_{\sigma_s}}(0)\]
	und zentralsymmetrisch und kompakt.
	Es ist 
	\[\mu(X)=2^r\prod_{i=1}^rc_{\rho_i}\cdot 2^s\pi^sc_{\sigma_i}^2\] da \(\tilde\beta|_{\CC}\) das zweifache vom Standardskalarprodukt auf \(\RR^2\) ist.
	Somit ist 
	\begin{align*}
		\mu(X)&=2^n(\frac \pi 2)^s\prod_\tau c_\tau\\
		&\geq 2^n(\frac \pi 2)^sN(I)A\\
		&=2^nN(I)\sqrt{|d_K|}\\
		&=2^n\Vol(I)
		\end{align*}
	Nach Gitterpunktsatz Kompakte Version ??? folgt die Aussage.
\end{proof}
\begin{Lemma}
	Eigenschaften der Norm:
	\begin{enumerate}
		\item \(N(IJ)=N(I)N(J)\)
		\item Für \(a\in \calO_K\) gilt \(N(a\calO_K)=|N_{K/\QQ}(a)|\)
	\end{enumerate}
\end{Lemma}
\begin{proof}
	Es ist \(()\calO_K/IJ)/(I/IJ)\cong \calO_K/I\)
	also ist \(N(I)=N(IJ)/|I/IJ|\).
	Angeommen \(J\) ist Primideal und \(IJ\subseteq K\subseteq I\) für ein Ideal \(K\). Dann ist
	wegen der Eindeutigkeit der PrimidealZerlegung \(K=I\) oder \(K=IJ\).
	Also ist \(I/IJ\) ein \(\calO_K/J\) Vektorraum der Dimension 1 und somit \(N(J)=|I/IJ|\).
	Somit gilt \(N(IJ)=N(I)N(J)\) wenn \(J\) ein Primideal ist
	und somit allgemein.
	Da \(a\calO_K\subset \calO_K\) Gitter ist,
	ist \(N(a\calO_K)=|\det(A)|\) wobei \(A\) eine Übergangsmatrix der Basen der Gitter ist.
	Dann ist \(A\) aber auch die Abbildungsmatrix von Multiplikation mit \(a\), somit \(|N_{K/\QQ}(a)|=|\det(A)|\) und die zweite Aussage folgt.
\end{proof}
\begin{Lemma}
	Für \(c\in \RR>0\) gibt es nur endlich viele Ideale 
	\(I\subseteq \calO_K\) ungleich 0 sodass
	\(N(I)\leq c\).
\end{Lemma}
\begin{proof}
	Wegen ??? reicht es das für Primideale zu zeigen.
	Sei also \(I=\frakp\) prim. Es ist \(\frakp\cap \ZZ=(p)\) für eine Primzahl \(p\).
	Dann ist \(\FF_p=\ZZ/p\ZZ\to \calO_K/\frakp\) injektiv,
	Also ist \(\calO_K\) \(\FF_p\)-Vektorraum der Dimension \(\geq 1\).
	Also ist \(N(\frakp)=p^e\) mit \(e\geq 1\).
	Da \(p\calO_K\subseteq\frakp\) ist \(\frakp\) ein Faktor in der Primidealzerlegung von \(p\calO_K\).
	Somit gibt es zu jedem \(p\) nur endlich viele \(\frakp\).
	Wenn \(N(\frakp)\leq c\) dann ist \(p\leq c\) und das sind nur endlich viele Möglichkeiten.
\end{proof}
\begin{Satz}
	Die Klassenzahl \(Cl_K\) ist endlich.
\end{Satz}
\begin{proof}
	\(Cl_K\) wird von allen \(I\subseteq \calO_K\) ungleich 0 erzeugt (Multiplikation der Nenner).
	Nach ?? reicht es zu Zeigen, dass \(I\) modulo \(P_K\) äquivalent ist zu einem \(I'\) mit \(N(I')\leq A=(\frac 2 \pi)^s\sqrt{|d_K|}\).
	Wähle \(b\in K\) sodass \(bI^{-1}=J\subseteq\calO_K\) und 
	\(c_\tau\) wie in ??? sodass \(\prod c_\tau=N(J)A\).
	Dann existiert ein \(a\in J\setminus\{0\}\) sodass
	\(\prod |\tau(a)|\leq \prod c_\tau=N(J)A\).
	Es ist
	\[\prod |\tau(a)|=|N_{K/\QQ}(a)|=|N(a\calO_K)|\]
	Es ist \(a\calO_K\subseteq J=bI^{-1}\subseteq \calO_K\).
	Sei \(I'=\frac a b I\subseteq \calO_K\).
	Dann ist \[N(I')=\dfrac{N(a\calO_K)}{N(b\calO_K)}N(I)\leq \dfrac{N(b)}{N(b)}N(I^{-1})N(I)A=A\]
\end{proof}
\begin{Def}
	Die Klassenzahl von \(K\) ist \(h_K=|Cl_K|\)
\end{Def}
