\chapter{Minkowski Theorie}
\begin{Def} Sei \(V\) ein endlich-dimensionaler \(\RR\)-Vektorraum.
	\begin{enumerate}
		\item Eine Untergruppe \(\Lambda\subseteq V\) ist ein Gitter, wenn jede \(\ZZ\)-lineare unabhängige Menge in 
		\(\Lambda\) in \(V\) \(\RR\)-linear unabhängig ist.
		\item Ein Gitter \(\Lambda\subseteq V\) ist vollständig, wenn \(\Lambda\) \(V\) als \(\RR\)-Vektorraum erzeugt.
	\end{enumerate}
	
\end{Def}
\begin{Bsp}
	\(\ZZ^n\subseteq \RR^n\) ist vollständiges Gitter. \(\ZZ[i]\subseteq\CC\) ist vollständiges Gitter mit Basis \(1,i\). Aber \(\ZZ[\sqrt 2]\subseteq \RR\) ist kein Gitter.
\end{Bsp}
\begin{Bem}
	Sei \(\Lambda\subseteq V\) und \(\alpha\colon\Lambda \otimes_\ZZ\RR\to V,(x,v)\mapsto xv\).
	Es ist \(\Lambda\) ein Gitter \(\iff \alpha\) injektiv ist.
	\(Lambda\) ist vollständiges Gitter \(\iff \alpha\). bijektiv.
	
	Wenn \(\Lambda\subseteq V\) vollständiges Gitter ist, dann gibt eine Wahl von Basis von \(\Lambda\) und 
	% https://tikzcd.yichuanshen.de/#N4Igdg9gJgpgziAXAbVABwnAlgFyxMJZABgBpiBdUkANwEMAbAVxiRAB12AZOgWwCModEAF9S6TLnyEUARnJVajFmwBqo8SAzY8BImVmL6zVog7sAWhYB6hMRJ3Si8w9WMqznAEpfboxTBQAObwRKAAZgBOELxIAEzUOBBIAMyJdFgMbAAWEBAA1hoR0bGIZCBJ8dQMWGCmIFAQTPwMrNTZMHRQbJB1IOmZPQRtSiZsnNE4dDgw-BAAHsAAnMQiwJwAFJzYvJwAlCJFIFExSPIVyYhpIB1dQ33VtfWNza39FRlZZr0j7vUTECmMzmixWa022ywu3YByOJ1K5UqiHOU0GZlyBX8IiAA
	\begin{tikzfigure}
		\Lambda \arrow[d, "\rotatebox{90}{\(\sim\)}", no head, equal] \arrow[r, hook] & V \arrow[d, "\rotatebox{90}{\(\sim\)}", no head, equal] \\
		\ZZ^n \arrow[r, hook]                                                              & \RR^n                                                       
	\end{tikzfigure}
\end{Bem}
\begin{Satz} Sei \(V\) ein \(\RR\)-Vektorraum.
	\begin{enumerate}
		\item Eine Untergruppe \(\Lambda\subseteq V\) ist Gitter \(iff \Lambda\) ist diskret in \(V\).
		\item Ein Gitter \(\Lambda\subseteq V\) ist vollständig \(iff V/\Lambda\) ist kompakt.
	\end{enumerate}
\end{Satz}
\begin{proof}
	\begin{enumerate}
		\item Wenn \(Lambda\) Gitter dann ist es diskret da \(\ZZ^n\subseteq \RR^n\) diskret ist.
		Sei also \(\Lambda\subseteq V\) diskret.
		Ersetze \(V\) durch \(RR\cdot \Lambda\) und wähle \(RR\)-Basis \(v_1,\dots,v_n\in \Lambda\).
		Sei \(Lambda_0=\langle v_1,\dots,v_n\rangle_\ZZ\).
		Dann ist \(\Lambda_0\) Gitter in \(V\).
		Es ist \(\Lambda/\Lambda_0\subseteq V/\Lambda_0\) wobei die Quotiententopologie gleich der Teilraumtopologie ist nach ???.
		Da \(\Lambda\subseteq V\) diskret, ist \(\Lambda/\Lambda_0\subseteq V/\Lambda_0\) diskret.
		Da \(\Lambda_0\subseteq\) vollständiges Gitter ist, ist \(V/\Lambda_0\) kompakt nach (2).
		Also ist \(\Lambda/Lambda_0\) auch kompakt da \(\Lambda/\Lambda_0\subseteq V/\Lambda_0\) abgeschlossen denn \(\Lambda\subseteq V\) ist diskrete Untergruppe einer Hausdorff Gruppe.
		Also ist zusammen \(\Lambda/\Lambda_0\) kompakt und diskret also endlich.
		Also ist \(\Lambda\) endlich erzeugte abelsche Gruppe.
		Nach ??? Struktursatz ist \(\Lambda\cong ZZ^r\). Da \(v_1,\dots,v_n\) linear unabhängig sind, ist \(\Lambda_0=\Lambda\) und somit \(\Lambda\) Gitter.
		\item Wenn \(\Lambda\) vollständig, dann ist \(\RR^n/\ZZ^n=(S^1)^n\) kompakt.
		Wenn \(\Lambda\subseteq V\) Gitter sodass \(V/\Lambda\) kompakt ist, dann sei \(V_0=\RR\Lambda\).
		Es ist \(V=V_0\oplus V_1\) nach ergänzen einer Basis von \(V_0\).
		Also ist \(V/\Lambda\cong V_0/\Lambda \oplus V_1=\text{kompakt}\times \RR^m\) kompakt. Also ist \(m=0\) und damit \(V_0=V\).
	\end{enumerate}
\end{proof}
\begin{Def}
	Sei \(\Lambda\subseteq V\) ein Gitter. Eine Grundmasche ist eine Menge der Form
	\(\{\sum_{i=1}^na_iv_i\mid a_i\in [0,1]\}\) für eine Basis \(v_1,\dots,v_n\) von \(\Lambda\).
\end{Def}
Sei ab jetzt \(V\) ein endlich-dimensionaler eukldischer Vektorraum. Wähle Isometrie \(\varphi\colon (V,\langle , \rangle)\to (\RR^n,\cdot)\)
und definiere ein Maß \(\mu\) auf \(V\) durch \(\mu(A)=\lambda(\phi(A))\) für Messbare Mengen \(A\) wobei \(\lambda\) das Lebesgueq Maß auf \(\RR^n\) ist. Das ist wohldefiniert denn für eine andere Isometrie sind die Orthonormalbasen verbunden durch eine orthonormale Matrix und das Lebesgue Maß ist invariant unter diesen Transformationen nach ???.
\begin{Lemma}
	Sei \(V\) ein Euklidischer Vektorraum und \(v_1,\dots,v_n\in V\) und sei \(\Gamma\) das Parallelopid das aufgespannt wird von \(v_1,\dots,v_n\).
	Dann ist \(\mu(\Gamma)=|\det(A)|\) wobei \(A\) die Darstellungsmatrix ist der \(v_1\) bezüglich einer Orthonormalbasis.
\end{Lemma}
\begin{proof}
	Eine Orthonormalbasis gibt Isometrie zu \(\RR^n\) mit Standardskalarprodukt. Somit ist ohne Einschränkung \(V=\RR^n\) und die Orthonormalbasis die Standardbasis.
	Sei \(A\) die Abbildungsmatrix. Nach QR-Zerlegung ???
	ist \(A=QR\) wobei \(Q\) Orthogonal ist und \(R\) obere Dreiecksmatrix. Wegen Invarianz des Lebesgue Maß von Orthogonalen Transformationen ist ohne Einschränkung \(A\)
	obere Dreiecksmatrix. Dann ist sowohl das Volumen als auch der Betrag der Determinante gleich dem Produkt der Diagoaleinträge.
\end{proof}
\begin{Lemma}
	Seien \(M\subseteq N\) zwei vollständige Gitter in \(\RR^n\) und sei \(A\in M_n(\ZZ)\) die Übergangsmatrix von einer Basis von \(N\) zu einer Basis von \(M\).
	Es gilt
	\begin{enumerate}
		\item \(\Vol(M)=|\det(A)|\Vol(N)\)
		\item \(N/M=|det(A)|\)
	\end{enumerate}
\end{Lemma}
\begin{proof}
	Sei \(v_1,\dots,v_n\) eine Basis von \(N\) und \(w_1,\dots,w_n\) eine Basis von \(M\) und \(A\) die Übergangsmatrix.
	Dann ist \((w_1,\dots,w_n)=(v_1,\dots,v_n)\cdot A\) und somit
	\(\Vol(M)=|\det(w_1,\dots,w_n)|=|\det(A)|\cdot |\det(v_1,\dots,v_n)|=|\det(A)|\Vol(N)\).
	Die zweite Aussage kann man rein algebraisch Zeigen. Oder aber so:
	Sei \(x\in N\). Es gibt Darstellung \(x=\sum_i \beta_iw_i\) mit \(\beta_i\in\RR\). Sei \(y=\sum_i\lfloor \beta_i \rfloor w_i\in M\).
	Dann ist \(x-y\in \Gamma_M\cap N\).
	Wenn \(x_1-y_1=x_2-y_2\in \Gamma_M\cap N\) für \(x_1,x_2\in N\) und \(y_1,y_2\in M\) dann ist 
	sind die Koeffizienten von \(x_1-x_2\) ganze Zahlen aber auch im Betrag \(<1\) also 0.
	Das heißt es gibt Bijektion \(N/M\to N\cap \Gamma_M\).
	Sei \(\Gamma_M\cap N=\{s_1,\dots,s_k\}\). 
	Es ist \(s_i+\Gamma_N\) disjunkt für verschiedene \(i\)
	Es ist \(\Gamma_M=\bigcup_i s_i+\Gamma_N\) 
	
\end{proof}
\begin{Def}
	Das Volumen \(\Vol(\Lambda)\) von Gitter \(\Lambda\) ist definiert als \(\mu(\Gamma)\) wobei \(\Gamma\) eine Grundmasche ist.Das ist wohldefiniert, denn zwei Basen von \(Lambda\) unterscheiden sich um ein Element von \(GL_n(\ZZ)\) und das hat Determinante \(\\pm 1\).
	Wirkung auf Volumen geschiet durch Betrag der Determinante.
	
\end{Def}
\begin{Bem}
	Es ist \(A^t\cdot A=\langle v_i,v_j\rangle_{i,j}=B\)
	und somit \(\det(B)=\det(A)^2\).
	Also ist \(\Vol(A)=\sqrt{\det(B)}\)
\end{Bem}
\section{Miskowskysche Gitterpunktsatz}
\begin{Satz}[Gitterpunktsatz] \label{Satz:Gitterpunktsatz}
	
	Sei \(V\) ein Euklidischer Vektorraum und \(\Lambda\subseteq V\) ein vollständiges Gitter und \(X\subseteq V\) eine konvexe, zentralsymmetrische (dh. \(x\in X\implies -x\in X\)) Menge.
	Wenn \[\mu(X)>2^n\Vol(\Lambda)\] dann ist \(X\cap \Lambda\neq\{0\}\)
\end{Satz}
\begin{proof}
	Es reicht zu zeigen, dass es \(\gamma_1,\gamma_2\in \Lambda\) gibt mit \(\gamma_1\neq\gamma_2\) sodass
	\[(\frac 1 2 X+\gamma_1)\cap(\frac 1 2 X+\gamma_2)\neq \emptyset\] denn wenn
	\[\frac 1 2 x_1+\gamma_1=\frac 1 2 x_2+\gamma_2\] mit \(x_1,x_2\in X\) dann ist 
	\[\gamma=\gamma_1-\gamma_2=\frac 1 2 x_2-\frac 1 2 x_1\in X\cap \Lambda\]
	Also angenommen \(\frac 1 2 X+\gamma\) für \(\gamma\in\Lambda\) sind paarweise disjunkt.
	Dann sind auch \(\Gamma\cap (\frac 1 2 X+\gamma)\) paarweise disjunkt wobei \(\Gamma\) eine Grundmasche ist von \(\Lambda\)
	Dann ist 
	\begin{align*}
		\Vol(\Lambda)&\geq \sum_{\gamma\in \Lambda}\Vol(\Gamma\cap() \frac 1 2 +\gamma))\\
		&=\sum_{\gamma\in\Lambda}\Vol((\Gamma-\gamma)\cap \frac 1 2 X)\\
		&= \Vol(\frac 1 2 X)\\
		&= \frac{1}{2^n}\Vol(X)
	\end{align*} was ein Widerspruch ist.
\end{proof}
\begin{Kor}
	Jede natürliche Zahl lässt sich schreiben als Summe von vier Quadraten.
\end{Kor}
\begin{proof}
	Es ist 
	\begin{align*}
	(a^2+b^2+c^2+d^2)(A^2+B^2+C^2+D^2)&=(aA-bB-cC-dD)^2+
	(aB+bA+cD-dC)^2\\&+(aC-bD+cA+dB)^2+(aD+bC-cB+dA)^2
	\end{align*}
	Also reicht es die Aussage für eine Primzahl \(p\) zu zeigen.
	Es ist \(2=1^2+1^2+0^2+0^2\) also ist ohne Einschränkung \(p\geq 3\).
	Es ist \[|\{m^2 \mod p\mid m=0,\dots p-1\}|=\frac{p+1}{2}\] und
	\[|\{-1-n^2 \mod p\mid n=0,\dots p-1\}|=\frac{p+1}{2}\]
	Also gibt es \(m,n\in\ZZ\) sodass \(m^2+n^2+1\equiv 0 \mod p\).
	Sei \[\Lambda=\{(a,b,c,d)\in\ZZ^4\mid c\equiv ma+nb,\, d\equiv mb-na \mod p\}\] Gitter.
	Dann ist \(p\ZZ^4\subseteq \Lambda\subseteq ZZ^4\)
	und \(\Lambda/p\ZZ^4\subseteq \FF_p^4\) Unterraum der Dimension 2.
	Also ist der Index \([\ZZ^4\colon \Lambda]=p^2\)
	und somit ist \(\Vol(\Lambda)=p^2\).
	Sei \(T=\bar{B_r(0)}\) und \(r\) so gewählt, dass \(2p>r^2>\frac{4\sqrt{2}}{\pi}p\).
	Dann ist
	\[\Vol(T)=\frac{\pi^2r^4}{2}>16p^2=\Vol(\Lambda)\]
	Nach \nameref{Satz:Gitterpunktsatz} gibt es \((a,b,c,d)\in \Lambda\setminus\{0\}\cap T\) das heißt
	\(a^2+b^2+c^2+d^2\equiv a^2(1+m^2+n^2)+b^2(1+n^2+m^2)\equiv 0 \mod p\).
	Da \((a,b,c,d)\in T\) ist \(a^2+b^2+ c^2+ d^2<2p\).
	Also ist \(a^2+b^2+c^2+d^2=p\).
\end{proof}
\begin{Bem}
	Sei \(K\) ein Zahlkörper. \(\calO_K\cong \ZZ^n\) soll ein Gitter werden in einem passenden \(\RR\)-Vektorraum.
	Da jede \(\ZZ\)-Basis von \(\calO_K\) eine \(\QQ\)-Basis ist, ist die natürliche Abbildung
	\(\calO_K\otimes_\ZZ\QQ\to K\) bijektiv. 
	Betrachte Diagramm
	% https://tikzcd.yichuanshen.de/#N4Igdg9gJgpgziAXAbVABwnAlgFyxMJZABgBpiBdUkANwEMAbAVxiRAB12AtLkAX1LpMufIRQBGclVqMWbTgEUF-QSAzY8BIgCYp1es1aIO7AEqmVQjaKIBmPTMPz2AYReW1wzWJKlx0gzljTgB5EIB9AGkPdREtCT8A2SMTMKjOCDwAW3hwzh5FZQErOJ9df31k5zTIjOzc-K46rBy4AAJOcxivGxR7Cscg1Ija9kyWhu4m13c+aRgoAHN4IlAAMwAnCCykMhAcCCRJfbosBjYACwgIAGsPTe2j6gOkXROzy+u74pAHncQ3i9EPZ3udjFdbvctv89kCACw-P5PfaHRAAVkR0Nez1RADZMY9gTikAB2An-OHE9HPU5gkAQ76qJHUlFIXE0j7gr5Qwns1mIEkcukM-gUPhAA
	\begin{tikzfigure}
		\ZZ \arrow[r, hook] \arrow[d] & \QQ \arrow[r, hook] \arrow[d]       & \RR \arrow[r, hook] \arrow[d]               & \CC \arrow[d]       \\
		\calO_K \arrow[r, hook]         & \calO_K\otimes_\ZZ\QQ \arrow[r, hook] & \calO_K\otimes_\ZZ\otimes \RR \arrow[r, hook] & \calO_K\otimes_\ZZ\CC
	\end{tikzfigure}
	Sei \(K_\RR=\calO_K\otimes_\ZZ\RR\cong (\calO_K\otimes_\ZZ \QQ)\otimes_\QQ\RR\cong K\otimes_\QQ\RR\) der Minkowski Raum. Es sei \(K_\CC=\calO_K\otimes_\ZZ\CC\).
	Es ist \(\calO_K\) ein vollständiges Gitter in \(K_\RR\) denn die Abbildung 
	\(\calO_K\otimes_\ZZ\RR\to K_\RR, (x,a)\mapsto a\cdot x=(1\otimes a)(x\otimes 1)=x\otimes a\) ist bijektiv.
\end{Bem}
\begin{Lemma}
	Die Komplexe Konjugation \(\tau\colon\CC\to\CC\) induziert einen Homomorphismus 
	\[\tilde\tau=\id\otimes \tau\colon K\otimes_\QQ\CC\toK\otimes_\QQ\CC\].
	Die Invarianten davon sind 
	\[(K\otimes_\QQ\CC)^{\tilde\tau}=\{x\in K\otimes\CC\mid \tilde\tau(x)=x\}=K_\RR\]
\end{Lemma}
\begin{proof}
	Wähle eine \(\QQ\)-Basis von \(K\cong \QQ^n\).
	Dann ist \(K\otimes_\QQ\RR\cong \RR^n\) und \(K_\CC\cong \CC^n\).
	Also ist
	\((K_\CC)^{\tilde\tau}\cong (\CC^n)^{\tilde\tau}=\RR^n\cong K\otimes_\QQ\RR\).
\end{proof}
\begin{Lemma}
	Es gibt Isomorphismus von \(\CC\)-Algebren \(\psi\colon K_\CC\to \prod_{\tau}\CC,\psi(a\otimes b)=(\tau(a)\cdot b)_\tau\) wobei \(\tau\) alle Körperhomomorphismen \(\tau\colon K\to\CC\) durchläuft.
\end{Lemma}
\begin{proof}
	\(K_\CC\) ist endlich-dimensionale \(\CC\)-algebra, also ein Modul endlicher Länge und damit Artinscher Ring.
	Nach \ref{Kor:StruktArtinring} ist \(K_\CC\cong \prod R_i\) mit \((R_i,\frakm_i)\) lokale artinsche Ringe. Es gilt dass
	\(R_i/\frakm_i=\CC\) da \(\CC\) algebraisch abgeschlossen ist.
	Seien \(\tau_1,\dots,\tau_n\) die Homomorphismen \(K\to \CC\). Es ist \(n=[K:\QQ]_S=[K:\QQ]\).
	Also sind auch \(\tilde\tau_1,\dots,\tilde\tau_n\) verschiedene Abbildungen \(K_\CC\to \CC\).
	Jeder Homomorphismus von \(\CC\)-Algebren \(\prod R_i\to \CC\) hat die Form \(\prod R_i\to R_i\to R_i/\frakm_i \cong \CC\) siehe ???.
	Also ist Anzahl \(r\) der Homomorphismen \(K_\CC\to\CC\) gleich Anzahl der Faktoren.
	Da jede Algebra \(R_i\) als Vektorraum mindestens Dimension 1 hat, muss \(n\geq r\) sein. Aber oben haben wir gesehen, dass \(r\geq n\). Also ist \(r=n\) und \(\dim_\CC(R_i)=1\). Somit ist \(R_i\cong \CC\).\\
	Alternativ kann man auch wie folgt argumentieren:
	Sei \(b\in K\) sodass \(K=\QQ[b]\cong \QQ[X]/(f)\).
	Dann ist \(K_\CC\cong \CC[X]/(f)\).
	Wenn \(f=\prod_{i=0}^r(X-b_i)\) mit paarweisen verschiedenen Nullstellen in \(\CC\) dann ist 
	\[K_\CC\cong \prod_{i=1}^r\CC[X]/(X-b_i)\cong \prod\CC_i\]
	Es ist \(\{b_1,\dots,b_n\}=\{\tau(b)\colon \tau\colon K\to \CC\}\).
	Sei \(\tau_i\colon K\to \CC\) sodass \(\tau(b)=b_i\).
	dann ist die Abbildung gegeben durch
	% https://tikzcd.yichuanshen.de/#N4Igdg9gJgpgziAXAbVABwnAlgFyxMJZABgBpiBdUkANwEMAbAVxiRAGkB9AHW4GE+IAL6l0mXPkIoAjOSq1GLNrzQAnaJ2BYAvNKEA9MLwHDRIDNjwEiAJjnV6zVohDHBIsZclEy0+Y6UXACNeCDwAW3gAAmlTTwlrGVI-B0VnEAAKIM4sAEpNWV4oMLhSMCE483ErKWQ7FIUnNmydXhw6Jhys3OF5GCgAc3giUAAzdXCkMhAcCCRZRsDXbmxJjxBxiEnEBdmkO0X0tRzKze2AZmo9xAAWK7osBjZwujQ4PfWzpDuZucQAVnuj2er3ecyEFCEQA
	\begin{tikzfigure}
		K_\CC \arrow[r, "\sim"]       & \prod_{i=1}^n\CC \arrow[r, "pr_i"]     & \CC           \\
		b\otimes 1 \arrow[r, maps to] & {(b_i)_{1,\dots,n}} \arrow[r, maps to] & b_i=\tau_i(b)
	\end{tikzfigure}
	folglich ist die Abbildung wie im Satz auch für alle Elemente in \(K_\CC\).
\end{proof}
\begin{Bsp}
	Sei \(d>1\) quadratfrei und \(K=\QQ[\sqrt{d}]\cong \QQ[X]/(X^2-d)\)
	Dann ist \(K\otimes_\QQ\RR\cong \RR[X]/(X-\sqrt{d})\times \RR[X]/(X+\sqrt{d)\cong \RR\times\RR\) wobei \(1\otimes a\) auf \(a,a\) geschickt wird und \(\sqrt{d}\) auf \(\sqrt{d},-\sqrt{d}\).
		Wenn \(d\not\equiv 1 \mod 4\) dann ist \(d_K=d\) und \(1,\sqrt{d}\) eine Ganzheitsbasis.
		Es ist \(\Vol(\calO_K)=|\det(\begin{pmatrix} 1 & \sqrt d\\ 1 & -\sqrt d)|=\sqrt{d}=\sqrt{d_K}\)
		Analog wenn \(d\equiv 1 \mod 4\) dann 
		\(\Vol(\calO_K)=\sqrt{(d_K)}\)
\end{Bsp}
