\chapter{Minkowski Theorie}
\begin{Def} Sei \(V\) ein endlich-dimensionaler \(\RR\)-Vektorraum.
	\begin{enumerate}
		\item Eine Untergruppe \(\Lambda\subseteq V\) ist ein Gitter, wenn jede \(\ZZ\)-lineare unabhängige Menge in 
		\(\Lambda\) in \(V\) \(\RR\)-linear unabhängig ist.
		\item Ein Gitter \(\Lambda\subseteq V\) ist vollständig, wenn \(\Lambda\) \(V\) als \(\RR\)-Vektorraum erzeugt.
	\end{enumerate}
	
\end{Def}
\begin{Bsp}
	\(\ZZ^n\subseteq \RR^n\) ist vollständiges Gitter. \(\ZZ[i]\subseteq\CC\) ist vollständiges Gitter mit Basis \(1,i\). Aber \(\ZZ[\sqrt 2]\subseteq \RR\) ist kein Gitter.
\end{Bsp}
\begin{Bem}
	Sei \(\Lambda\subseteq V\) und \(\alpha\colon\Lambda \otimes_\ZZ\RR\to V,(x,v)\mapsto xv\).
	Es ist \(\Lambda\) ein Gitter \(\iff \alpha\) injektiv ist.
	\(Lambda\) ist vollständiges Gitter \(\iff \alpha\). bijektiv.
	
	Wenn \(\Lambda\subseteq V\) vollständiges Gitter ist, dann gibt eine Wahl von Basis von \(\Lambda\) und 
	% https://tikzcd.yichuanshen.de/#N4Igdg9gJgpgziAXAbVABwnAlgFyxMJZABgBpiBdUkANwEMAbAVxiRAB12AZOgWwCModEAF9S6TLnyEUARnJVajFmwBqo8SAzY8BImVmL6zVog7sAWhYB6hMRJ3Si8w9WMqznAEpfboxTBQAObwRKAAZgBOELxIAEzUOBBIAMyJdFgMbAAWEBAA1hoR0bGIZCBJ8dQMWGCmIFAQTPwMrNTZMHRQbJB1IOmZPQRtSiZsnNE4dDgw-BAAHsAAnMQiwJwAFJzYvJwAlCJFIFExSPIVyYhpIB1dQ33VtfWNza39FRlZZr0j7vUTECmMzmixWa022ywu3YByOJ1K5UqiHOU0GZlyBX8IiAA
	\begin{tikzfigure}
		\Lambda \arrow[d, "\rotatebox{90}{\(\sim\)}", no head, equal] \arrow[r, hook] & V \arrow[d, "\rotatebox{90}{\(\sim\)}", no head, equal] \\
		\ZZ^n \arrow[r, hook]                                                              & \RR^n                                                       
	\end{tikzfigure}
\end{Bem}
\begin{Satz} Sei \(V\) ein \(\RR\)-Vektorraum.
	\begin{enumerate}
		\item Eine Untergruppe \(\Lambda\subseteq V\) ist Gitter \(iff \Lambda\) ist diskret in \(V\).
		\item Ein Gitter \(\Lambda\subseteq V\) ist vollständig \(iff V/\Lambda\) ist kompakt.
	\end{enumerate}
\end{Satz}
\begin{proof}
	\begin{enumerate}
		\item Wenn \(Lambda\) Gitter dann ist es diskret da \(\ZZ^n\subseteq \RR^n\) diskret ist.
		Sei also \(\Lambda\subseteq V\) diskret.
		Ersetze \(V\) durch \(RR\cdot \Lambda\) und wähle \(RR\)-Basis \(v_1,\dots,v_n\in \Lambda\).
		Sei \(Lambda_0=\langle v_1,\dots,v_n\rangle_\ZZ\).
		Dann ist \(\Lambda_0\) Gitter in \(V\).
		Es ist \(\Lambda/\Lambda_0\subseteq V/\Lambda_0\) wobei die Quotiententopologie gleich der Teilraumtopologie ist nach ???.
		Da \(\Lambda\subseteq V\) diskret, ist \(\Lambda/\Lambda_0\subseteq V/\Lambda_0\) diskret.
		Da \(\Lambda_0\subseteq\) vollständiges Gitter ist, ist \(V/\Lambda_0\) kompakt nach (2).
		Also ist \(\Lambda/Lambda_0\) auch kompakt da \(\Lambda/\Lambda_0\subseteq V/\Lambda_0\) abgeschlossen denn \(\Lambda\subseteq V\) ist diskrete Untergruppe einer Hausdorff Gruppe.
		Also ist zusammen \(\Lambda/\Lambda_0\) kompakt und diskret also endlich.
		Also ist \(\Lambda\) endlich erzeugte abelsche Gruppe.
		Nach ??? Struktursatz ist \(\Lambda\cong ZZ^r\). Da \(v_1,\dots,v_n\) linear unabhängig sind, ist \(\Lambda_0=\Lambda\) und somit \(\Lambda\) Gitter.
		\item Wenn \(\Lambda\) vollständig, dann ist \(\RR^n/\ZZ^n=(S^1)^n\) kompakt.
		Wenn \(\Lambda\subseteq V\) Gitter sodass \(V/\Lambda\) kompakt ist, dann sei \(V_0=\RR\Lambda\).
		Es ist \(V=V_0\oplus V_1\) nach ergänzen einer Basis von \(V_0\).
		Also ist \(V/\Lambda\cong V_0/\Lambda \oplus V_1=\text{kompakt}\times \RR^m\) kompakt. Also ist \(m=0\) und damit \(V_0=V\).
	\end{enumerate}
\end{proof}
\begin{Def}
	Sei \(\Lambda\subseteq V\) ein Gitter. Eine Grundmasche ist eine Menge der Form
	\(\{\sum_{i=1}^na_iv_i\mid a_i\in [0,1]\}\) für eine Basis \(v_1,\dots,v_n\) von \(\Lambda\).
\end{Def}
Sei ab jetzt \(V\) ein eukldischer Vektorraum, dh. \(V\) hat ein Skalarprodukt. Dann trägt \(V\) ein Maß \(\mu\).
\begin{Def}
	Das Volumen \(\Vol(\Lambda)\) von Gitter \(\Lambda\) ist definiert als \(\mu(\Gamma)\) wobei \(\Gamma\) eine Grundmasche ist.Das ist wohldefiniert, denn zwei Basen von \(Lambda\) unterscheiden sich um ein Element von \(GL_n(\ZZ)\) und das hat Determinante \(\\pm 1\).
	Wirkung auf Volumen geschiet durch Betrag der Determinante.
\end{Def}
