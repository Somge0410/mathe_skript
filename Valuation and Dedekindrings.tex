\chapter{Bewertungsringe und Dedekindringe}
\section{Bewertungsringe}
\begin{Def} Sei \(A\) Integritätsbereich und \(K=\Quot(A)\).
	\(A\) heißt Bewertungsring, falls für alle \(x\in K^*\) gilt dass \(x\in A\) oder \(x^{-1}\in A\).
	Sei \(\Gamma=K^*/A^*\).  Das ist abelsche Gruppe mit \([a]+[b]=[a+b]\). Definiere partielle Ordnung durch 
	\(p\geq 0 \iff p=[a]\) für ein \(a\in A\), also \([b]\geq [a] \iff \frac b a \in A\).
	
\end{Def}
\begin{Def} Sei \(K\) ein Körper und \(\Gamma\) eine total geordnete abelsche Gruppe. Eine Bewertung von \(K\) ist eine 
	surjektive Abbildung \(\nu\colon K^*\to \Gamma\) sodass \(\nu(xy)=\nu(x)+\nu(y)\) und \(\nu(x+y)\geq \mid\{\nu(x),\nu(y)\}\) ist für alle
	\(x,y\in K^*\).
	Es gilt die Konvention  \(\nu(0)=\infty\).
	\(\nu\) heißt diskrete Bewertung falls \(\Gamma\cong\ZZ\) ist.
\end{Def}
\begin{Satz} Sei \(A\) ein Integritätsbereich und \(K=\Quot(A)\).
	Dann ist äquivalent:
	\begin{enumerate}
		\item \(\forall x\in K^*\colon x\in A \text{ or } x^{-1}\in A\).
		\item Für zwei Ideale \(I,J\subseteq A\) gilt \(I\subseteq J\) oder \(J\subseteq I\).
		\item Das gleiche wie 2. nur für Hauptideale.
		\item Es gibt eine total geordnete abelsche Gruppe \(\Gamma\) (Bewertungsgruppe genannt)
		und eine Bewertung \(\nu\colon K^*\to \Gamma\) sodass \(A=\{x\in K^*\mid \nu(x)\geq 0\}\cup\{0\}\).
		\item \(A\) ist ein lokaler Bézout Ring.
	\end{enumerate}
	In diesem Fall heißt \(A\) diskreter Bewertungsring mit Bewertungsgruppe \(\Gamma\) und Bewertung \(\nu\).
	Ist \(\nu\) zusätzlich eine diskrete Bewertung, so heißt \(A\) diskreter Bewertungsring.
\end{Satz}
\begin{proof}
	Zeige 1 nach 2: Sei \(x\in I\) mit \(x\not\in J\) und sei \(y\in J\). Sei ohne Einschränkung \(y\neq 0\).
	Aus \(\frac x y\in A\) folgt \(x\in J\), also ist \(\frac y x\in A\) und damit \(y\in I\).
	Also ist \(J\subseteq I\).\\
	2. nach 3. ist klar.
	Gelte 3. Sei \(\Gamma=K^*/A^*\) mit \([a]+[b]=[ab]\). Das ist abelsche Gruppe. Definiere partielle Ordnung durch
	\[[b]\geq [a] \iff \frac b a \in A\] und sei \(nu\colon K^*\to \Gamma\) die Projektion. 
	Seien \(x=\frac{a}{b},y=\frac{c}{d}\in K^*\) mit \(a,b,c,d\in A\setminus\{0\}\). Dann ist \((ad)\subseteq (bc)\) oder \((bc)\subseteq (ad)\).
	das heißt \(\frac x y \in A\) oder \(\frac y x\in A\) also ist die Ordnung total.
	Wenn 3. gilt dann ist \(x\in K^*\) eine Einheit von \(A\) genau dann wenn \(nu(x)=0\) ist da \(\nu(x^{-1})=-\nu(x)\).
	Also ist \(\frakm=\{x\in A\mid \nu(x)>0\}\) das maximale Ideal von \(A\).
	Sei \(I\subseteq A\) ein endlich erzeugtes Ideal und seien \(x_1,\dots,x_n\) Erzeuger sodass \(\nu(x_1)<\dots<\nu(x_n)\). Da \(\nu\) 
	surjektiv ist, gibt es zu \(k>1\) ein \(t\in K^*\) sodass \(\nu(t)=\nu(x_k)-\nu(x_1)>0\). Dann ist \(t\in A\) und \(\nu(tx_1)=\nu(x_k)\) woraus folgt
	dass \(x_k=tux_1\) ist für eine Einheit \(u\in A\). Also ist \(I=(x_1)\) und somit \(A\) ein Bézout Ring.
	Gelte 4 und sei \(x=\frac a b \in K^*\) mit \(a,b\in A^*\). Es ist \(a,b)=(c)\) nach 4 und somit \(c=ua+vb\) und \(a=wc\) und \(b=zc\).
	Dann ist \(c(1-uw-vz)=0\). Da \(A\) lokal ist, ist \(1-uw-vz\) im maximalen Ideal denn sonst wäre es Einheit und damit \(c=0\).
	Also ist \(w\) oder \(z\) eine Einheit und \(a,b)=(a)\) oder \(a,b)=(b)\) und somit \(x\in A\) oder \(x^{-1}\in A\).
\end{proof}
\begin{Bem} Ein Bewertungsring \(A\) ist lokal mit maximalem Ideal \(\frakm=\{x\in K\mid \nu(x)> 0\}\).
	
\end{Bem}
\begin{Lemma}\label{Lem:NoethIntIntersect}
	Sei \(A\) ein noetherscher Integritätsbereich und \(t\in A\) keine Einheit. Dann ist \[\bigcap_{n=1}^\infty (t^n)=0.\]
	
\end{Lemma}
\begin{proof}
	Angenommen \(0\neq x\in \bigcap _{n=1}^\infty (t^n)=0\).
	Dann ist \(x=tx_1=t^2x_2=\dots=t^nx_n=\dots\) und somit \((x)\subsetneq (x_1)\subsetneq \dots\). Da \(A\) noethersch ist,
	muss die Kette stoppen.
\end{proof}
\begin{Satz}\label{Satz:DBRIntersect} Sei \(A\) ein Integritätsbereich. Es ist äquivalent:
	\begin{enumerate}
		\item \(A\) ist ein diskreter Bewertungsbereich.
		\item \(A\) ist lokal mit maximalem Ideal \(\frakm=(t)\) für \(t\neq 0\) und \(\bigcap_{n=1}^\infty (t^n)=0\)
	\end{enumerate}
	In dem Fall hat jedes Element \(0\neq x\in A\) die Darstellung \(x=t^nu\) mit \(n\geq 0\) und \(u\) eine Einheit in \(A\) und jedes Ideal
	\(I\neq 0\) ist von der Form \(I=(t^n\). Insbesondere ist \(A\) noethersch.
\end{Satz}
\begin{proof}
	Gelte 1. Es gibt \(t\in A\) mit \(\nu(t)=1\). Wenn also \(x\in\frakm\) dann ist \(\nu(x)\geq \nu(t^n)\geq 1\) und also \(x=t^nu\) für
	eine Einheit \(u\). Somit ist \(\frakm=(t)\). Wenn \(x=t^nu\) ist für eine Einheit \(u\) dann ist \(\nu(x)=n\) also ist \(\bigcap_{n=1}^\infty(t^n)=0\).
	Gelte 2. Dann gibt es für jedes \(x\in A\) ein \(n\) sodass \(x\in (t^n)\setminus(t^{n+1}\). Also hat \(x\) Darstellung \(x=t^n\cdot u\).
	Definiere \(\nu(x)=n\) und \(\nu(\frac x y)=\nu(x)-\nu(y)\). Das ist eine diskrete Bewertung sodass \(A\) diskreter Bewertungsring ist.
	Sei \(I\) ein Ideal und \(n=\min\{k\mid t^n\in I\}\). Dann ist \(I=(t^n)\).
\end{proof}


\begin{Satz} Sei \(A\) ein Bewertungsring. Dann ist äquivalent:
	\begin{enumerate}
		\item \(A\) ist noethersch
		\item \(A\) ist Hauptidealring.
	\end{enumerate}
	Wenn zusätzlich \(A\) kein Körper ist, ist auch äquivalent:
	\begin{enumerate}
		\item[(3)] \(A\) ist diskreter Bewertungsring.
	\end{enumerate}
	\begin{proof}
		Da Jedes Ideal in einem noetherschen Ring endlich erzeugt ist, ist jedes Ideal ein Hauptideal.
		Ein Hauptidealring ist auch noethersch.
		Zeige also das 3. aus den Bedingen folgt.
		Wenn \(A\) noethersch ist, dann folgt mit \cref{Lem:NoethIntIntersect}
		dass \(\bigcap_{n=1}^\infty (t^n)=0\) für \((t)=\frakm\). Der Rest ist Satz \cref{Satz:DBRIntersect}.
	\end{proof}
\end{Satz}
\begin{Satz}\label{Satz:DBRNormNoeth1dim} Sei \(A\) Integritätsbereich. \(A\) ist diskreter Bewertungsring genau dann, wenn \(A\) noethersch,
	normal und \(\Spec(A)=\{0,\frakm\}\) ist wobei \(\frakm\neq 0\).
	
\end{Satz}
\begin{proof}
	Nach \cref{Satz:DBRIntersect} sind diskrete Bewertungsringe noethersch und faktoriell, also normal (\cref{Bem:FaktNormal}) und \(\Spec(0,\frakm)\) klar.
	Sei also \(A\) gegeben wie im Satz.
	Wir zeigen dass \(\frakm\) ein Hauptideal ist. Der Rest folgt dann aus \cref{Lem:NoethIntIntersect} und \cref{Satz:DBRIntersect}.
	Sei \(I\subseteq A\) ein Ideal. Das ist endlich erzeugter \(A\)-Modul. Nach \nameref{Lem:Nakayama}
	gilt \(\frakm I=I\implies I=0\). Also ist \(\frakm^2\neq \frakm\).
	Sei also \(x\in \frakm\setminus \frakm^2\).
	Sei \(M=\frakm/(x)\). Wenn \(M\neq 0\) dann ist \(\Ass(M)\neq 0\) nach \cref{Satz:NoetASSNonZero}.
	Das heißt es gibt esin \(0\neq y\in \frakm\setminus (x)\) sodass \(my\subseteq (x)\).
	Dann ist \(\frak x y\in A\) also \(\frac y x\not\in A\) aber \(\frac y x\frakm\subseteq A\) ein Ideal.
	Angenommen \(\frac y x\frakm A=A\). Dann ist \(\frac{yy'}{x}=1\) für ein \(y'\in \frakm\) also wäre
	\(x\in\frakm^2\) was ein Widerspruch ist. Also ist \(\frac y x\frakm \subseteq \frakm\).
	Multiplikation mit \(\frak x y\) ist Lineare Abbildung \(\varphi\colon\frakm\to\frakm\) zwischen endlichen
	\(A\)-Moduln. Wähle also Surjektion \(\pi\colon R^n\to M, e_i\mapsto x_i\) und sei \(B=(b_{ij})\) die Matrix mit
	\(\varphi(x_i)=\sum b_{ij}x_j\). Nach Cayley Hamilton ??? gibt es Polynom \(P=X^n+a_{n-1}X^n+\dots+a_0\) mit 
	\(P(B)=0\) also ist \(P(\varphi)=0\).  Für \(0\neq z\in\frakm\) ist also
	\[((\frac x y)^n+\dots+a_0)z=0\] Also ist \(\frac x y\) integral. Da \(A\) normal ist, ist \(\frac x y\in A\) was Ein Widerspruch ist.
	Also ist \(\frakm=(x)\).
\end{proof}
\begin{Lemma}
	Lokalisierungen von Bewertungsringen sind wieder Bewertungsringe. Quotienten mit Primidealen von Bewertungsringen
	sind wieder Bewertungsringe.
\end{Lemma}
\begin{proof}
	Klar wegen Beschreibung durch Ordnung der Ideale.
\end{proof}
\begin{Lemma}
	Sei \(R\) ein Bewertungsring mit maximalem Ideal \(\frakm\) und sei \(A\subseteq R/\frakm\) ein Bewertungsring
	mit Quotientenkörper \(R/\frakm\). Sei \(R\to R/\frakm\) die Projektion.
	Dann ist \(\pi^(-1)(A)\) Bewertungsring. 
\end{Lemma}
\begin{proof}
	
	Wenn \(R\) Körper ist, ist alles klar. Sei \(R\) kein Körper. Wähle \(t\in \frakm\) ungleich \(0\).
	Dann ist für alle \(a\in R\) \[a=\frac{ta}{t}\] und \(ta\in R'=\pi^{-1}(A)\).
	Also ist \(\Quot(R)=\Quot(R')\).
	Sei \(x\in\Quot(R')\) nicht \(0\). Es ist \(x\in R\) oder \(x^{-1}\in R\).
	Wenn einer der beiden in \(\frakm\) dann ist einer der beiden in \(R'\).
	Sei also \(x\) in \(R\) aber nicht in \(\frakm\). Dann \(\pi(x)\in A\) oder \(\pi(x)^{-1}\in A\).
	Also \(x\in R'\) oder \(x^{-1}\in R'\).
\end{proof}

\begin{Bsp}
	Sei \(R\) ein Ring und \(R[[X]]\) der Ring der formalen Potenzreihen. Ein Element 
	\(f=\sum_{i=0}^\infty a_iX^i\in R[[X]]\) ist eine Einheit, genau dann wenn \(a_0\) eine Einheit ist. Wenn
	\(R=K\) ein Körper ist, dann ist \(K[[x]]\) ein Bewertungsring.
\end{Bsp}
\begin{proof}
	Wenn \(f\) Einheit, dann gibt es \(g=b_0+b_1X+\dots\) sodass 
	\(1=gf=a_0b_0\) also ist \(a_0\) eine Einheit.
	Wenn \(a_0\) Einheit ist, dann ist \(a_o^{-1}f=1+\dots\) also ohne Einschränkung \(a_0=1\).
	Setze \(b_0=1\) und rekursiv \(b_n=-\sum_{i=1}^na_ib_{n-1}\). Dann ist 
	\[f\cdot \sum_{i=0}^\infty b_iX^i=1\].
	Definiere \(\nu(f)=\min\{n\in\NN\mid a_n\neq 0\}\) für \(f\neq 0\).
	und für \(\frac f g \in \Quot(K[[X]])\) setze \(\nu(\frac f g)=\nu(f)-\nu(g)\).
	dann ist \(\nu\) eine diskrete Bewertung und \(K[[X]]=\{ y \mid \nu(y)\geq 0\}\).
\end{proof}
\begin{Bsp}
	Das letzte Lemma ist erfüllt für \(R=\QQ[[x]]\) und \(\frakm=(x)\) und \(A=\ZZ_p\)
	In dem Fall ist \(\pi^{-1}A=\{\sum_{i=0}^\infty a_iX^i\mid a_0\in\ZZ_p\}\).
\end{Bsp}




\section{Dedekindringe}
\begin{Def} Ein Dedekindring ist ein normaler, neotherscher Integritätsring der Dimension  \(leq 1\).
	
	
\end{Def}
\begin{Bsp} Jeder Hauptidealring ist Dedekindring.
	
\end{Bsp}
\begin{Satz} Sei \(R\) ein Ring- es ist äquivalent:
	\begin{enumerate}
		\item \(R\) ist Dedekindring
		\item Jedes Ideal \((0)\neq I\subseteq I\) kann geschrieben werden als Produkt \(I=\frakp_1\dots \frakp_r\) von 
		bis auf Reihenfolge eindeutigen Primidealen \(\frakp_i\neq 0\).
		\item \(R\) ist noethersch und für jedes maximale Ideal \(\frakm\neq 0\) ist \(R_\frakm\) ein diskreter Bewertungsring.
		
	\end{enumerate}
	\begin{Lemma}
		Sei \(R\) ein Ring und \(I,J\) Ideale ungleich \(0\) sodass \(IJ=(f)\) für ein Nichtnullteiler \(f\in A\).
		Dann sind \(I\) und \(J\) endlich erzeugt und endlich lokal frei vom Rang 1 als \(A\)-Modul.
	\end{Lemma}
	\begin{proof}
		Endlich lokal frei impliziert endlich erzeugt nach ???.
		Sei also \(f=\sum_{i=1}^nx_iy_i\) mit \(x_i\in I\) und \(y_i\in J\) und \(x_iy_i=a_if\) für ein \(a_i\in A\).
		Da \(f\) Nichtnulleiler ist \(\sum a_i=1\). Es reicht also, dass \(I_{a_i},J_{a_i}\) frei von Rank 1 über \(A_{a_i}\).
		Ersetze also \(A\) durch \(A_{a_i}\). Dann ist \(f=xy\) für \(x\in I\) und \(y\in J\).
		Wenn \(x'\in I\) dann ist \(x'y=af=axy\) für \(a\in A\). Da \(y\) Nichtnullteiler ist \(x'=ax\) und \(I=(x)\). Analog ist
		\(J=(y)\).
	\end{proof}
	\begin{Lemma}
		Sei \(R\) ein Ring.
		\begin{enumerate}
			\item Jedes Ideal \(I\subseteq R\) mit der Eigenschaft maximal unter nicht-endlich erzeugten Idealen ist ein ein Primideal.
			\item Wenn jedes Primideal von \(R\) endlich erzeugt ist, dann ist jedes Ideal endlich erzeugt.
		\end{enumerate}
	\end{Lemma}
	\begin{proof}
		Sei \(I\) maximal mit der Eigenschaft. Angenommen \(ab\in I\) aber \(a\not\in I\) und \(b\not\in I\).
		Dann ist \((I,a)\neq I\) und \(b\in (I:a)\) sodass \(I\neq (I:a)\). Also enthalten \(I,a\) und \(I:a\) beide echt 
		\(I\) und sind damit endlich erzeugt. Wenn aber \(I:a)\) erzeugt ist von \(a_i\) und \((I,a)\) erzeugt ist von
		\(a\) und \(b_i\) dann ist \(I\) erzeugt von \(aa_i\) und \(b_i\). Also ist \(I\) prim.
		Angenommen es gibt ein Ideal \(I\) das nicht endlich erzeugt ist.
		Die Vereinigung einer total geordneten Familie nicht-endlich erzeugter Ideale ist nicht endlich erzeugt.
		Also gibt es nach Lemma von Zorn?? ein Ideal maximal mit der Eigenschaft nicht endlich erzeugt zu sein. Nach Teil 1
		ist das dann prim, was ein Widerspruch ist.
	\end{proof}
\end{Satz}
\begin{proof}
	Die Äquivalenz von 1 und 3 ist \cref{Satz:NormalIntegrit} zusammen mit \cref{Satz:DBRNormNoeth1dim}.
	Gelte also 2. Wegen Eindeutigkeit der Primzerlegung ist \(\frakp\neq \frakp^2\) für jedes Primideal \(\frakp\subseteq R\).
	Sei \(x\in \frakp\setminus\frakp^2\) und \(y\in\frakp\). Dann ist \((x,y)=\frakp_1\dots\frakp_r\) und da 
	\((x,y)\subseteq\frakp\) muss \(\frakp_i\subseteq \frakp\) für ein \(i\). Da aber \(x\not\in\frakp^2\) kann 
	das höchstens für ein \(i\) geschehen. OE ist also \(\frakp_1\subseteq \frakp\).
	Also ist \((x,y)R_\frakp=\frakp_1R_\frakp\) prim für jede Wahl von \(y\), also auch für \(y^2\).
	Also ist \(x,y^2R_\frakp\) prim und somit \(y\in (x,y^2)R_\frakp\). Dann gibt es Darstellung 
	\(y=ax+by^2\) in \(R_\frakp\) und somit ist \[(1-by)y=ax\in (x)R_\frakp\] Also ist \(y\in (x)R_\frakp\) und \((x)R_\frakp=\frakp_1R_\frakp=\frakp R_\frakp\).
	Nach Lemma ?? ist jedes primideal also endlich erzeugt und damit ist \(R\) noethersch. Also ist jede lokalisierung ein diskreter Bewertungsring.
	Gelten jetzt 1 und 3. Definiere für ein Ideal \(I\neq 0\) 
	\[\nu_\frakm(I)=\min\{\nu_\frakm(I)\mid a\in I\}\] wobei \(\nu_\frakm\) diskrete Bewertung von \(A_\frakm\) ist.
	Es gilt \(I\subseteq J\iff \nu_\frakm(I)\geq \nu_\frakm(J)\). Denn wenn \(a\in I\) dann gibt es \(b_\frakm\in J\)
	sodas \(\nu_\frakm(a)\geq\nu_\frakm(b)\).
	Also ist \(a\in J_\frakm\) für alle \(m\) also \(a\in J\) nach ????.
	Eine Rechnung zeigt, dass \(\nu(IJ)=\nu(I)+\nu(J)\).
	Es ist \(\nu_\frakm(I)>0 \iff \frakm\supseteq I\) und da \(\frakm/I\) minimales Primideal in \(A/a\)
	und da Ring noethersch ist folgt, dass das nur für endlich viele maximale Ideale \(\frakm\) passieren kann.
	Es ist 
	\[\nu_\frakm(\prod_\frakn \frakn^{\nu_\frakn(I)}=\sum\nu_\frakm(frakn^{\nu_\frakn(I)})=
	\sum_\frakn \nu_\frakn(I)\nu_\frakm(\frakn)=\nu_\frakn(I)\]
	Also \(I=\prod_\frakn \frakn^{\nu_\frakn(I)}\).
\end{proof}
\begin{Lemma}
	Sei \(A\) Dedekindring. Zu jedem Ideal \(I\) gib es ein Ideal \(J\) sodass \(IJ\) ein Hauptideal ist.
\end{Lemma}
\begin{proof}
	OE \(I\neq 0\) und \((0)\neq (x)\subseteq I\).
	Sei \(I=\prod \frakm_i^{e_i}\) und \((x)=\prod \frakm_i^{f_i}\).
	Dann ist \(r_i-n_i\geq 0\) nach obigem.
	Definiere also \(J=\prod \frakm_i^{f_i-e_i}\). Dann ist \(IJ=(x)\).
\end{proof}
\begin{Lemma}
	Sei \(R\) noetherscher Integritätsbereich der Dimension \(1\) mit Quotientenkörper \(K\) und sei \(L/K\) eine endliche
	Erweiterung. Dann ist jeder Ring \(R\subseteq A\subseteq L\) noethersch.
\end{Lemma}
\begin{proof}
	Sei \(I\subseteq A\) ein Ideal ungleich \((0)\) und \(x\in I\setminus\{0\}\). Wähle \(r_nx^n+\dots+r_0=0\) mit \(r_i\in R\).
	und \(r_0,r_n\neq 0\). Dann ist \(r_0\in R\cap I\).
	Also ist \(I/r_0A\subseteq A/r_0A\).
\end{proof}
\begin{Satz}[Krull-Akizuki]
	Sei \(A\) noetherscher Integritätsbereich der Dimension \(1\) mit Quotientenkörper \(K\) und sei \(L/K\) eine endliche
	Erweiterung. Sei \(B\) der ganze Abschluss of von \(A\) in \(L\).
	Dann ist \(B\) ein Dedekindring.
	
\end{Satz}
\begin{proof}
	
\end{proof}
\section{Idealklassengruppe}
\begin{Def} Ein gebrochenes Ideal von einem Dedekindring \(R\) ist ein endlich erzeugter \(R\)-Untermodul von 
	\(K=\Quot(R)\).
	Für zwei gebrochene Ideale \(I,J\) ist \(IJ\) der \(R\)-Modul der erzeugt ist von allen \(xy\).
	
\end{Def}
\begin{Satz} Die Menge der gebrochenen Ideale ungleich Null ist mit diesem Produkt eine abelsche Gruppe \(J_R\).
	
\end{Satz}
\begin{proof}
	Sei \(I\) gebrochenes Ideal erzeugt durch \(\frac{1}{x_1},\dots,\frac{1}{x_n}\).
	Sei \(x=\prod x_i\) das heißt \(xI\subseteq R\) Ideal.
	Sei \(xI=\frakp_1\dots\frakp_r\) Primzerlegung.
	Sei \(\fraka=\frakp_2\dots\frakp_r\). Dann ist \(\frakp\fraka=(x)\) und \(\frakp(\frac 1 x)\fraka=(1)\)
	Also ist \(\frakp_i\) invertierbar für alle \(i\).
	Dann ist \[x\frakp_i^{-1}\dots\frakp_r^{-1}I=(1)\] und somit \(I\) invertierbar.
\end{proof}
\begin{Def} sei \(a\in K^*\). Dann ist \(aR\) gebrochenes Ideal. Gebrochene Ideale von der Form nennt man 
	gebrochene Hauptideale. 
	Wir haben exakte Sequenz
	% https://tikzcd.yichuanshen.de/#N4Igdg9gJgpgziAXAbVABwnAlgFyxMJZABgBpiBdUkANwEMAbAVxiRGJAF9T1Nd9CKAIzkqtRizYAlAHoAqLjxAZseAkQBMo6vWatEIANLzFvVQKIBmbeL1sAUgH0pp5XzWDkAFhu7JBgGEGAAopAEpXFX51FABWXwl9di4xGCgAc3giUAAzACcIAFskMhAcCCQhblyC4sQRMorEDWqQfKKkLUaka1t-EDoAHUHCujQ4coACOhdW9rre8qQvOdrl6iXEWM4KTiA
	
	\begin{tikzfigure} 
		0 \arrow[r] & R^* \arrow[r] & K^* \arrow[r, "a\mapsto aR"] & J_R \arrow[r] & Cl(R) \arrow[r] & 0
	\end{tikzfigure} und der Quotient \(Cl(R)\) heißt Idealklassengruppe.
\end{Def}
\begin{Bsp} sei \(R=\ZZ[\sqrt{-5}\). Es ist \(6=2\cdot 3=(1+\sqrt{-5})(1-\sqrt{-5})\)
	Also ist  \(R\) nicht faktoriell. Da \(-5\equiv 3 \mod 4\) ist, ist \(R=\OO_K\) Ring der ganzen Zahlen in 
	\(K=\QQ[\sqrt{-5}]\). Also ein Dedekindring, der kein Hauptidealring ist.
	Es gibt unter anderen die maximalen Ideale 
	\(\frakp_1=(2,1+\sqrt{-5})\),\(\frakp_2=(3,1+\sqrt{-5})\) und \(\frakp_3=(3,1-\sqrt{-5})\).
	Das Minimalpolynom von \(1+\sqrt{-5}\) ist nämlich \(\mu=(X-1)^2+5=X^2-2X+6\) was irreduzibel ist nach Eisenstein. ???
	also ist \(R=\ZZ[X]/(\mu)\) woibe i \(X\leftrightarrow 1+\sqrt{-5}\) und \(R/\frakp_1\cong \FF_2\) Körper.
	Genauso ist \(R/\frakp_i=\FF_3\) Körper für \(i=2,3\).
	Es gilt \((2)=\frakp_1^2\) und \(1+\sqrt{-5})=\frakp_1\frakp_2\) und \((3)=\frakp_2\frakp_3\) und 
	\((1-\sqrt{-5})=\frakp_1\frakp_3\). Also ist \((6)=\frakp_1\frakp_2\frakp_3\frakp_4\) die Eindeutige Zerlegung.
	
\end{Bsp}

\begin{Lemma}
	Sei \(R\) ein Dedekindring. Es ist äquivalent
	\begin{enumerate}
		\item \(R\) ist ein Hauptidealring
		\item Jedes Primideal ist ein Hauptideal 
		\item \(Cl(R)\) hat nur ein Element.
		\item \(R\) ist faktoriell
	\end{enumerate}
\end{Lemma}
\begin{proof}
	1 nach 2 ist klar. Gelte 1. Sei \(I\) gebrochenes Ideal. Es gibt \(d\in R\) sodass \(dR\subseteq R\) ein Ideal ist.
	Dann ist \(dI=(x)\) und \(I=\frac x d R\) gebrochenes Hauptideal. Rest ist einfach.
\end{proof}
\begin{Satz} Sei \(R\) ein Dedekindring und \(M\) ein endlich erzeugter \(R\)-Modul der Torsionsmodul ist.
	Dann ist \(\Supp(M)\) endlich und \(M\cong \bigoplus_{i\in I}R/\frakm_i^{n_i}\) wobei \(I\) endlich und \(\frakm_i\) maximal.
	
\end{Satz}
\begin{proof}
	Da \(R\) noethersch ist ist der Support endlich nach??
	Betrachte \(\varphi\colon M\to \bigoplus_{\frakp\in\Supp(M)}M_\frakp\). 
	Für \(\frakq\in\Spec(R)\) ist \[M_\frakp\otimes R_\frakq=M\otimes (R_\frakp)_\frakq)=
	\begin{cases} 0 & \frakp\neq \frakq\\ 
		M_\frakp & \frakp=\frakq\\
	\end{cases}\]
	wobei oberes gilt da \(M\) ein Torsionsmodul ist und \(\frakp\cap\frakq=(0)\).
	Somit ist \(\varphi\) ein Isomorphismus.
	Da \(R_\frakp\) Hauptidealring ist folgt mit Struktursatz, dass jeder Torsionsmodul von der Form
	\(\bigoplus_{n\in J}R_p/(\frakp R_\frakp)^nR_\frakp=\bigoplus (R/\frakp^n)_\frakp\) ist.
	Da \(R/\frakp^n\) lokal ist, ist \(R/\frakp^n)_\frakp\cong R/\frakp^n\). Also gilt die Aussage.
\end{proof}