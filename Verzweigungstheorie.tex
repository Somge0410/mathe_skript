\chapter{Hilbert Verzweigungstheorie}
Sei \(A\) ein Dedekindring und \(K=\Quot(A)\). Sei \(L/K\) eine endliche separable Erweiterung und \(B\subseteq L\) der ganze Abschluss von \(A\) in \(L\). Nach ??? ist \(B\) wieder Dedekindring.

\begin{Def}Für Primideal \(\frakp\subseteq \calO_K\) gibt es eindeutige Zerlegung 
	\(\frakp B=\frakq_1^{e_1}\cdots \frakq_r^{e_r}\).
	Es ist \(\frakq_i\cap A=\frakp\) wegen Eindeutigkeit der Primzerlegung.
	Das gibt injektiven Homomoprhismus \(k(\frakp)=A/\frakp\to B/\frakq_i=k(\frakq_i)\).
	\begin{enumerate}
		\item 
		Der Verzweigungsindex von \(\frakq_i\) ist \(e_i\).
		\item Der Trägheitsgrad von \(\frakq_i\) ist \(f_i=[k(\frakq_i)\colon k(\frakp)]\).
		\item \(\frakq_i\) heißt unverzweigt über \(\frakp\), wenn \(e_i=1\).
	\end{enumerate}
	
\end{Def}
\begin{Lemma}
	Sei \(M/L\) eine weitere endliche separable Körpererweiterung und \(C\) der ganze Abschluss von \(A\) in \(M\). Sei \(\frakq_C\) ein maximales Ideal von \(C\)
	und \(\frakq_B=\frakq C\cap B\) und \(\frakq_A=\frakq_B\cap A\).
	Dann gilt 
	\(e_{\frakq_C/\frakq_A}=e_{\frakq_C/\frakq_B}\cdot e_{\frakq_B/\frakq_A}\) und \(f_{\frakq_C/\frakq_A}=f_{\frakq_C/\frakq_B}\cdot f_{\frakq_B/\frakq_A}\)
\end{Lemma}
\begin{proof}
	Sei \(\frakq_AB=\frakp_1^{e_1}\cdots \frakp_r^{e_r}\)
	und \(\frakq_AC=\frakq_1^{e_1}\cdots \frakq_s^{e_s}\)
	und ohne Einschränkung \(\frakq_B=\frakp_1\) und \(\frakq_C=\frakq_1\).
	Es ist
	\[\frakq_AC=(\frakq_AB)C=\frakp_1^{e_1}\cdots \frakp_r^{e_r}C\] ist \(e_1\cdot e_{\frakq_C/\frakq_B}\leq e_{\frakq_C/\frakq_A}\).
	Wenn \(\frakq_C\mid \frakq_iC\) für ein \(i\) dann ist
	\(\frakq_i\subseteq \frakq_C\cap B=\frakq\) also \(\frakq_i=\frakq_B\).
	Also teilt \(\frakq_C\) nur \(\frakq_1=\frakq_B\).
	Somit ist \(e_1\cdot e_{\frakq_C/\frakq_B}= e_{\frakq_C/\frakq_A}\).
	Die Aussage über \(f\) ist einfach Multiplikation der Körpergrade ????.
\end{proof}
\begin{Bsp}
	Es sei \(K_n=\QQ[\sqrt[n]{p}]\cong \QQ[X]/(X^n-p)\) für eine Primzahl \(p\). Dann ist \(\frakq=(\sqrt[n]{p})\) ein maximales Ideal und 
	\(p\calO_K=\frakq^n\). Also ist \(e=n=[K_n\colon \QQ]\).
	
\end{Bsp}
\begin{Bsp}
	Wenn \(K\) Zahlkörper ist und \(\frakp\subseteq\calO_K\) maximales Ideal. Sei \(m\in\NN\) und \((p)=\frakp\cap \ZZ\) und \(n=m\cdot e_{\frakp/(p)}\).
	Sei \(L=K_n\cdot K\) und \(\frakq\) maximales Ideal von \(\calO_L\) mit  \(\frakq\cap \calO_K=\frakp\).
	Sei \(\frakp'=\frakq\cap \calO_{K_n}\).
	Dann ist nach Bsp ???
	\begin{align*}
		e_{\frakq/\frakp}e_{\fraq/(p)}&=e_{\frakq/\frakp'}e_{\frakp'/(p)}\\
		&= e_{\frakq/\frakp'}\cdot n\\
		&= m e_{\frakq/\frakp'}\cdot e_{\frakp/(p)}
	\end{align*}
	Also ist \(e_{\frakq/\frakp}=m e_{\frakq/\frakp'}\geq m\)
	Somit ist \(L/K\) eine Körpererwetierung sodass jedes Ideal über \(\frakp\) mindestens Verweigungsindex \(m\) hat.
\end{Bsp}





